\documentclass[b4paper, landscape, dvipdfmx]{jsarticle}
%----- 必要なパッケージ -----
\usepackage{fancybox,ascmac,otf,ulem}
\usepackage{amssymb, amsthm}
\usepackage[leqno]{amsmath}
\usepackage{geometry}
\usepackage{multicol}
\usepackage{tcolorbox}
\usepackage{xcolor}
\usepackage{fancyhdr}
\usepackage{tikz}

% shadowsライブラリ
\usetikzlibrary{
    positioning,
    arrows.meta,
    calc,
    shadows,
    shadows.blur,
    intersections
}

\tcbuselibrary{skins, breakable, theorems}
\usepackage{enumitem}
\setlist[enumerate,1]{label=(\arabic*)}
\setlist[itemize]{leftmargin=*}
\newcommand{\ds}{\displaystyle}

%----- レイアウト設定 -----
\geometry{
  left=15mm,
  right=15mm,
  top=20mm,
  bottom=15mm,
  headheight=25pt
}

%----- 数式環境の上下の余白調整 -----
\AtBeginDocument{
  \setlength{\abovedisplayskip}{5pt}
  \setlength{\belowdisplayskip}{5pt}
  \setlength{\abovedisplayshortskip}{0pt}
  \setlength{\belowdisplayshortskip}{3pt}
}

%===========================================================
%  デザイン設定
%===========================================================

%--- 色の定義 ---
\definecolor{printBlue}{RGB}{0, 50, 100}     % 濃紺
\definecolor{printRed}{RGB}{140, 20, 20}     % 濃エンジ
\definecolor{printTeal}{RGB}{0, 60, 60}      % 濃い青緑

%--- 共通スタイル定義 ---
\tcbset{
    chartbox/.style={
        enhanced,
        fonttitle=\sffamily\bfseries,
        boxrule=1pt,
        arc=2pt,
        top=1.0em,
        nobeforeafter,
        enlarge left by=-2mm,
        enlarge right by=-2mm,
        drop fuzzy shadow,
        colback=white,
        attach boxed title to top left={xshift=10pt, yshift*=-\tcboxedtitleheight/2},
        boxed title style={frame hidden, sharp corners, rounded corners=southeast, arc=3pt}
    }
}

% 各種ボックス環境定義
\newtcolorbox{any}[1]{
    enlarge left by=0mm, enlarge right by=0mm,
    enhanced, frame hidden, colback=white, title={#1},
    attach boxed title to top left={xshift=0mm, yshift=0mm},
    coltitle=white, fonttitle=\bfseries\sffamily,
    boxed title style={
        colback=black!80, frame hidden, arc=4pt, outer arc=4pt,
        sharp corners=south, boxrule=0pt,
        top=1mm, bottom=1mm, left=3mm, right=3mm
    },
    underlay boxed title={
        \draw[thick, black!80] (title.south west) -- (title.south west-|frame.east);
    },
    breakable, top=5mm, left=2mm, right=2mm, bottom=0mm,
    before skip=1em, after skip=1em,
    segmentation style={draw=black!40, dashed}
}

\newenvironment{eg}[1]{
\begin{tcolorbox}[
    chartbox,
    colframe=printBlue,
    coltitle=white,
    title=\textbf{例題 #1},
    boxed title style={colback=printBlue},
    segmentation style={draw=printBlue, line width=0.5pt, dashed}
]}
{\end{tcolorbox}}

\newenvironment{prac}[1]{
\begin{tcolorbox}[
    chartbox,
    colframe=printRed,
    coltitle=white,
    title=\textbf{練習 #1},
    boxed title style={colback=printRed}
]}
{\end{tcolorbox}}

\newenvironment{answer}[1][height fill]{
    \begin{tcolorbox}[
        enhanced,
        title={Memo / Answer},
        colframe=black!80,
        colback=white,
        coltitle=black!60,
        fonttitle=\sffamily\bfseries,
        attach boxed title to top left={xshift=5mm, yshift*=-\tcboxedtitleheight/2},
        boxed title style={frame hidden, colback=white},
        boxrule=1pt,
        arc=1pt,
        nobeforeafter,
        enlarge left by=2mm, 
        enlarge right by=2mm, 
        height fill,
        segmentation style={draw=black!20, solid},
        underlay={
            \begin{tcbclipinterior}
                \draw[step=5mm, black!5, ultra thin] (interior.south west) grid (interior.north east);
            \end{tcbclipinterior}
        }, 
        #1
    ]}
{ \end{tcolorbox}}

\newenvironment{proofbox}[1][height fill]{
    \begin{tcolorbox}[
        enhanced,
        title={Proof},
        colframe=black!80,
        colback=white,
        coltitle=black!60,
        fonttitle=\sffamily\bfseries,
        attach boxed title to top left={xshift=5mm, yshift*=-\tcboxedtitleheight/2},
        boxed title style={frame hidden, colback=white},
        boxrule=1pt,
        arc=1pt,
        nobeforeafter,
        width=\linewidth,
        underlay={
            \begin{tcbclipinterior}
                \draw[step=5mm, black!5, ultra thin] (interior.south west) grid (interior.north east);
            \end{tcbclipinterior}
        },
        #1
    ]}
{ \end{tcolorbox}}

%----- 段組の設定 -----
\setlength{\columnsep}{15mm}
\setlength{\columnseprule}{0.4pt}
\renewcommand{\columnseprulecolor}{\color{black!30}}

%----- ヘッダーの設定 -----
\pagestyle{fancy}
\fancyhf{}

% ヘッダーデザイン
\fancyhead[C]{%
    \begin{tikzpicture}[remember picture, overlay]
        \node[anchor=north west, fill=printBlue, minimum width=\paperwidth, minimum height=5pt] at (current page.north west) {};
    \end{tikzpicture}
}
\fancyhead[L]{\small \textcolor{black!90}{数学\ajRoman{2} $>$ 第3章 式と証明 $>$ 第15回--\textbf{発展:コーシー・シュワルツの不等式}}}
\fancyhead[R]{\small 年 \hspace{1cm} 組 \hspace{1cm} 番 \quad 氏名 \hspace{6cm}}
\renewcommand{\headrulewidth}{0pt}

\begin{document}

\begin{multicols*}{2}

%===========================================================
% 左カラム: 公式の導入と証明
%===========================================================

{\large \textbf{1. コーシー・シュワルツの不等式}}

\begin{any}{「2乗の積」は「積の2乗」よりデカい}
相加・相乗平均と並ぶ, 不等式のスーパースターを紹介する.
これは「ベクトルの内積」とも深く関係している美しい不等式である.

\begin{center}
\begin{tcolorbox}[colframe=printRed, colback=white, boxrule=1.5pt, arc=0pt]
    $\boldsymbol{a, b, x, y}$ が実数のとき,
    \[ \boldsymbol{(a^2+b^2)(x^2+y^2) \geqq (ax+by)^2} \]
    (等号成立は $\boldsymbol{ay=bx}$ のとき)
\end{tcolorbox}
\end{center}
\end{any}

\begin{eg}{1(不等式の証明)}
コーシー・シュワルツの不等式を証明せよ.
\end{eg}

\begin{proofbox}
% $(\text{左辺}) - (\text{右辺})$
% $= (a^2+b^2)(x^2+y^2) - (ax+by)^2$ \\
% $= (a^2x^2 + a^2y^2 + b^2x^2 + b^2y^2) - (a^2x^2 + 2abxy + b^2y^2)$ \\
% $= a^2y^2 - 2abxy + b^2x^2$ \\
% $= (ay - bx)^2$

% ここで, $a, b, x, y$ は実数であるから, $(ay-bx)^2 \geqq 0$.
% よって, $(\text{左辺}) \geqq (\text{右辺})$.
% 等号成立は $ay-bx=0$, すなわち $ay=bx$ のとき. \qed
\end{proofbox}

% \begin{any}{Reference: ベクトルとの関係}
% $\vec{u} = (a, b), \ \vec{v} = (x, y)$ とすると, 不等式は $|\vec{u}|^2 |\vec{v}|^2 \geqq (\vec{u} \cdot \vec{v})^2$ を意味する.
% これは $-1 \leqq \cos\theta \leqq 1$ ($|\vec{u}\cdot\vec{v}| \leqq |\vec{u}||\vec{v}|$) と本質的に同じである.
% \end{any}

\columnbreak

%===========================================================
% 右カラム: 応用問題
%===========================================================

{\large \textbf{2. 最大・最小問題への応用}}

\begin{any}{2乗の和がわかっているとき}
この不等式は, 「$x^2+y^2=k$ のとき $ax+by$ の範囲を求めよ」といった問題で威力を発揮する.
\end{any}

\begin{eg}{2(最大・最小)}
$x^2 + y^2 = 4$ のとき, $3x + 4y$ の最大値と最小値を求めよ.
\end{eg}

\begin{proofbox}
% コーシー・シュワルツの不等式において,
% $a=3, \ b=4$ と考えると,
% \[ (3^2+4^2)(x^2+y^2) \geqq (3x+4y)^2 \]
% 数値を代入すると,
% \[ (9+16) \cdot 4 \geqq (3x+4y)^2 \]
% \[ 100 \geqq (3x+4y)^2 \]
% よって,
% \[ -10 \leqq 3x+4y \leqq 10 \]
% したがって, 最大値 $10$, 最小値 $-10$.
\end{proofbox}

\begin{any}{3変数の場合(拡張)}
この不等式は変数の数が増えても成り立つ.
\[ (a^2+b^2+c^2)(x^2+y^2+z^2) \geqq (ax+by+cz)^2 \]
\end{any}

\begin{eg}{3(3変数の最小値)}
$a>0, b>0, c>0$ とする. $\ds \left( a+b+c \right) \left( \frac{1}{a}+\frac{1}{b}+\frac{1}{c} \right)$ の最小値を求めよ.
\end{eg}

\begin{proofbox}
% 3変数のコーシー・シュワルツの不等式において,
% $( \sqrt{a}, \sqrt{b}, \sqrt{c} )$ と $( \frac{1}{\sqrt{a}}, \frac{1}{\sqrt{b}}, \frac{1}{\sqrt{c}} )$ のペアで考える.
% \[ \{(\sqrt{a})^2+(\sqrt{b})^2+(\sqrt{c})^2\} \{(\frac{1}{\sqrt{a}})^2+(\frac{1}{\sqrt{b}})^2+(\frac{1}{\sqrt{c}})^2\} \]
% \[ \geqq \left( \sqrt{a}\cdot\frac{1}{\sqrt{a}} + \sqrt{b}\cdot\frac{1}{\sqrt{b}} + \sqrt{c}\cdot\frac{1}{\sqrt{c}} \right)^2 \]
% \[ = (1+1+1)^2 = 9 \]
% よって最小値は $9$.
\end{proofbox}

\end{multicols*}

\newpage

%===========================================================
% 裏面: 演習問題
%===========================================================

\begin{multicols*}{2}

{\large \textbf{確認テスト}}

\begin{prac}{A1(基本適用)}
$x^2 + y^2 = 1$ のとき, $x + 2y$ のとりうる値の範囲を求めよ.
\end{prac}

\begin{proofbox}
$a=1, b=2$ としてコーシー・シュワルツの不等式を適用する.
\[ (1^2+2^2)(x^2+y^2) \geqq (x+2y)^2 \]
\[ 5 \cdot 1 \geqq (x+2y)^2 \]
\[ (x+2y)^2 \leqq 5 \]
\[ -\sqrt{5} \leqq x+2y \leqq \sqrt{5} \]
\end{proofbox}


\begin{prac}{A2(証明)}
$x^2 + y^2 + z^2 = 1$ のとき, $x+y+z$ の最大値が $\sqrt{3}$ であることを証明せよ.
\end{prac}

\begin{proofbox}
3変数のコーシー・シュワルツの不等式で $a=b=c=1$ とする.
\[ (1^2+1^2+1^2)(x^2+y^2+z^2) \geqq (x+y+z)^2 \]
\[ 3 \cdot 1 \geqq (x+y+z)^2 \]
\[ -\sqrt{3} \leqq x+y+z \leqq \sqrt{3} \]
よって最大値は $\sqrt{3}$.
\end{proofbox}

\columnbreak

\begin{prac}{B1(条件付きの最小値)}
$3x + 4y = 5$ のとき, $x^2 + y^2$ の最小値を求めよ.
\end{prac}

\begin{any}{Hint}
コーシー・シュワルツの不等式 $(3^2+4^2)(x^2+y^2) \geqq (3x+4y)^2$ を利用し, 不等式の向きに注意して解く.
\end{any}

\begin{answer}
% (9+16)(x^2+y^2) >= (5)^2
% 25(x^2+y^2) >= 25
% x^2+y^2 >= 1
% 最小値 1.
\vspace{6cm}
\end{answer}

\begin{prac}{B2(数学史コラム:3次方程式の解の公式)}
16世紀イタリア, 数学者のタルタリアとカルダノは3次方程式の解の公式を発見した. しかし, その公式を使うと $\sqrt{-1}$ (虚数) が途中式に出てきてしまう場合があった.
彼らは当時「不合理な数」としてこれを嫌ったが, この「虚数」を認めることで, 数学の世界は飛躍的に広がった.
今日の授業で扱った $x^2+y^2$ のような2乗の和も, 複素数平面上では「原点からの距離の2乗」として意味を持つ.

\vspace{0.5em}
\textbf{問い:} 複素数 $z = 3+4i$ について, 原点からの距離 $|z| = \sqrt{3^2+4^2}$ を求めよ.
\end{prac}

\begin{answer}
$|z| = \sqrt{9+16} = \sqrt{25} = 5$.
\end{answer}

\end{multicols*}

\end{document}