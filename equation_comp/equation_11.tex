\documentclass[b4paper, landscape, dvipdfmx]{jsarticle}
%----- 必要なパッケージ -----
\usepackage{fancybox,ascmac,otf,ulem}
\usepackage{amssymb, amsthm}
\usepackage[leqno]{amsmath}
\usepackage{geometry}
\usepackage{multicol}
\usepackage{tcolorbox}
\usepackage{xcolor}
\usepackage{fancyhdr}
\usepackage{tikz}

% shadowsライブラリ
\usetikzlibrary{
    positioning,
    arrows.meta,
    calc,
    shadows,
    shadows.blur,
    intersections
}

\tcbuselibrary{skins, breakable, theorems}
\usepackage{enumitem}
\setlist[enumerate,1]{label=(\arabic*)}
\setlist[itemize]{leftmargin=*}
\newcommand{\ds}{\displaystyle}

%----- レイアウト設定 -----
\geometry{
  left=15mm,
  right=15mm,
  top=20mm,
  bottom=15mm,
  headheight=25pt
}

%----- 数式環境の上下の余白調整 -----
\AtBeginDocument{
  \setlength{\abovedisplayskip}{5pt}
  \setlength{\belowdisplayskip}{5pt}
  \setlength{\abovedisplayshortskip}{0pt}
  \setlength{\belowdisplayshortskip}{3pt}
}

%===========================================================
%  デザイン設定
%===========================================================

%--- 色の定義 ---
\definecolor{printBlue}{RGB}{0, 50, 100}     % 濃紺
\definecolor{printRed}{RGB}{140, 20, 20}     % 濃エンジ
\definecolor{printTeal}{RGB}{0, 60, 60}      % 濃い青緑

%--- 共通スタイル定義 ---
\tcbset{
    chartbox/.style={
        enhanced,
        fonttitle=\sffamily\bfseries,
        boxrule=1pt,
        arc=2pt,
        top=1.0em,
        nobeforeafter,
        enlarge left by=-2mm,
        enlarge right by=-2mm,
        drop fuzzy shadow,
        colback=white,
        attach boxed title to top left={xshift=10pt, yshift*=-\tcboxedtitleheight/2},
        boxed title style={frame hidden, sharp corners, rounded corners=southeast, arc=3pt}
    }
}

% 各種ボックス環境定義
\newtcolorbox{any}[1]{
    enlarge left by=0mm, enlarge right by=0mm,
    enhanced, frame hidden, colback=white, title={#1},
    attach boxed title to top left={xshift=0mm, yshift=0mm},
    coltitle=white, fonttitle=\bfseries\sffamily,
    boxed title style={
        colback=black!80, frame hidden, arc=4pt, outer arc=4pt,
        sharp corners=south, boxrule=0pt,
        top=1mm, bottom=1mm, left=3mm, right=3mm
    },
    underlay boxed title={
        \draw[thick, black!80] (title.south west) -- (title.south west-|frame.east);
    },
    breakable, top=5mm, left=2mm, right=2mm, bottom=0mm,
    before skip=1em, after skip=1em,
    segmentation style={draw=black!40, dashed}
}


\newenvironment{eg}[1]{
\begin{tcolorbox}[
    chartbox,
    colframe=printBlue,
    coltitle=white,
    title=\textbf{例題 #1},
    boxed title style={colback=printBlue},
    segmentation style={draw=printBlue, line width=0.5pt, dashed}
]}
{\end{tcolorbox}}

\newenvironment{prac}[1]{
\begin{tcolorbox}[
    chartbox,
    colframe=printRed,
    coltitle=white,
    title=\textbf{練習 #1},
    boxed title style={colback=printRed}
]}
{\end{tcolorbox}}

\newenvironment{answer}[1][height fill]{
    \begin{tcolorbox}[
        enhanced,
        title={Memo / Answer},
        colframe=black!80,
        colback=white,
        coltitle=black!60,
        fonttitle=\sffamily\bfseries,
        attach boxed title to top left={xshift=5mm, yshift*=-\tcboxedtitleheight/2},
        boxed title style={frame hidden, colback=white},
        boxrule=1pt,
        arc=1pt,
        nobeforeafter,
        enlarge left by=2mm, 
        enlarge right by=2mm, 
        height fill,
        segmentation style={draw=black!20, solid},
        underlay={
            \begin{tcbclipinterior}
                \draw[step=5mm, black!5, ultra thin] (interior.south west) grid (interior.north east);
            \end{tcbclipinterior}
        }, 
        #1
    ]}
{ \end{tcolorbox}}

\newenvironment{proofbox}[1][height fill]{
    \begin{tcolorbox}[
        enhanced,
        title={Proof},
        colframe=black!80,
        colback=white,
        coltitle=black!60,
        fonttitle=\sffamily\bfseries,
        attach boxed title to top left={xshift=5mm, yshift*=-\tcboxedtitleheight/2},
        boxed title style={frame hidden, colback=white},
        boxrule=1pt,
        arc=1pt,
        nobeforeafter,
        width=\linewidth,
        underlay={
            \begin{tcbclipinterior}
                \draw[step=5mm, black!5, ultra thin] (interior.south west) grid (interior.north east);
            \end{tcbclipinterior}
        },
        #1
    ]}
{ \end{tcolorbox}}

%----- 段組の設定 -----
\setlength{\columnsep}{15mm}
\setlength{\columnseprule}{0.4pt}
\renewcommand{\columnseprulecolor}{\color{black!30}}

%----- ヘッダーの設定 -----
\pagestyle{fancy}
\fancyhf{}

% ヘッダーデザイン
\fancyhead[C]{%
    \begin{tikzpicture}[remember picture, overlay]
        \node[anchor=north west, fill=printBlue, minimum width=\paperwidth, minimum height=5pt] at (current page.north west) {};
    \end{tikzpicture}
}
\fancyhead[L]{\small \textcolor{black!90}{数学\ajRoman{2} $>$ 第3章 式と証明 $>$ 第11回--\textbf{等式の証明}}}
\fancyhead[R]{\small 年 \hspace{1cm} 組 \hspace{1cm} 番 \quad 氏名 \hspace{6cm}}
\renewcommand{\headrulewidth}{0pt}

\begin{document}

\begin{multicols*}{2}

%===========================================================
% 左カラム: 証明の基本方針
%===========================================================

{\large \textbf{1. 証明の作法}}

\begin{any}{等式 $A=B$ を証明する3つの方法}
「等しい」ことを論理的に説明するには, 主に以下の3つのアプローチがある.
\begin{enumerate}
    \item \textbf{一方変形型}: 複雑な方の辺を変形して, 他方の辺を導く.
    \[ A = \cdots = \cdots = B \]
    \item \textbf{両辺変形型}: 左辺と右辺をそれぞれ変形し, 同じ式になることを示す.
    \[ A = \cdots = C, \quad B = \cdots = C \quad \therefore A=B \]
    \item \textbf{引き算型}: $A-B$ を計算し, $0$ になることを示す.
    \[ A - B = \cdots = 0 \quad \therefore A=B \]
\end{enumerate}

【超重要】 示したいしきを先に書いてはいけない!!
例えば「$a^3+b^3=(a+b)^3-3ab(a+b)$を示せ」という問題の解答の1行目に,
\[
a^3+b^3=(a+b)^3-3ab(a+b)
\]
と書いたら0点である. 
\end{any}

\begin{eg}{1(等式の証明・基本)}
次の等式を証明せよ.
\[ a^3+b^3=(a+b)^3-3ab(a+b)\]
\end{eg}

\begin{proofbox}

\end{proofbox}



\columnbreak

%===========================================================
% 右カラム: 条件付きの等式
%===========================================================

{\large \textbf{2. 条件付きの等式の証明}}

\begin{any}{文字を減らす}
「$a+b+c=0$ のとき」のような条件がある場合, それを利用して\textbf{文字を消去}し, 変数を減らして証明するのが基本である.
\[ a+b+c=0 \iff c = -(a+b) \]
これを代入して $c$ を消去すれば, $a, b$ だけの式になり計算しやすくなる.
\end{any}

\begin{eg}{2(条件付き等式)}
$a+b+c=0$ のとき, 次の等式を証明せよ.
\[ a^3+b^3+c^3=3abc \]
\end{eg}

\begin{proofbox}

\end{proofbox}

{\large \textbf{3. 比例式}}

\begin{any}{「$\boldsymbol{=k}$」とおく}
比例式 $\ds \frac{a}{b} = \frac{c}{d}$ の証明は, 値を $k$ とおいて
\[ a=bk, \quad c=dk \]
と表し, 文字を $b, d, k$ に統一して計算する.
\end{any}

\begin{eg}{3(比例式の証明)}
$\ds \frac{a}{b} = \frac{c}{d}$ のとき, 等式 $\ds \frac{a+c}{b+d} = \frac{a-c}{b-d}$ を証明せよ.
\end{eg}

\begin{answer}
% 解説
% a=bk, c=dk とおく
% 左辺 = (bk+dk)/(b+d) = k(b+d)/(b+d) = k
% 右辺 = bk/b = k
% よって左辺=右辺
\vspace{5cm}
\end{answer}

\end{multicols*}

\newpage

%===========================================================
% 裏面: 演習問題
%===========================================================

\begin{multicols*}{2}

{\large \textbf{確認テスト}}

\begin{prac}{A1(恒等式の証明)}
次の等式を証明せよ.
\[ (a+b)^2 + (a-b)^2 = 2(a^2+b^2) \]
\end{prac}

\begin{proofbox}
\vspace{4cm}
\end{proofbox}


\begin{prac}{A2(条件付き等式)}
$a+b+c=0$ のとき, 次の等式を証明せよ.
\[ (a+b)(b+c)(c+a) + abc = 0 \]
\end{prac}

\begin{any}{Hint}
$a+b=-c, \quad b+c=-a, \quad c+a=-b$ と変形して代入すると早い.
\end{any}

\begin{proofbox}
\vspace{5cm}
\end{proofbox}

\columnbreak

\begin{prac}{B1(比例式)}
$\ds \frac{a}{b} = \frac{c}{d}$ のとき, 次の等式を証明せよ.
\[ \frac{a^2+c^2}{b^2+d^2} = \frac{ac}{bd} \]
\end{prac}

\begin{proofbox}
\vspace{6cm}
\end{proofbox}

\begin{prac}{B2(少なくとも1つは...)}
$x+y+z=a, \quad a(xy+yz+zx)=xyz$ が成り立つとき, $x, y, z$ のうち少なくとも1つは $a$ であることを証明せよ.
\end{prac}

\begin{any}{方針}
「少なくとも1つは $a$」$\iff$ 「$(x-a)(y-a)(z-a) = 0$」
この式の左辺を展開し, 条件式を代入して $0$ になることを示せばよい.
\end{any}

\begin{proofbox}
\end{proofbox}

\end{multicols*}

\end{document}