\documentclass[b4paper, landscape, dvipdfmx]{jsarticle}
%----- 必要なパッケージ -----
\usepackage{fancybox,ascmac,otf,ulem}
\usepackage{amssymb, amsthm}
\usepackage[leqno]{amsmath}
\usepackage{wrapfig}
\usepackage{geometry}
\usepackage{multicol}
\usepackage{tcolorbox}
\usepackage{xcolor}
\usepackage{fancyhdr}
\usepackage{tikz}

% shadowsライブラリ
\usetikzlibrary{
    positioning,
    arrows.meta,
    calc,
    shadows,
    shadows.blur,
    intersections
}

\tcbuselibrary{skins, breakable, theorems}
\usepackage{enumitem}
\setlist[enumerate,1]{label=(\arabic*)}
\setlist[itemize]{leftmargin=*}
\newcommand{\ds}{\displaystyle}

%----- レイアウト設定 -----
\geometry{
  left=15mm,
  right=15mm,
  top=20mm,
  bottom=15mm,
  headheight=25pt
}

%----- 数式環境の上下の余白調整 -----
\AtBeginDocument{
  \setlength{\abovedisplayskip}{5pt}
  \setlength{\belowdisplayskip}{5pt}
  \setlength{\abovedisplayshortskip}{0pt}
  \setlength{\belowdisplayshortskip}{3pt}
}

%===========================================================
%  デザイン設定
%===========================================================

%--- 色の定義 ---
\definecolor{printBlue}{RGB}{0, 50, 100}     % 濃紺
\definecolor{printRed}{RGB}{140, 20, 20}     % 濃エンジ
\definecolor{printTeal}{RGB}{0, 60, 60}      % 濃い青緑
\definecolor{gridColor}{gray}{0.75}          % 解答欄の方眼

%--- 共通スタイル定義 ---
\tcbset{
    chartbox/.style={
        enhanced,
        fonttitle=\sffamily\bfseries,
        boxrule=1pt,
        arc=2pt,
        top=1.0em,
        nobeforeafter,
        enlarge left by=-2mm,
        enlarge right by=-2mm,
        drop fuzzy shadow,
        colback=white,
        attach boxed title to top left={xshift=10pt, yshift*=-\tcboxedtitleheight/2},
        boxed title style={frame hidden, sharp corners, rounded corners=southeast, arc=3pt}
    }
}

% 各種ボックス環境定義
\newenvironment{overall}[1]{
\begin{tcolorbox}[
    chartbox,
    colframe=printTeal,
    coltitle=white,
    title=\textbf{全体課題 #1},
    boxed title style={colback=printTeal},
]}
{\end{tcolorbox}}

\newtcolorbox{any}[1]{
    enlarge left by=0mm, enlarge right by=0mm,
    enhanced, frame hidden, colback=white, title={#1},
    attach boxed title to top left={xshift=0mm, yshift=0mm},
    coltitle=white, fonttitle=\bfseries\sffamily,
    boxed title style={
        colback=black!80, frame hidden, arc=4pt, outer arc=4pt,
        sharp corners=south, boxrule=0pt,
        top=1mm, bottom=1mm, left=3mm, right=3mm
    },
    underlay boxed title={
        \draw[thick, black!80] (title.south west) -- (title.south west-|frame.east);
    },
    breakable, top=5mm, left=2mm, right=2mm, bottom=0mm,
    before skip=1em, after skip=1em,
    segmentation style={draw=black!40, dashed}
}


\newenvironment{eg}[1]{
\begin{tcolorbox}[
    chartbox,
    colframe=printBlue,
    coltitle=white,
    title=\textbf{例題 #1},
    boxed title style={colback=printBlue},
    segmentation style={draw=printBlue, line width=0.5pt, dashed}
]}
{\end{tcolorbox}}

\newenvironment{prac}[1]{
\begin{tcolorbox}[
    chartbox,
    colframe=printRed,
    coltitle=white,
    title=\textbf{練習 #1},
    boxed title style={colback=printRed}
]}
{\end{tcolorbox}}

\newenvironment{answer}[1][height fill]{
    \begin{tcolorbox}[
        enhanced,
        title={Memo / Answer},
        colframe=black!80,
        colback=white,
        coltitle=black!60,
        fonttitle=\sffamily\bfseries,
        attach boxed title to top left={xshift=5mm, yshift*=-\tcboxedtitleheight/2},
        boxed title style={frame hidden, colback=white},
        boxrule=1pt,
        arc=1pt,
        nobeforeafter,
        enlarge left by=2mm, 
        enlarge right by=2mm, 
        height fill,
        segmentation style={draw=black!20, solid},
        underlay={
            \begin{tcbclipinterior}
                \draw[step=5mm, black!5, ultra thin] (interior.south west) grid (interior.north east);
            \end{tcbclipinterior}
        }, 
        #1
    ]}
{ \end{tcolorbox}}

%----- 段組の設定 -----
\setlength{\columnsep}{15mm}
\setlength{\columnseprule}{0.4pt}
\renewcommand{\columnseprulecolor}{\color{black!30}}

%----- ヘッダーの設定 -----
\pagestyle{fancy}
\fancyhf{}

% ヘッダーデザイン
\fancyhead[C]{%
    \begin{tikzpicture}[remember picture, overlay]
        \node[anchor=north west, fill=printBlue, minimum width=\paperwidth, minimum height=5pt] at (current page.north west) {};
    \end{tikzpicture}
}
\fancyhead[L]{\small \textcolor{black!90}{数学\ajRoman{2} $>$ 第2章 複素数と方程式 $>$ 第9回--\textbf{1の3乗根と複素数}}}
\fancyhead[R]{\small 年 \hspace{1cm} 組 \hspace{1cm} 番 \quad 氏名 \hspace{6cm}}
\renewcommand{\headrulewidth}{0pt}

\begin{document}

\begin{multicols*}{2}

%===========================================================
% 左カラム: 複素数の範囲での因数分解
%===========================================================

{\large \textbf{1. 複素数の範囲での因数分解}}

\begin{any}{「解く」ことと「因数分解」は同じ}
2次方程式 $ax^2+bx+c=0$ の2つの解を $\alpha, \beta$ とすると,
\[ ax^2+bx+c = a(x-\alpha)(x-\beta) \]
と因数分解できる.
これを利用すれば, 実数の範囲では因数分解できなかった式も, \textbf{複素数の範囲}まで拡張すれば, 必ず1次式の積に分解できる.
\end{any}

\begin{eg}{1(複素数の範囲での因数分解)}
次の式を, 複素数の範囲で因数分解せよ.
\begin{enumerate}
    \item $x^2 + 4$
    \item $x^2 - 2x + 4$
\end{enumerate}
\end{eg}

\begin{answer}

\end{answer}

\columnbreak

%===========================================================
% 右カラム: 1の3乗根オメガ
%===========================================================

{\large \textbf{2. 1の3乗根 $\boldsymbol{\omega}$}}

\begin{any}{3乗して1になる数}
方程式 $x^3 = 1$ の解を求めよう.
$x^3 - 1 = 0 \iff (x-1)(x^2+x+1) = 0$
よって解は $x = 1, \ \ds\frac{-1\pm\sqrt{3}i}{2}$.
このうち, 虚数解の一つを $\boldsymbol{\omega}$ (オメガ) と表す.
\end{any}

\begin{any}{最強の性質 2選}
$\omega$ は $x^3=1$ の解であり, $x^2+x+1=0$ の解でもあるため, 次の等式が成り立つ.
\begin{center}
\begin{tcolorbox}[colframe=printRed, colback=white, boxrule=1.5pt, arc=0pt]
    \begin{enumerate}
        \item $\boldsymbol{\omega^3 = 1}$ \quad (次数下げに使える!)
        \item $\boldsymbol{\omega^2 + \omega + 1 = 0}$ \quad (3項の和は0!)
    \end{enumerate}
\end{tcolorbox}
\end{center}
\end{any}

\begin{eg}{2($\omega$ の計算)}
$1$ の $3$ 乗根のうち虚数であるものの $1$ つを $\omega$ とするとき, 次の式の値を求めよ.
\begin{enumerate}
    \item $\omega^5$
    \item $\omega^4 + \omega^2 + 1$
\end{enumerate}
\end{eg}

\begin{answer}
% 解説
% (1) ω^5 = ω^3 * ω^2 = 1 * ω^2 = ω^2
% (2) ω^4 = ω. 与式 = ω + ω^2 + 1 = 0
\vspace{6cm}
\end{answer}

\columnbreak

%===========================================================
% 右カラム: 3数から方程式を作る
%===========================================================

{\large \textbf{2. 3数を解とする方程式}}

\begin{any}{逆の発想}
3つの数 $\alpha, \beta, \gamma$ を解にもつ3次方程式の一つは,
\[ (x-\alpha)(x-\beta)(x-\gamma) = 0 \]
展開すると,
\[ x^3 - (\alpha+\beta+\gamma)x^2 + (\alpha\beta+\beta\gamma+\gamma\alpha)x - \alpha\beta\gamma = 0 \]
つまり, \textbf{「和」「2つずつの積の和」「積」}が分かれば方程式を作れる.
\end{any}

\begin{eg}{2(方程式の作成)}
3次方程式 $x^3 + 2x^2 + x + 3 = 0$ の3つの解を $\alpha, \beta, \gamma$ とするとき,
$\alpha+1, \beta+1, \gamma+1$ を解とする3次方程式を作れ. ただし, $x^3$ の係数は $1$ とする.
\end{eg}

\begin{answer}
% 解と係数の関係より
% α+β+γ = -2, ...
% 新しい解の和: (α+1)+(β+1)+(γ+1) = (α+β+γ)+3 = 1
% などを計算していく
\vspace{8cm}
\end{answer}

\end{multicols*}

\newpage

%===========================================================
% 裏面: 演習問題
%===========================================================

\begin{multicols*}{2}

{\large \textbf{確認テスト}}

\begin{prac}{A1(複素数の範囲で因数分解)}
次の式を, 複素数の範囲で因数分解せよ.
\begin{enumerate}
    \item $x^2 + 9$
    \item $x^2 - 4x + 5$
    \item $2x^2 + 2x + 1$ (係数2を忘れないように!)
\end{enumerate}
\end{prac}

\begin{answer}
% (1) (x+3i)(x-3i)
% (2) 解 x=2±i -> (x-2-i)(x-2+i)
% (3) 解 x=(-1±i)/2 -> 2(x - (-1+i)/2)(x - (-1-i)/2)
\vspace{6cm}
\end{answer}


\begin{prac}{A2($\omega$ の基本)}
$1$ の $3$ 乗根のうち虚数であるものの $1$ つを $\omega$ とするとき, 次の式の値を求めよ.
\begin{enumerate}
    \item $\omega^{100}$
    \item $\omega^6 + \omega^3 + 1$
\end{enumerate}
\end{prac}

\begin{answer}
% (1) 100 = 3*33 + 1 -> (ω^3)^33 * ω = ω
% (2) 1 + 1 + 1 = 3
\vspace{6cm}
\end{answer}

\columnbreak

\begin{prac}{B1($\omega$ の応用)}
$1$ の $3$ 乗根のうち虚数であるものの $1$ つを $\omega$ とするとき, 次の式の値を求めよ.
\begin{enumerate}
    \item $\ds \omega + \frac{1}{\omega}$
    \item $(\omega + 1)^{10}$
\end{enumerate}
\end{prac}

\begin{any}{Hint}
(2) カッコの中身 $\omega+1$ を, 性質 $\omega^2+\omega+1=0$ を使って書き換えてみよう. ($\omega+1 = -\omega^2$)
\end{any}

\begin{answer}
% (1) 通分 (ω^2+1)/ω = -ω/ω = -1
% (2) (-ω^2)^10 = ω^20 = ω^2
\vspace{6cm}
\end{answer}

\begin{prac}{B2(高次方程式の理論)}
実数係数の4次方程式 $x^4 + ax^3 + bx^2 + cx + d = 0$ が, 虚数解 $1+i$ と $2-i$ をもつとき, 残りの解を求めよ.
\end{prac}

\begin{answer}
% 実数係数なので, 共役な複素数も解になる.
% 残りの解: 1-i, 2+i
\end{answer}

\end{multicols*}

\end{document}