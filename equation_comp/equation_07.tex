\documentclass[b4paper, landscape, dvipdfmx]{jsarticle}
%----- 必要なパッケージ -----
\usepackage{fancybox,ascmac,otf,ulem}
\usepackage{amssymb, amsthm}
\usepackage[leqno]{amsmath}
\usepackage{wrapfig}
\usepackage{geometry}
\usepackage{multicol}
\usepackage{tcolorbox}
\usepackage{xcolor}
\usepackage{fancyhdr}
\usepackage{tikz}

% shadowsライブラリ
\usetikzlibrary{
    positioning,
    arrows.meta,
    calc,
    shadows,
    shadows.blur,
    intersections
}

\tcbuselibrary{skins, breakable, theorems}
\usepackage{enumitem}
\setlist[enumerate,1]{label=(\arabic*)}
\setlist[itemize]{leftmargin=*}
\newcommand{\ds}{\displaystyle}

%----- レイアウト設定 -----
\geometry{
  left=15mm,
  right=15mm,
  top=20mm,
  bottom=15mm,
  headheight=25pt
}

%----- 数式環境の上下の余白調整 -----
\AtBeginDocument{
  \setlength{\abovedisplayskip}{5pt}
  \setlength{\belowdisplayskip}{5pt}
  \setlength{\abovedisplayshortskip}{0pt}
  \setlength{\belowdisplayshortskip}{3pt}
}

%===========================================================
%  デザイン設定
%===========================================================

%--- 色の定義 ---
\definecolor{printBlue}{RGB}{0, 50, 100}     % 濃紺
\definecolor{printRed}{RGB}{140, 20, 20}     % 濃エンジ
\definecolor{printTeal}{RGB}{0, 60, 60}      % 濃い青緑
\definecolor{gridColor}{gray}{0.75}          % 解答欄の方眼

%--- 共通スタイル定義 ---
\tcbset{
    chartbox/.style={
        enhanced,
        fonttitle=\sffamily\bfseries,
        boxrule=1pt,
        arc=2pt,
        top=1.0em,
        nobeforeafter,
        enlarge left by=-2mm,
        enlarge right by=-2mm,
        drop fuzzy shadow,
        colback=white,
        attach boxed title to top left={xshift=10pt, yshift*=-\tcboxedtitleheight/2},
        boxed title style={frame hidden, sharp corners, rounded corners=southeast, arc=3pt}
    }
}

% 各種ボックス環境定義
\newenvironment{overall}[1]{
\begin{tcolorbox}[
    chartbox,
    colframe=printTeal,
    coltitle=white,
    title=\textbf{全体課題 #1},
    boxed title style={colback=printTeal},
]}
{\end{tcolorbox}}

\newtcolorbox{any}[1]{
    enlarge left by=0mm, enlarge right by=0mm,
    enhanced, frame hidden, colback=white, title={#1},
    attach boxed title to top left={xshift=0mm, yshift=0mm},
    coltitle=white, fonttitle=\bfseries\sffamily,
    boxed title style={
        colback=black!80, frame hidden, arc=4pt, outer arc=4pt,
        sharp corners=south, boxrule=0pt,
        top=1mm, bottom=1mm, left=3mm, right=3mm
    },
    underlay boxed title={
        \draw[thick, black!80] (title.south west) -- (title.south west-|frame.east);
    },
    breakable, top=5mm, left=2mm, right=2mm, bottom=0mm,
    before skip=1em, after skip=1em,
    segmentation style={draw=black!40, dashed}
}

\newenvironment{eg}[1]{
\begin{tcolorbox}[
    chartbox,
    colframe=printBlue,
    coltitle=white,
    title=\textbf{例題 #1},
    boxed title style={colback=printBlue},
    segmentation style={draw=printBlue, line width=0.5pt, dashed}
]}
{\end{tcolorbox}}

\newenvironment{prac}[1]{
\begin{tcolorbox}[
    chartbox,
    colframe=printRed,
    coltitle=white,
    title=\textbf{練習 #1},
    boxed title style={colback=printRed}
]}
{\end{tcolorbox}}

\newenvironment{answer}[1][height fill]{
    \begin{tcolorbox}[
        enhanced,
        title={Memo / Answer},
        colframe=black!80,
        colback=white,
        coltitle=black!60,
        fonttitle=\sffamily\bfseries,
        attach boxed title to top left={xshift=5mm, yshift*=-\tcboxedtitleheight/2},
        boxed title style={frame hidden, colback=white},
        boxrule=1pt,
        arc=1pt,
        nobeforeafter,
        enlarge left by=2mm, 
        enlarge right by=2mm, 
        height fill,
        segmentation style={draw=black!20, solid},
        underlay={
            \begin{tcbclipinterior}
                \draw[step=5mm, black!5, ultra thin] (interior.south west) grid (interior.north east);
            \end{tcbclipinterior}
        }, 
        #1
    ]}
{ \end{tcolorbox}}

%----- 段組の設定 -----
\setlength{\columnsep}{15mm}
\setlength{\columnseprule}{0.4pt}
\renewcommand{\columnseprulecolor}{\color{black!30}}

%----- ヘッダーの設定 -----
\pagestyle{fancy}
\fancyhf{}

% ヘッダーデザイン
\fancyhead[C]{%
    \begin{tikzpicture}[remember picture, overlay]
        \node[anchor=north west, fill=printBlue, minimum width=\paperwidth, minimum height=5pt] at (current page.north west) {};
    \end{tikzpicture}
}
\fancyhead[L]{\small \textcolor{black!90}{数学\ajRoman{2} $>$ 第2章 複素数と方程式 $>$ 第7回--\textbf{高次方程式}}}
\fancyhead[R]{\small 年 \hspace{1cm} 組 \hspace{1cm} 番 \quad 氏名 \hspace{6cm}}
\renewcommand{\headrulewidth}{0pt}

\begin{document}

\begin{multicols*}{2}

%===========================================================
% 左カラム: 高次方程式の解法(基本)
%===========================================================

{\large \textbf{1. 高次方程式の解き方}}

\begin{any}{基本戦略:因数分解して次数を下げる}
3次以上の方程式を\textbf{高次方程式}という. 解の公式が存在しない(あるいは複雑すぎる)ため, 基本的には\textbf{因数定理}を用いて因数分解し, 1次式や2次式の積の形にもちこんで解く.
\begin{center}
\begin{tcolorbox}[colframe=printRed, colback=white, boxrule=1.5pt, arc=0pt]
    \textbf{手順}
    \begin{enumerate}
        \item $P(x)=0$ となる $x$ の値 ($k$) を見つける.
        \item $P(x)$ を $x-k$ で割り算し, $(x-k)(\text{2次式})=0$ にする.
        \item 2次方程式を解く (因数分解 or 解の公式).
    \end{enumerate}
\end{tcolorbox}
\end{center}
\end{any}

\begin{any}{割り算は「筆算」で十分!}
教科書等では「組立除法」という計算テクニックが紹介されることがあるが, \textbf{これは必須ではない}.
通常の\textbf{筆算による割り算}の方が, 「今何をしているか」が明確であり, ミスをした際の確認もしやすい. 堂々と筆算を使おう.
\end{any}

\begin{eg}{1(3次方程式)}
3次方程式 $x^3 - 4x^2 + 8x - 5 = 0$ を解け.
\end{eg}

\begin{answer}
\begin{minipage}[t]{0.55\textwidth}
\small
\begin{enumerate}
    \item[1.] $P(1) = 1 - 4 + 8 - 5 = 0$ なので, $x-1$ を因数にもつ.
    \item[2.] 筆算で割り算を行う(右図). \\
    商は $x^2 -3x +5$.
    \item[3.] $(x-1)(x^2 -3x +5) = 0$ \\
    $x=1$, または $x^2 -3x +5 = 0$ \\
    解の公式より \\
    $x = 1, \ \ds\frac{3\pm\sqrt{11}i}{2}$
\end{enumerate}
\end{minipage}
\hfill
\begin{minipage}[t]{0.4\textwidth}
% 筆算の図
\small
\[
\renewcommand{\arraystretch}{1.2}
\setlength\arraycolsep{1pt}
\begin{array}{r@{\,}l}
 & x^2 -3x +5 \\
\cline{2-2}
x-1 & \big) \ x^3 -4x^2 +8x -5 \\
 & x^3 -\phantom{4}x^2 \\
\cline{2-2}
 & \phantom{x^3} -3x^2 +8x \\
 & \phantom{x^3} -3x^2 +3x \\
\cline{2-2}
 & \phantom{x^3 -3x^2} 5x -5 \\
 & \phantom{x^3 -3x^2} 5x -5 \\
\cline{2-2}
 & \phantom{x^3 -3x^2 +5x} 0
\end{array}
\]
\end{minipage}
\end{answer}

\columnbreak

\begin{prac}{1}
次の3次方程式を解け.
\begin{enumerate}
    \item $x^3 - 2x^2 + x - 2 = 0$
    \item $x^3 - 3x^2 + 4x - 2 = 0$
\end{enumerate}
\end{prac}

\begin{answer}

\end{answer}

\columnbreak
%===========================================================
% 右カラム: 特殊な形・複二次式
%===========================================================

{\large \textbf{2. 置き換えを利用する形}}

\begin{any}{複二次式 $\boldsymbol{ax^4+bx^2+c=0}$}
$x^4$ と $x^2$ だけが登場する式は, $\boldsymbol{X=x^2}$ と置き換えることで2次方程式に帰着できる.
ただし, $X$ が求まった後, $x = \pm\sqrt{X}$ で $x$ に戻すことを忘れないこと.
\end{any}

\begin{eg}{2(複二次方程式)}
次の方程式を解け.
\begin{enumerate}
    \item $x^4 - 3x^2 - 4 = 0$
    \item $x^4 + 5x^2 - 36 = 0$
\end{enumerate}
\end{eg}

\begin{answer}
% 解説
% (1) X=x^2 とおく -> X^2 -3X -4 = 0 -> (X-4)(X+1)=0
%     X=4, -1 -> x^2=4, x^2=-1
%     x = ±2, ±i
% (2) (X+9)(X-4)=0 -> X=-9, 4
%     x = ±3i, ±2
\vspace{6cm}
\end{answer}

\begin{any}{Note: 重解の扱い}
$(x-1)^2(x+2)=0$ のように因数分解された場合, 解は $x=1$ (2重解), $x=-2$ となる.
3次方程式だからといって必ずしも異なる3つの解をもつとは限らない.
\end{any}

\end{multicols*}

\newpage

%===========================================================
% 裏面: 演習問題
%===========================================================

\begin{multicols*}{2}

{\large \textbf{確認テスト}}

\begin{prac}{A1(因数定理と筆算)}
次の方程式を解け.
\begin{enumerate}
    \item $x^3 - 3x^2 + 4 = 0$
    \item $x^3 + x^2 - 17x + 15 = 0$
\end{enumerate}
\end{prac}


\begin{prac}{A2(複二次式)}
次の方程式を解け.
\begin{enumerate}
    \item $x^4 - 5x^2 + 4 = 0$
    \item $x^4 - 16 = 0$
\end{enumerate}
\end{prac}

\begin{answer}
\vspace{5cm}
\end{answer}

\columnbreak

\begin{prac}{B1(虚数解をもつ高次方程式)}
$x^3 - 1 = 0$ を解け.
また, その虚数解の一つを $\omega$ (オメガ) とするとき, $\omega^2 + \omega + 1$ の値を求めよ.
\end{prac}

\begin{any}{Hint}
因数分解公式 $x^3-1 = (x-1)(x^2+x+1)$ を利用する.
\end{any}

\begin{prac}{B2(係数決定)}
3次方程式 $x^3 + ax^2 + bx - 6 = 0$ が $x=1$ と $x=2$ を解にもつとき, 定数 $a, b$ の値と, 残りの解を求めよ.
\end{prac}

\begin{answer}
% 代入して連立
% 1+a+b-6=0 -> a+b=5
% 8+4a+2b-6=0 -> 2a+b=-1
% これを解く a=-6, b=11
% x^3 -6x^2 +11x -6 = 0 -> (x-1)(x-2)(x-3)=0
% 残りの解 x=3
\end{answer}

\end{multicols*}

\end{document}