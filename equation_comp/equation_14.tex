\documentclass[b4paper, landscape, dvipdfmx]{jsarticle}
%----- 必要なパッケージ -----
\usepackage{fancybox,ascmac,otf,ulem}
\usepackage{amssymb, amsthm}
\usepackage[leqno]{amsmath}
\usepackage{geometry}
\usepackage{multicol}
\usepackage{tcolorbox}
\usepackage{xcolor}
\usepackage{fancyhdr}
\usepackage{tikz}

% shadowsライブラリ
\usetikzlibrary{
    positioning,
    arrows.meta,
    calc,
    shadows,
    shadows.blur,
    intersections
}

\tcbuselibrary{skins, breakable, theorems}
\usepackage{enumitem}
\setlist[enumerate,1]{label=(\arabic*)}
\setlist[itemize]{leftmargin=*}
\newcommand{\ds}{\displaystyle}

%----- レイアウト設定 -----
\geometry{
  left=15mm,
  right=15mm,
  top=20mm,
  bottom=15mm,
  headheight=25pt
}

%----- 数式環境の上下の余白調整 -----
\AtBeginDocument{
  \setlength{\abovedisplayskip}{5pt}
  \setlength{\belowdisplayskip}{5pt}
  \setlength{\abovedisplayshortskip}{0pt}
  \setlength{\belowdisplayshortskip}{3pt}
}

%===========================================================
%  デザイン設定
%===========================================================

%--- 色の定義 ---
\definecolor{printBlue}{RGB}{0, 50, 100}     % 濃紺
\definecolor{printRed}{RGB}{140, 20, 20}     % 濃エンジ
\definecolor{printTeal}{RGB}{0, 60, 60}      % 濃い青緑

%--- 共通スタイル定義 ---
\tcbset{
    chartbox/.style={
        enhanced,
        fonttitle=\sffamily\bfseries,
        boxrule=1pt,
        arc=2pt,
        top=1.0em,
        nobeforeafter,
        enlarge left by=-2mm,
        enlarge right by=-2mm,
        drop fuzzy shadow,
        colback=white,
        attach boxed title to top left={xshift=10pt, yshift*=-\tcboxedtitleheight/2},
        boxed title style={frame hidden, sharp corners, rounded corners=southeast, arc=3pt}
    }
}

% 各種ボックス環境定義
\newtcolorbox{any}[1]{
    enlarge left by=0mm, enlarge right by=0mm,
    enhanced, frame hidden, colback=white, title={#1},
    attach boxed title to top left={xshift=0mm, yshift=0mm},
    coltitle=white, fonttitle=\bfseries\sffamily,
    boxed title style={
        colback=black!80, frame hidden, arc=4pt, outer arc=4pt,
        sharp corners=south, boxrule=0pt,
        top=1mm, bottom=1mm, left=3mm, right=3mm
    },
    underlay boxed title={
        \draw[thick, black!80] (title.south west) -- (title.south west-|frame.east);
    },
    breakable, top=5mm, left=2mm, right=2mm, bottom=0mm,
    before skip=1em, after skip=1em,
    segmentation style={draw=black!40, dashed}
}


\newenvironment{eg}[1]{
\begin{tcolorbox}[
    chartbox,
    colframe=printBlue,
    coltitle=white,
    title=\textbf{例題 #1},
    boxed title style={colback=printBlue},
    segmentation style={draw=printBlue, line width=0.5pt, dashed}
]}
{\end{tcolorbox}}

\newenvironment{prac}[1]{
\begin{tcolorbox}[
    chartbox,
    colframe=printRed,
    coltitle=white,
    title=\textbf{練習 #1},
    boxed title style={colback=printRed}
]}
{\end{tcolorbox}}

\newenvironment{answer}[1][height fill]{
    \begin{tcolorbox}[
        enhanced,
        title={Memo / Answer},
        colframe=black!80,
        colback=white,
        coltitle=black!60,
        fonttitle=\sffamily\bfseries,
        attach boxed title to top left={xshift=5mm, yshift*=-\tcboxedtitleheight/2},
        boxed title style={frame hidden, colback=white},
        boxrule=1pt,
        arc=1pt,
        nobeforeafter,
        enlarge left by=2mm, 
        enlarge right by=2mm, 
        height fill,
        segmentation style={draw=black!20, solid},
        underlay={
            \begin{tcbclipinterior}
                \draw[step=5mm, black!5, ultra thin] (interior.south west) grid (interior.north east);
            \end{tcbclipinterior}
        }, 
        #1
    ]}
{ \end{tcolorbox}}

\newenvironment{proofbox}[1][height fill]{
    \begin{tcolorbox}[
        enhanced,
        title={Proof},
        colframe=black!80,
        colback=white,
        coltitle=black!60,
        fonttitle=\sffamily\bfseries,
        attach boxed title to top left={xshift=5mm, yshift*=-\tcboxedtitleheight/2},
        boxed title style={frame hidden, colback=white},
        boxrule=1pt,
        arc=1pt,
        nobeforeafter,
        width=\linewidth,
        underlay={
            \begin{tcbclipinterior}
                \draw[step=5mm, black!5, ultra thin] (interior.south west) grid (interior.north east);
            \end{tcbclipinterior}
        },
        #1
    ]}
{ \end{tcolorbox}}

%----- 段組の設定 -----
\setlength{\columnsep}{15mm}
\setlength{\columnseprule}{0.4pt}
\renewcommand{\columnseprulecolor}{\color{black!30}}

%----- ヘッダーの設定 -----
\pagestyle{fancy}
\fancyhf{}

% ヘッダーデザイン
\fancyhead[C]{%
    \begin{tikzpicture}[remember picture, overlay]
        \node[anchor=north west, fill=printBlue, minimum width=\paperwidth, minimum height=5pt] at (current page.north west) {};
    \end{tikzpicture}
}
\fancyhead[L]{\small \textcolor{black!90}{数学\ajRoman{2} $>$ 第3章 式と証明 $>$ 第14回--\textbf{相加・相乗平均(1)}}}
\fancyhead[R]{\small 年 \hspace{1cm} 組 \hspace{1cm} 番 \quad 氏名 \hspace{6cm}}
\renewcommand{\headrulewidth}{0pt}

\begin{document}

\begin{multicols*}{2}

%===========================================================
% 左カラム: 公式の導入
%===========================================================

{\large \textbf{1. 相加平均と相乗平均}}

\begin{any}{足して2で割る vs 掛けてルート}
2つの正の数 $a, b$ について,
\begin{itemize}
    \item \textbf{相加平均}: $\ds \frac{a+b}{2}$ \quad (いわゆる普通の平均)
    \item \textbf{相乗平均}: $\ds \sqrt{ab}$ \quad (掛け算の平均)
\end{itemize}
この2つの間には, 常に「相加平均の方が大きい(か等しい)」という関係が成り立つ.
\end{any}

\begin{any}{相加・相乗平均の大小関係}
\begin{center}
\begin{tcolorbox}[colframe=printRed, colback=white, boxrule=1.5pt, arc=0pt]
    $\boldsymbol{a>0, \ b>0}$ のとき,
    \[ \frac{a+b}{2} \geqq \sqrt{ab} \quad \iff \quad \boldsymbol{a+b \geqq 2\sqrt{ab}} \]
    (等号成立は $\boldsymbol{a=b}$ のとき)
\end{tcolorbox}
\end{center}
※この公式は, 必ず\textbf{「正の数」}のときしか使えない!
\end{any}

\begin{any}{図形的な意味(半径と半弦)}

半円において, 半径は常に「弦への垂線」以上になる.
\begin{center}
\begin{tikzpicture}[scale=1.5]
    % Coordinates
    \coordinate (O) at (0,0);
    \coordinate (L) at (-2,0);
    \coordinate (R) at (2,0);
    \coordinate (M) at (0.5,0); % Split point
    
    % Semicircle
    \draw[thick] (R) arc (0:180:2);
    \draw[thick] (L) -- (R);
    
    % Radius (Arithmetic Mean)
    \draw[->, thick, printBlue] (O) -- (0,2) 
    % node[midway, left] {$\frac{a+b}{2}$}
    ;
    
    % Perpendicular (Geometric Mean)
    \draw[dashed] (M) -- (M |- 0,1.936); % Intersection calc
    \draw[->, thick, printRed] (M) -- (0.5, {sqrt(1.5*2.5)})
    % node[midway, right] {$\sqrt{ab}$}
    ;
    
    % Labels
    \node[below] at (-0.75, 0) {$a$};
    \node[below] at (1.25, 0) {$b$};
    \node[below] at (O) {中心};
    
    % Note
    \node[above] at (0, 2.2) {\textbf{相加平均 (半径)}};
    \node[right] at (0.5, 2.0) {\textbf{相乗平均 (垂線)}};
    
\end{tikzpicture}
\end{center}
\end{any}

\columnbreak

%===========================================================
% 右カラム: 使いどころ
%===========================================================

{\large \textbf{2. 逆数の和の最小値}}

\begin{any}{文字を消去できる!}
この不等式が最強の威力を発揮するのは, 積 $ab$ が定数(特に $ab=1$)になるときである.
例えば $x$ と $\frac{1}{x}$ の和などは, 掛け算すると文字が消える.
\end{any}

\begin{eg}{1(基本的な利用)}
$x>0$ のとき, 次の不等式を証明せよ.
\[ x + \frac{4}{x} \geqq 4 \]
\end{eg}

\begin{proofbox}
% $x>0$ より $\frac{4}{x} > 0$ であるから, 相加・相乗平均の大小関係より
% \[ x + \frac{4}{x} \geqq 2\sqrt{x \cdot \frac{4}{x}} \]
% 右辺を計算すると,
% \[ 2\sqrt{4} = 2 \cdot 2 = 4 \]
% よって, $x + \frac{4}{x} \geqq 4$. \qed
\end{proofbox}

\begin{any}{等号成立条件のチェック}
「最小値は4である」と言い切るためには, 実際に $=4$ になる瞬間が存在することを確認する必要がある.
等号が成立するのは $x = \frac{4}{x}$ のとき.
\[ x^2 = 4 \iff x = 2 \quad (\because x>0) \]
つまり, $x=2$ のとき最小値 $4$ をとる.
\end{any}

\begin{eg}{2(最小値を求める)}
$x>0$ のとき, $x + \frac{1}{x}$ の最小値を求めよ.
\end{eg}

\begin{answer}
% 解説
% 相加相乗より x+1/x >= 2√(x*1/x) = 2
% 等号成立は x=1/x -> x^2=1 -> x=1
% 最小値 2 (x=1のとき)
\vspace{4cm}
\end{answer}

{\large \textbf{3. 応用:分母を作り出す}}

\begin{any}{約分できないときは?}
相加・相乗平均の関係は「掛け算すると文字が消える」ときに最強の威力を発揮する.
しかし, $\ds x + \frac{1}{x+1}$ のような式は, 単純に掛けても $\ds \frac{x}{x+1}$ となり, 文字が消えない.

\vspace{0.5em}
\textbf{Strategy:} \textbf{分母と同じ形を無理やり作る!}
\begin{center}
\begin{tcolorbox}[colframe=printRed, colback=white, boxrule=1.5pt, arc=0pt]
    \centering
    $\ds x + \frac{k}{x+a} = (\boldsymbol{x+a}) + \frac{k}{x+a} \boldsymbol{- a}$
\end{tcolorbox}
\end{center}
勝手に足した分を, 後ろで引いて帳尻を合わせればよい.
\end{any}

\begin{eg}{3(定数項の調整)}
$x > -2$ のとき, $\ds x + \frac{9}{x+2}$ の最小値を求めよ.
\end{eg}

\begin{answer}
    
\end{answer}

\columnbreak

%===========================================================
% 右カラム: 最大・最小問題
%===========================================================

{\large \textbf{4. 条件付きの最大・最小}}

\begin{any}{和が一定なら積は最大}
$a+b=const$(一定)のとき, $ab$ の最大値を求めるのにも使える.
\[ a+b \geqq 2\sqrt{ab} \iff \frac{a+b}{2} \geqq \sqrt{ab} \]
両辺を2乗すれば, $ab$ の上限が見えてくる.
\end{any}

\begin{eg}{4(積の最大値)}
$x>0, \ y>0$ とする. $2x + 3y = 12$ のとき, $xy$ の最大値を求めよ.
\end{eg}

\begin{answer}
    
\end{answer}

\begin{any}{等号成立時の $x, y$}
等号成立は $2x = 3y$ のとき.
これと $2x+3y=12$ を連立させると,
$2x + 2x = 12 \to 4x=12 \to x=3$.
$3y = 6 \to y=2$.
つまり $(x, y) = (3, 2)$ のとき最大となる.
\end{any}

\end{multicols*}

\newpage

%===========================================================
% 裏面: 演習問題
%===========================================================

\begin{multicols*}{2}

{\large \textbf{確認テスト}}

\begin{prac}{A1(基本証明)}
$a>0, \ b>0$ のとき, 次の不等式を証明せよ.
\[ a + b \geqq 2\sqrt{ab} \]
を利用して, $2a + 3b \geqq 2\sqrt{6ab}$ を示せ.
\end{prac}

\begin{proofbox}
$a>0, b>0$ より $2a>0, 3b>0$.
相加・相乗平均の大小関係より
\[ 2a + 3b \geqq 2\sqrt{2a \cdot 3b} = 2\sqrt{6ab} \]
等号成立は $2a=3b$ のとき.
\end{proofbox}


\begin{prac}{A2(逆数の和)}
$x>0$ のとき, 次の不等式を証明せよ. また, 等号成立条件を求めよ.
\[ x + \frac{9}{x} \geqq 6 \]
\end{prac}

\begin{proofbox}
\vspace{6cm}
\end{proofbox}

\columnbreak

\begin{prac}{B1(展開してから使う)}
$a>0, \ b>0$ のとき, 次の不等式を証明せよ.
\[ \left( a + \frac{1}{b} \right) \left( b + \frac{1}{a} \right) \geqq 4 \]
\end{prac}

\begin{any}{Hint}
まずは左辺を展開してみよう.
$ab + 1 + 1 + \frac{1}{ab} = ab + \frac{1}{ab} + 2$
この $ab + \frac{1}{ab}$ の部分に相加・相乗平均を使う.
\end{any}

\begin{proofbox}
\vspace{6cm}
\end{proofbox}

\begin{prac}{B2(最小値)}
$x > 0$ のとき, 関数 $y = 3x + \frac{12}{x} + 5$ の最小値を求めよ.
\end{prac}

\begin{answer}
% 3x + 12/x >= 2√(36) = 12
% 全体 >= 12+5 = 17
% 等号 3x=12/x -> x^2=4 -> x=2
% 最小値 17 (x=2)
\end{answer}

\end{multicols*}

\end{document}