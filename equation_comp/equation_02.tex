\documentclass[b4paper, landscape, dvipdfmx]{jsarticle}
%----- 必要なパッケージ -----
\usepackage{fancybox,ascmac,otf,ulem}
\usepackage{amssymb, amsthm}
\usepackage[leqno]{amsmath}
\usepackage{wrapfig}
\usepackage{geometry}
\usepackage{multicol}
\usepackage{tcolorbox}
\usepackage{xcolor}
\usepackage{fancyhdr}
\usepackage{tikz}

% shadowsライブラリ
\usetikzlibrary{
    positioning,
    arrows.meta,
    calc,
    shadows,
    shadows.blur,
    intersections
}

\tcbuselibrary{skins, breakable, theorems}
\usepackage{enumitem}
\setlist[enumerate,1]{label=(\arabic*)}
\setlist[itemize]{leftmargin=*}
\newcommand{\ds}{\displaystyle}

%----- レイアウト設定 -----
\geometry{
  left=15mm,
  right=15mm,
  top=20mm,
  bottom=15mm,
  headheight=25pt
}

%----- 数式環境の上下の余白調整 -----
\AtBeginDocument{
  \setlength{\abovedisplayskip}{5pt}
  \setlength{\belowdisplayskip}{5pt}
  \setlength{\abovedisplayshortskip}{0pt}
  \setlength{\belowdisplayshortskip}{3pt}
}

%===========================================================
%  デザイン設定(白黒印刷対応・ハイコントラスト)
%===========================================================

%--- 色の定義 ---
% 印刷時にしっかり濃く出るように調整
\definecolor{printBlue}{RGB}{0, 50, 100}     % 濃紺(例題・解説)
\definecolor{printRed}{RGB}{140, 20, 20}     % 濃エンジ(練習・重要)
\definecolor{printTeal}{RGB}{0, 60, 60}      % 濃い青緑(全体課題)
\definecolor{gridColor}{gray}{0.75}          % 解答欄の方眼(印刷飛び防止のため少し濃く)

%--- 共通スタイル定義 ---
\tcbset{
    chartbox/.style={
        enhanced,
        fonttitle=\sffamily\bfseries,
        boxrule=1pt,
        arc=2pt,
        top=1.0em,
        nobeforeafter,
        enlarge left by=-2mm,
        enlarge right by=-2mm,
        drop fuzzy shadow,
        colback=white, % 背景を白に固定(コピー対策)
        attach boxed title to top left={xshift=10pt, yshift*=-\tcboxedtitleheight/2},
        boxed title style={frame hidden, sharp corners, rounded corners=southeast, arc=3pt}
    }
}

% 1. 「全体課題」用のボックス環境
\newenvironment{overall}[1]{
\begin{tcolorbox}[
    chartbox,
    colframe=printTeal,
    coltitle=white,
    title=\textbf{全体課題 #1},
    boxed title style={colback=printTeal},
]}
{\end{tcolorbox}}

% 5. 解説・雑記用(any環境:黒枠・白背景)
\newenvironment{any}[1]{
    \begin{tcolorbox}[
        enhanced,
        colframe=black,       % 真っ黒に変更してくっきりさせる
        colback=white,
        coltitle=white,
        title=\textbf{#1},
        fonttitle=\sffamily,
        attach boxed title to top left={xshift=3mm, yshift*=-\tcboxedtitleheight/2},
        boxed title style={frame hidden, colback=black, sharp corners, rounded corners=northeast, arc=3pt},
        boxrule=1pt,
        arc=2pt,
        top=1em,
        nobeforeafter,
        enlarge left by=-2mm,
        enlarge right by=-2mm,
        segmentation style={draw=black!60, dashed}
    ]}
{ \end{tcolorbox}}

% 6. 例題用(濃紺枠・白背景)
\newenvironment{eg}[1]{
\begin{tcolorbox}[
    chartbox,
    colframe=printBlue,
    coltitle=white,
    title=\textbf{例題 #1},
    boxed title style={colback=printBlue},
    segmentation style={draw=printBlue, line width=0.5pt, dashed}
]}
{\end{tcolorbox}}

% 7. 練習用(濃エンジ枠・白背景)
\newenvironment{prac}[1]{
\begin{tcolorbox}[
    chartbox,
    colframe=printRed,
    coltitle=white,
    title=\textbf{練習 #1},
    boxed title style={colback=printRed}
]}
{\end{tcolorbox}}

\newenvironment{answer}[1][height fill]{
    \begin{tcolorbox}[
        enhanced,
        title={Memo / Answer},
        colframe=black!80,
        colback=white,
        coltitle=black!60,
        fonttitle=\sffamily\bfseries,
        attach boxed title to top left={xshift=5mm, yshift*=-\tcboxedtitleheight/2},
        boxed title style={frame hidden, colback=white},
        boxrule=1pt,
        arc=1pt,
        nobeforeafter,
        enlarge left by=2mm, 
        enlarge right by=2mm, 
        height fill,
        segmentation style={draw=black!20, solid},
        underlay={% 方眼を描画
            \begin{tcbclipinterior}
                \draw[step=5mm, black!5, ultra thin] (interior.south west) grid (interior.north east);
            \end{tcbclipinterior}
        }, 
        #1
    ]}
{ \end{tcolorbox}}

\newenvironment{proofbox}[1][height fill]{
    \begin{tcolorbox}[
        enhanced,
        title={Proof},
        colframe=black!80,
        colback=white,
        coltitle=black!60,
        fonttitle=\sffamily\bfseries,
        attach boxed title to top left={xshift=5mm, yshift*=-\tcboxedtitleheight/2},
        boxed title style={frame hidden, colback=white},
        boxrule=1pt,
        arc=1pt,
        nobeforeafter,
        width=\linewidth,
        underlay={% 方眼を描画
            \begin{tcbclipinterior}
                \draw[step=5mm, black!5, ultra thin] (interior.south west) grid (interior.north east);
            \end{tcbclipinterior}
        },
        #1
    ]}
{ \end{tcolorbox}}

%----- 段組の設定 -----
\setlength{\columnsep}{15mm}
\setlength{\columnseprule}{0.4pt}
\renewcommand{\columnseprulecolor}{\color{black!30}}

%----- ヘッダーの設定 -----
\pagestyle{fancy}
\fancyhf{}

% ヘッダーデザイン
\fancyhead[C]{%
    \begin{tikzpicture}[remember picture, overlay]
        % ヘッダーの帯も濃紺に
        \node[anchor=north west, fill=printBlue, minimum width=\paperwidth, minimum height=5pt] at (current page.north west) {};
    \end{tikzpicture}
}
\fancyhead[L]{\small \textcolor{black!90}{数学\ajRoman{2} $>$ 第1章 式と証明 $>$ 第2回--\textbf{整式の除法と分数式}}}
\fancyhead[R]{\small 年 \hspace{1cm} 組 \hspace{1cm} 番 \quad 氏名 \hspace{6cm}}
\renewcommand{\headrulewidth}{0pt}

\begin{document}

\begin{multicols*}{2}

%===========================================================
% 左カラム: 整式の除法と分数式の計算
%===========================================================

{\large \textbf{1. 整式の除法}}

\begin{any}{商と余りの関係}
整式 $A$ を整式 $B$ で割ったときの商を $Q$, 余りを $R$ とすると, 次の等式が成り立つ.

\begin{center}
\begin{tcolorbox}[colframe=printRed, colback=white, boxrule=1.5pt, arc=0pt]
    \centering
    $\ds A = BQ + R $
\end{tcolorbox}
\end{center}
\textbf{注意点}:
\begin{itemize}
    \item (余り $R$ の次数) $<$ (割る式 $B$ の次数)
    \item 割り切れるとき, $R=0$ となり, $A=BQ$ と表せる.
\end{itemize}
\end{any}

\begin{eg}{1(整式の割り算)}
整式 $A=x^3-4x^2+5$ を 整式 $B=x-3$ で割った商と余りを求めよ. また, $A=BQ+R$ の形で表せ.
\end{eg}

% 解説用スペース(筆算を板書する)
\begin{answer}
% ここに板書:筆算の書き方を確認
% x^2 -x -3
% __________
% x-3 ) x^3 -4x^2 +0x +5
% ...
\end{answer}

\begin{prac}{1}
次の整式 $A$ を整式 $B$ で割った商と余りを求めよ.
\begin{enumerate}
    \item $A = 2x^3 - x^2 + 1, \quad B = x^2 + 1$
\end{enumerate}
\end{prac}

\begin{answer}
\vspace{4cm}
\end{answer}

{\large \textbf{2. 分数式の乗法・除法・加法・減法}}

\begin{any}{計算のPoint}
\begin{enumerate}
    \item \textbf{乗法・除法}: まず\textbf{因数分解}し, 約分して簡単な形にする.
    \item \textbf{加法・減法}: 分母が異なるときは, \textbf{通分}する.
    \[ \frac{A}{C} + \frac{B}{C} = \frac{A+B}{C}, \quad \frac{A}{B} + \frac{C}{D} = \frac{AD+BC}{BD} \]
\end{enumerate}
\end{any}

\begin{eg}{2(分数式の計算)}
次の計算をせよ.
\begin{enumerate}
    \item $\ds \frac{x^2-1}{x^2-x-6} \times \frac{x-3}{x-1}$
    \item $\ds \frac{2}{x+1} - \frac{1}{x-2}$
\end{enumerate}
\end{eg}


\begin{answer}
\end{answer}

\columnbreak

%===========================================================
% 右カラム: 恒等式の概念と部分分数分解
%===========================================================

{\large \textbf{3. 恒等式とは}}

\begin{any}{恒等式 vs 方程式}
\begin{itemize}
    \item \textbf{方程式}: 特定の $x$ の値についてのみ成り立つ等式.
    \begin{itemize}
        \item 例: $2x - 4 = 0$ \quad ($x=2$ のときのみ真)
    \end{itemize}
    \item \textbf{恒等式}: どのような $x$ の値を代入しても成り立つ等式.
    \begin{itemize}
        \item 例: $(x+1)^2 = x^2+2x+1$ \quad (展開公式はすべて恒等式)
        \item 例: $\ds \frac{1}{x(x+1)} = \frac{1}{x} - \frac{1}{x+1}$ \quad (右辺を通分すると左辺になる)
    \end{itemize}
\end{itemize}
\end{any}

\begin{eg}{3(恒等式の係数決定)}
等式 $ax^2 + (b-1)x + c = 2x^2 + 3x - 1$ が $x$ についての恒等式となるように, 定数 $a, b, c$ の値を定めよ.
\end{eg}

\begin{answer}
    
\end{answer}

% \begin{answer}
% % メモ:係数比較法の解説
% % x^2の係数, xの係数, 定数項を比較する
% \vspace{3cm}
% \end{answer}

\columnbreak

{\large \textbf{4. 応用:部分分数分解}}

\begin{any}{分数を「分ける」テクニック}
分数式の計算において, 複雑な分数を「単純な分数の和・差」に分解することを\textbf{部分分数分解}という.
特に次の形はよく用いられる(数列や積分で重要).

\[ \frac{1}{A \times B} = \frac{1}{B-A} \left( \frac{1}{A} - \frac{1}{B} \right) \]
\end{any}

\begin{eg}{4(部分分数分解)}
次の等式が恒等式となるように, 定数 $a, b$ の値を定めよ.
\[ \frac{2}{x(x+2)} = \frac{a}{x} + \frac{b}{x+2} \]
\end{eg}

% \begin{answer}
% % ここに板書:右辺を通分して分子を比較する思考プロセス
% \vspace{6cm}
% \end{answer}




\begin{answer}
\end{answer}

\end{multicols*}

\newpage

%===========================================================
% 裏面: 演習問題
%===========================================================

\begin{multicols*}{2}

{\large \textbf{確認テスト}}

\begin{prac}{A1(整式の除法)}
整式 $A=3x^3-5x+2$ を 整式 $B=x^2-2$ で割った商と余りを求めよ.
\end{prac}

\begin{prac}{A2(分数式の計算)}
次の計算をせよ.
\begin{enumerate}
    \item $\ds \frac{x^2-4}{2x+4} \times \frac{4x}{x^2-2x}$
    \item $\ds \frac{2x}{x^2-1} - \frac{1}{x-1}$
\end{enumerate}
\end{prac}

\begin{answer}
\vspace{5cm}
\end{answer}


\columnbreak

\begin{prac}{B1(繁分数式)}
次の式を簡単にせよ.
\[ 1 - \frac{1}{1-\frac{1}{x+1}} \]
\end{prac}

\begin{any}{Hint}
分母の $\ds 1-\frac{1}{x+1}$ から順に計算する.
あるいは, 分母と分子に $(x+1)$ を掛けて一気に解消する方法もある.
\end{any}

\begin{prac}{B2(恒等式と部分分数分解)}
次の等式が $x$ についての恒等式となるように, 定数 $a, b, c$ の値を定めよ.
\[ \frac{1}{x(x+1)(x+2)} = \frac{a}{x(x+1)} + \frac{b}{(x+1)(x+2)} \]
また, この結果を利用して $\ds \frac{c}{x} + \frac{d}{x+1} + \frac{e}{x+2}$ の形に分解せよ.
\end{prac}

\begin{answer}
\end{answer}

\end{multicols*}

\end{document}