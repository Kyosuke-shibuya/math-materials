\documentclass[b4paper, landscape, dvipdfmx]{jsarticle}
%----- 必要なパッケージ -----
\usepackage{fancybox,ascmac,otf,ulem}
\usepackage{amssymb, amsthm}
\usepackage[leqno]{amsmath}
\usepackage{geometry}
\usepackage{multicol}
\usepackage{tcolorbox}
\usepackage{xcolor}
\usepackage{fancyhdr}
\usepackage{tikz}

% shadowsライブラリ
\usetikzlibrary{
    positioning,
    arrows.meta,
    calc,
    shadows,
    shadows.blur,
    intersections
}

\tcbuselibrary{skins, breakable, theorems}
\usepackage{enumitem}
\setlist[enumerate,1]{label=(\arabic*)}
\setlist[itemize]{leftmargin=*}
\newcommand{\ds}{\displaystyle}

%----- レイアウト設定 -----
\geometry{
  left=15mm,
  right=15mm,
  top=20mm,
  bottom=15mm,
  headheight=25pt
}

%----- 数式環境の上下の余白調整 -----
\AtBeginDocument{
  \setlength{\abovedisplayskip}{5pt}
  \setlength{\belowdisplayskip}{5pt}
  \setlength{\abovedisplayshortskip}{0pt}
  \setlength{\belowdisplayshortskip}{3pt}
}

%===========================================================
%  デザイン設定
%===========================================================

%--- 色の定義 ---
\definecolor{printBlue}{RGB}{0, 50, 100}     % 濃紺
\definecolor{printRed}{RGB}{140, 20, 20}     % 濃エンジ
\definecolor{printTeal}{RGB}{0, 60, 60}      % 濃い青緑
\definecolor{printOrange}{RGB}{200, 100, 0}  % まとめ用オレンジ

%--- 共通スタイル定義 ---
\tcbset{
    chartbox/.style={
        enhanced,
        fonttitle=\sffamily\bfseries,
        boxrule=1pt,
        arc=2pt,
        top=1.0em,
        nobeforeafter,
        enlarge left by=-2mm,
        enlarge right by=-2mm,
        drop fuzzy shadow,
        colback=white,
        attach boxed title to top left={xshift=10pt, yshift*=-\tcboxedtitleheight/2},
        boxed title style={frame hidden, sharp corners, rounded corners=southeast, arc=3pt}
    }
}

% 各種ボックス環境定義
\newenvironment{summary}[1]{
    \begin{tcolorbox}[
        chartbox,
        colframe=printOrange,
        coltitle=white,
        title=\textbf{#1},
        boxed title style={colback=printOrange},
    ]}
{ \end{tcolorbox}}

\newtcolorbox{any}[1]{
    enlarge left by=0mm, enlarge right by=0mm,
    enhanced, frame hidden, colback=white, title={#1},
    attach boxed title to top left={xshift=0mm, yshift=0mm},
    coltitle=white, fonttitle=\bfseries\sffamily,
    boxed title style={
        colback=black!80, frame hidden, arc=4pt, outer arc=4pt,
        sharp corners=south, boxrule=0pt,
        top=1mm, bottom=1mm, left=3mm, right=3mm
    },
    underlay boxed title={
        \draw[thick, black!80] (title.south west) -- (title.south west-|frame.east);
    },
    breakable, top=5mm, left=2mm, right=2mm, bottom=0mm,
    before skip=1em, after skip=1em,
    segmentation style={draw=black!40, dashed}
}


\newenvironment{eg}[1]{
\begin{tcolorbox}[
    chartbox,
    colframe=printBlue,
    coltitle=white,
    title=\textbf{例題 #1},
    boxed title style={colback=printBlue},
    segmentation style={draw=printBlue, line width=0.5pt, dashed}
]}
{\end{tcolorbox}}

\newenvironment{prac}[1]{
\begin{tcolorbox}[
    chartbox,
    colframe=printRed,
    coltitle=white,
    title=\textbf{練習 #1},
    boxed title style={colback=printRed}
]}
{\end{tcolorbox}}

\newenvironment{answer}[1][height fill]{
    \begin{tcolorbox}[
        enhanced,
        title={Memo / Answer},
        colframe=black!80,
        colback=white,
        coltitle=black!60,
        fonttitle=\sffamily\bfseries,
        attach boxed title to top left={xshift=5mm, yshift*=-\tcboxedtitleheight/2},
        boxed title style={frame hidden, colback=white},
        boxrule=1pt,
        arc=1pt,
        nobeforeafter,
        enlarge left by=2mm, 
        enlarge right by=2mm, 
        height fill,
        segmentation style={draw=black!20, solid},
        underlay={
            \begin{tcbclipinterior}
                \draw[step=5mm, black!5, ultra thin] (interior.south west) grid (interior.north east);
            \end{tcbclipinterior}
        }, 
        #1
    ]}
{ \end{tcolorbox}}

%----- 段組の設定 -----
\setlength{\columnsep}{15mm}
\setlength{\columnseprule}{0.4pt}
\renewcommand{\columnseprulecolor}{\color{black!30}}

%----- ヘッダーの設定 -----
\pagestyle{fancy}
\fancyhf{}

% ヘッダーデザイン
\fancyhead[C]{%
    \begin{tikzpicture}[remember picture, overlay]
        \node[anchor=north west, fill=printBlue, minimum width=\paperwidth, minimum height=5pt] at (current page.north west) {};
    \end{tikzpicture}
}
\fancyhead[L]{\small \textcolor{black!90}{数学\ajRoman{2} $>$ 第3章 式と証明 $>$ 第17回--\textbf{単元の総整理(2)}}}
\fancyhead[R]{\small 年 \hspace{1cm} 組 \hspace{1cm} 番 \quad 氏名 \hspace{6cm}}
\renewcommand{\headrulewidth}{0pt}

\begin{document}

\begin{multicols*}{2}

%===========================================================
% 左カラム: パラメータと解の個数
%===========================================================

{\large \textbf{1. 発展:方程式の実数解の個数}}

\begin{any}{定数を分離して「グラフ」で見る}
方程式 $f(x)=a$ の実数解は,
\begin{itemize}
    \item 曲線 $y=f(x)$
    \item 直線 $y=a$ (横線)
\end{itemize}
の\textbf{共有点}として視覚化できる.
「解の個数を求めよ」と言われたら, $\boldsymbol{a}$ \textbf{を分離してグラフを描け!}
\end{any}

\begin{eg}{1(定数分離)}
3次方程式 $x^3 - 3x^2 - a = 0$ が, 異なる3つの実数解をもつように, 定数 $a$ の値の範囲を定めよ.
\end{eg}

\begin{any}{Step}
\begin{enumerate}
    \item $x^3 - 3x^2 = a$ と変形する.
    \item $y = x^3 - 3x^2$ のグラフを描く.
    \item 直線 $y=a$ を上下に動かし, 3回交わる範囲を探す.
\end{enumerate}
\end{any}

\begin{answer}
% \begin{minipage}[t]{0.6\textwidth}
% \small
% $y = x^3 - 3x^2 = x^2(x-3)$
% このグラフは $x=0$ で接し, $x=3$ で交わる.
% 極大値は $x=0$ のとき $y=0$.
% 極小値は...
% (本来は微分で求めるが, 今回は比の性質や点描で概形を掴む)
% $x=2$ のとき $y = 8 - 12 = -4$ (極小).

% グラフより, $y=a$ が3点で交わるのは
% $\boldsymbol{-4 < a < 0}$ のとき.
% \end{minipage}
% \hfill
% \begin{minipage}[t]{0.35\textwidth}
% \begin{center}
% \begin{tikzpicture}[scale=0.4]
%     \draw[->] (-2,0) -- (4,0) node[right] {$x$};
%     \draw[->] (0,-5) -- (0,2) node[above] {$y$};
%     \draw[blue, thick, smooth, samples=100, domain=-1.2:3.2] plot (\x, {\x*\x*(\x-3)});
%     \draw[dashed] (0,0) -- (0,0) node[above left] {0};
%     \draw[dashed] (2,-4) -- (0,-4) node[left] {-4};
%     \draw[red, thick] (-1.5, -2) -- (3.5, -2) node[right] {$y=a$};
% \end{tikzpicture}
% \end{center}
% \end{minipage}
\end{answer}

\columnbreak

%===========================================================
% 右カラム: 三角不等式
%===========================================================

{\large \textbf{2. 発展:絶対値を含む不等式}}

\begin{any}{三角不等式}
絶対値についても, 重要な不等式がある.
\begin{center}
\begin{tcolorbox}[colframe=printRed, colback=white, boxrule=1.5pt, arc=0pt]
    \[\boldsymbol{|a+b| \leqq |a| + |b|}\]
\end{tcolorbox}
\end{center}
「寄り道せずに足した長さ($|a+b|$)」よりも, 「それぞれの長さを足したもの($|a|+|b|$)」の方が長い(か等しい), という意味. 三角形の成立条件とも関係している.
\end{any}

\begin{eg}{2(三角不等式の証明)}
不等式 $|a+b| \leqq |a| + |b|$ を証明せよ.
\end{eg}

\noindent
\begin{answer}[height=8cm]
% 両辺ともに $0$ 以上なので, \textbf{2乗の差}をとる.
% \begin{align*}
% (\text{右辺})^2 - (\text{左辺})^2 &= (|a|+|b|)^2 - |a+b|^2 \\
% &= (|a|^2 + 2|a||b| + |b|^2) - (a+b)^2 \\
% &= (a^2 + 2|ab| + b^2) - (a^2 + 2ab + b^2) \\
% &= 2(|ab| - ab)
% \end{align*}
% ここで, 一般に $|X| \geqq X$ であるから, $|ab| \geqq ab$.
% よって $2(|ab|-ab) \geqq 0$.
% したがって $(\text{右辺})^2 \geqq (\text{左辺})^2$ となり,
% $|a+b| \leqq |a| + |b|$ が成り立つ. \qed
\end{answer}

\begin{summary}{単元のおわりに}
\small
私たちは数を「複素数」まで広げ, 方程式が必ず解ける完全な世界を手に入れた.
しかし, 同時に「大小関係」という順序を失った.
だからこそ, 私たちが普段扱う「実数」がいかに特別な存在であるか(大小があり, 2乗して正になる), そのありがたみが分かったはずだ.
この「実数の性質」は, 次の単元「図形と方程式」や「三角関数」でも土台として君たちを支えてくれるだろう.
\end{summary}

\end{multicols*}

\newpage

%===========================================================
% 裏面: 演習問題
%===========================================================

\begin{multicols*}{2}

{\large \textbf{Last Challenge!}}

\begin{prac}{1(解の配置問題)}
2次方程式 $x^2 - 2ax + a + 2 = 0$ が, 次のような解をもつように定数 $a$ の値の範囲を定めよ.
\begin{enumerate}
    \item 異なる2つの正の解をもつ
    \item 異符号の解をもつ
\end{enumerate}
\end{prac}

\begin{answer}
\vspace{6cm}
\end{answer}


\begin{prac}{2(絶対値の証明)}
$|a| < 1, \ |b| < 1$ のとき, 次の不等式を証明せよ.
\[ |a+b| < 1 + ab \]
\end{prac}

\begin{any}{Hint}
両辺正であることを確認して, 2乗の差をとる.
$(1+ab)^2 - (a+b)^2 = 1 + 2ab + a^2b^2 - (a^2+2ab+b^2) = 1 - a^2 - b^2 + a^2b^2$.
因数分解すると $(1-a^2)(1-b^2)$. 条件 $|a|<1$ より $1-a^2>0$ ...
\end{any}

\begin{answer}
\vspace{6cm}
\end{answer}

\columnbreak

\begin{prac}{3(パラメータと3次方程式)}
方程式 $x^3 - 3x + 1 = k$ が異なる3つの実数解をもつような定数 $k$ の値の範囲を求めよ.
\end{prac}

\begin{answer}
% y = x^3 - 3x のグラフを描く.
% x(x^2-3) = x(x-√3)(x+√3)
% x=1 -> 1-3=-2 (極小)
% x=-1 -> -1+3=2 (極大)
% -2 < k-1 < 2 ... おっと、式は x^3-3x+1=k なので
% y = x^3 - 3x + 1 と y=k の共有点を見る.
% x=1 -> -1, x=-1 -> 3
% よって -1 < k < 3
\vspace{6cm}
\end{answer}

\begin{any}{Congratulations!}
これで「式と証明・複素数と方程式」の全課程は終了です.
この単元で培った「論理力(証明)」と「計算力(方程式)」は, 数学II・Bのあらゆる場面で武器になります. 自信を持って次に進んでください!
\end{any}

\end{multicols*}

\end{document}