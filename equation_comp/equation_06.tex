\documentclass[b4paper, landscape, dvipdfmx]{jsarticle}
%----- 必要なパッケージ -----
\usepackage{fancybox,ascmac,otf,ulem}
\usepackage{amssymb, amsthm}
\usepackage[leqno]{amsmath}
\usepackage{wrapfig}
\usepackage{geometry}
\usepackage{multicol}
\usepackage{tcolorbox}
\usepackage{xcolor}
\usepackage{fancyhdr}
\usepackage{tikz}

% shadowsライブラリ
\usetikzlibrary{
    positioning,
    arrows.meta,
    calc,
    shadows,
    shadows.blur,
    intersections
}

\tcbuselibrary{skins, breakable, theorems}
\usepackage{enumitem}
\setlist[enumerate,1]{label=(\arabic*)}
\setlist[itemize]{leftmargin=*}
\newcommand{\ds}{\displaystyle}

%----- レイアウト設定 -----
\geometry{
  left=15mm,
  right=15mm,
  top=20mm,
  bottom=15mm,
  headheight=25pt
}

%----- 数式環境の上下の余白調整 -----
\AtBeginDocument{
  \setlength{\abovedisplayskip}{5pt}
  \setlength{\belowdisplayskip}{5pt}
  \setlength{\abovedisplayshortskip}{0pt}
  \setlength{\belowdisplayshortskip}{3pt}
}

%===========================================================
%  デザイン設定
%===========================================================

%--- 色の定義 ---
\definecolor{printBlue}{RGB}{0, 50, 100}     % 濃紺
\definecolor{printRed}{RGB}{140, 20, 20}     % 濃エンジ
\definecolor{printTeal}{RGB}{0, 60, 60}      % 濃い青緑
\definecolor{gridColor}{gray}{0.75}          % 解答欄の方眼

%--- 共通スタイル定義 ---
\tcbset{
    chartbox/.style={
        enhanced,
        fonttitle=\sffamily\bfseries,
        boxrule=1pt,
        arc=2pt,
        top=1.0em,
        nobeforeafter,
        enlarge left by=-2mm,
        enlarge right by=-2mm,
        drop fuzzy shadow,
        colback=white,
        attach boxed title to top left={xshift=10pt, yshift*=-\tcboxedtitleheight/2},
        boxed title style={frame hidden, sharp corners, rounded corners=southeast, arc=3pt}
    }
}

% 各種ボックス環境定義
\newenvironment{overall}[1]{
\begin{tcolorbox}[
    chartbox,
    colframe=printTeal,
    coltitle=white,
    title=\textbf{全体課題 #1},
    boxed title style={colback=printTeal},
]}
{\end{tcolorbox}}

% 解説用ボックス (黒/白) - 親コンテナとして使用
\newtcolorbox{any}[1]{
    enlarge left by=0mm, enlarge right by=0mm,
    enhanced, frame hidden, colback=white, title={#1},
    attach boxed title to top left={xshift=0mm, yshift=0mm},
    coltitle=white, fonttitle=\bfseries\sffamily,
    boxed title style={
        colback=black!80, frame hidden, arc=4pt, outer arc=4pt,
        sharp corners=south, boxrule=0pt,
        top=1mm, bottom=1mm, left=3mm, right=3mm
    },
    underlay boxed title={
        \draw[thick, black!80] (title.south west) -- (title.south west-|frame.east);
    },
    breakable, top=5mm, left=2mm, right=2mm, bottom=0mm,
    before skip=1em, after skip=1em,
    segmentation style={draw=black!40, dashed}
}

\newenvironment{eg}[1]{
\begin{tcolorbox}[
    chartbox,
    colframe=printBlue,
    coltitle=white,
    title=\textbf{例題 #1},
    boxed title style={colback=printBlue},
    segmentation style={draw=printBlue, line width=0.5pt, dashed}
]}
{\end{tcolorbox}}

\newenvironment{prac}[1]{
\begin{tcolorbox}[
    chartbox,
    colframe=printRed,
    coltitle=white,
    title=\textbf{練習 #1},
    boxed title style={colback=printRed}
]}
{\end{tcolorbox}}

\newenvironment{answer}[1][height fill]{
    \begin{tcolorbox}[
        enhanced,
        title={Memo / Answer},
        colframe=black!80,
        colback=white,
        coltitle=black!60,
        fonttitle=\sffamily\bfseries,
        attach boxed title to top left={xshift=5mm, yshift*=-\tcboxedtitleheight/2},
        boxed title style={frame hidden, colback=white},
        boxrule=1pt,
        arc=1pt,
        nobeforeafter,
        enlarge left by=2mm, 
        enlarge right by=2mm, 
        height fill,
        segmentation style={draw=black!20, solid},
        underlay={
            \begin{tcbclipinterior}
                \draw[step=5mm, black!5, ultra thin] (interior.south west) grid (interior.north east);
            \end{tcbclipinterior}
        }, 
        #1
    ]}
{ \end{tcolorbox}}

%----- 段組の設定 -----
\setlength{\columnsep}{15mm}
\setlength{\columnseprule}{0.4pt}
\renewcommand{\columnseprulecolor}{\color{black!30}}

%----- ヘッダーの設定 -----
\pagestyle{fancy}
\fancyhf{}

% ヘッダーデザイン
\fancyhead[C]{%
    \begin{tikzpicture}[remember picture, overlay]
        \node[anchor=north west, fill=printBlue, minimum width=\paperwidth, minimum height=5pt] at (current page.north west) {};
    \end{tikzpicture}
}
\fancyhead[L]{\small \textcolor{black!90}{数学\ajRoman{2} $>$ 第2章 複素数と方程式 $>$ 第6回--\textbf{剰余の定理と因数定理}}}
\fancyhead[R]{\small 年 \hspace{1cm} 組 \hspace{1cm} 番 \quad 氏名 \hspace{6cm}}
\renewcommand{\headrulewidth}{0pt}

\begin{document}

\begin{multicols*}{2}

%===========================================================
% 左カラム: 剰余の定理(基本)
%===========================================================

{\large \textbf{1. 割り算をせずに「余り」だけを知る}}

\begin{any}{仕組み:恒等式の利用}
整式 $P(x)$ を $x-k$ で割ったときの商を $Q(x)$, 余りを $R$ とする.
余り $R$ は定数であり, 次の等式(恒等式)が成り立つ.
\[ P(x) = (x-k)Q(x) + R \]
この式の両辺に $\boldsymbol{x=k}$ を代入すると...
\[ P(k) = \underbrace{(k-k)Q(k)}_{0} + R \quad \therefore R = P(k) \]
つまり, \textbf{代入するだけで余りが求まる!}
\end{any}

\begin{any}{定理:剰余の定理}
\begin{center}
\begin{tcolorbox}[colframe=printRed, colback=white, boxrule=1.5pt, arc=0pt]
    整式 $P(x)$ を 1次式 $\boldsymbol{x-k}$ で割った余りは, $\boldsymbol{P(k)}$ に等しい.
\end{tcolorbox}
\end{center}
\begin{itemize}
    \item $x-1$ で割った余り $\to$ $x=1$ を代入 ($P(1)$)
    \item $x+2$ で割った余り $\to$ $x=-2$ を代入 ($P(-2)$)
\end{itemize}
\end{any}

\begin{eg}{1(剰余の定理の利用)}
整式 $P(x) = x^3 - 3x^2 + 4x - 5$ を, 次の1次式で割った余りを求めよ.
\begin{enumerate}
    \item $x-2$
    \item $x+1$
\end{enumerate}
\end{eg}

\begin{answer}
% 解説
% (1) P(2) = 8 - 12 + 8 - 5 = -1
% (2) P(-1) = -1 - 3 - 4 - 5 = -13
\vspace{4cm}
\end{answer}

\columnbreak

%===========================================================
% 右カラム: 剰余の定理(拡張)
%===========================================================

{\large \textbf{2. 1次式 $\boldsymbol{ax+b}$ で割る場合}}

\begin{any}{割る式が0になる値を代入せよ}
$ax+b$ で割った余りを求めたい場合も, 同様に考える.
商を $Q(x)$, 余りを $R$ とすると
\[ P(x) = (ax+b)Q(x) + R \]
右辺の $(ax+b)$ を $0$ にするためには, $\boldsymbol{x = -\frac{b}{a}}$ を代入すればよい.

\begin{center}
\begin{tcolorbox}[colframe=printRed, colback=white, boxrule=1.5pt, arc=0pt]
    整式 $P(x)$ を $\boldsymbol{ax+b}$ で割った余りは, $\ds \boldsymbol{P\left(-\frac{b}{a}\right)}$ に等しい.
\end{tcolorbox}
\end{center}
\end{any}

\begin{eg}{2($ax+b$ で割る場合)}
整式 $P(x) = 2x^3 - 3x^2 - x + 2$ を, 次の1次式で割った余りを求めよ.
\begin{enumerate}
    \item $2x-1$
    \item $2x+3$
\end{enumerate}
\end{eg}

\begin{answer}
% 解説
% (1) x=1/2を代入
% P(1/2) = 2(1/8) - 3(1/4) - 1/2 + 2 
%        = 1/4 - 3/4 - 2/4 + 8/4 = 4/4 = 1
% (2) x=-3/2を代入
\vspace{6cm}
\end{answer}

\end{multicols*}

\begin{multicols*}{2}

%===========================================================
% 左カラム: 因数定理の導入
%===========================================================

{\large \textbf{3. 因数定理}}

\begin{any}{「余りが0」の意味}
剰余の定理 $P(k)=R$ において, 特に余り $R$ が $0$ になる場合を考える.
\[ P(k)=0 \iff P(x) \text{ は } x-k \text{ で割り切れる} \]
割り切れるということは, $P(x)=(x-k)Q(x)$ の形に変形できるということ.
つまり, \textbf{因数分解できる!}ということである.
\end{any}

\begin{any}{因数定理}
\begin{center}
\begin{tcolorbox}[colframe=printRed, colback=white, boxrule=1.5pt, arc=0pt]
    整式 $P(x)$ について,
    \begin{itemize}
        \item $\boldsymbol{P(k)=0 \iff P(x) \text{ は } x-k \text{ を因数にもつ}}$
    \end{itemize}
\end{tcolorbox}
\end{center}
「代入して0になる値」を見つければ, それをヒントに因数分解ができる.
\end{any}

\begin{eg}{3(因数を見つける)}
$P(x) = x^3 - 4x^2 + x + 6$ とする.
\begin{enumerate}
    \item $P(-1)$ の値を求めよ.
    \item $P(x)$ が因数としてもつ1次式を答えよ.
\end{enumerate}
\end{eg}

\begin{answer}
% 解説
% (1) P(-1) = -1 - 4 - 1 + 6 = 0
% (2) よって P(x) は x-(-1), つまり x+1 を因数にもつ.
\vspace{4cm}
\end{answer}

\columnbreak

%===========================================================
% 右カラム: 高次式の因数分解
%===========================================================

{\large \textbf{4. 因数定理を利用した因数分解}}

\begin{any}{候補の探し方}
$P(k)=0$ となる $k$ の候補は, 次の数の中から探すとよい.
\[ \pm \frac{\text{定数項の約数}}{\text{最高次の係数の約数}} \]
(基本的には $\pm 1, \pm 2, \pm 3$ など, \textbf{定数項の約数}を順に代入して探す)
\end{any}

\begin{eg}{4(3次式の因数分解)}
$x^3 - 4x^2 + x + 6$ を因数分解せよ.
\end{eg}

\begin{any}{手順}
\begin{enumerate}
    \item $P(k)=0$ となる $k$ を見つける. $\to$ $x=-1$ で $0$ になった!
    \item $P(x)$ を $(x+1)$ で割り算する(筆算 or 組立除法).
    \item $P(x) = (x+1)(\text{2次式})$ の形になる.
    \item 後ろの2次式をさらに因数分解する.
\end{enumerate}
\end{any}

\begin{answer}
% 筆算スペース
% (x^3 -4x^2 +x +6) ÷ (x+1) = x^2 -5x +6
% よって (x+1)(x^2 -5x +6)
% = (x+1)(x-2)(x-3)
\vspace{6cm}
\end{answer}
\end{multicols*}

\newpage

%===========================================================
% 裏面: 演習問題
%===========================================================

\begin{multicols*}{2}

{\large \textbf{確認テスト}}

\begin{prac}{A1(基本計算)}
整式 $P(x) = x^3 + 2x^2 - 5x + 1$ を, 次の式で割った余りを求めよ.
\begin{enumerate}
    \item $x-1$
    \item $x-3$
    \item $x+2$
\end{enumerate}
\end{prac}

\begin{prac}{A2($ax+b$ で割る)}
整式 $P(x) = 4x^3 - 2x^2 + 3$ を, 次の式で割った余りを求めよ.
\begin{enumerate}
    \item $2x+1$
    \item $2x-3$
\end{enumerate}
\end{prac}

\begin{answer}

\end{answer}

\columnbreak

\begin{prac}{B1(余りからの係数決定)}
整式 $P(x) = x^3 + kx^2 - 3x + 2$ を $x-2$ で割った余りが $4$ であるとき, 定数 $k$ の値を求めよ.
\end{prac}

\begin{prac}{B2(2次式で割った余り)}
整式 $P(x)$ を $x-1$ で割ると $3$ 余り, $x+2$ で割ると $-3$ 余る.
このとき, $P(x)$ を $(x-1)(x+2)$ で割った余りを求めよ.
\end{prac}

\begin{any}{Hint}
2次式で割った余りは「1次以下の式」になるので, $\boldsymbol{ax+b}$ とおける.
\[ P(x) = (x-1)(x+2)Q(x) + ax+b \]
この恒等式に $x=1$ と $x=-2$ を代入して, $a, b$ の連立方程式を作ろう.
\end{any}

\begin{answer}
% P(1) = a+b = 3
% P(-2) = -2a+b = -3
% 引き算: 3a=6 -> a=2, b=1
% 余り: 2x+1
\end{answer}

\end{multicols*}

\begin{multicols*}{2}

{\large \textbf{確認テスト}}

\begin{prac}{A1(基本計算)}
整式 $P(x) = x^3 + 2x^2 - 5x + 1$ を, 次の式で割った余りを求めよ.
\begin{enumerate}
    \item $x-1$
    \item $x-3$
    \item $x+2$
\end{enumerate}
\end{prac}

\begin{prac}{A2($ax+b$ で割る)}
整式 $P(x) = 4x^3 - 2x^2 + 3$ を, 次の式で割った余りを求めよ.
\begin{enumerate}
    \item $2x+1$
    \item $2x-3$
\end{enumerate}
\end{prac}

\begin{answer}
\color{printRed}
\textbf{A1}
\begin{enumerate}
    \item $P(1) = 1^3 + 2(1)^2 - 5(1) + 1 = 1 + 2 - 5 + 1 = \boldsymbol{-1}$
    \item $P(3) = 3^3 + 2(3)^2 - 5(3) + 1 = 27 + 18 - 15 + 1 = \boldsymbol{31}$
    \item $P(-2) = (-2)^3 + 2(-2)^2 - 5(-2) + 1 = -8 + 8 + 10 + 1 = \boldsymbol{11}$
\end{enumerate}

\vspace{0.5em}
\hrule
\vspace{0.5em}

\textbf{A2}
\begin{enumerate}
    \item $x=-\frac{1}{2}$ を代入. \\
    $\ds P\left(-\frac{1}{2}\right) = 4\left(-\frac{1}{8}\right) - 2\left(\frac{1}{4}\right) + 3 = -\frac{1}{2} - \frac{1}{2} + 3 = \boldsymbol{2}$
    \item $x=\frac{3}{2}$ を代入. \\
    $\ds P\left(\frac{3}{2}\right) = 4\left(\frac{27}{8}\right) - 2\left(\frac{9}{4}\right) + 3 = \frac{27}{2} - \frac{9}{2} + 3 = 9 + 3 = \boldsymbol{12}$
\end{enumerate}
\end{answer}

\columnbreak

\begin{prac}{B1(余りからの係数決定)}
整式 $P(x) = x^3 + kx^2 - 3x + 2$ を $x-2$ で割った余りが $4$ であるとき, 定数 $k$ の値を求めよ.
\end{prac}

\begin{prac}{B2(2次式で割った余り)}
整式 $P(x)$ を $x-1$ で割ると $3$ 余り, $x+2$ で割ると $-3$ 余る.
このとき, $P(x)$ を $(x-1)(x+2)$ で割った余りを求めよ.
\end{prac}

\begin{any}{Hint}
2次式で割った余りは「1次以下の式」になるので, $\boldsymbol{ax+b}$ とおける.
\[ P(x) = (x-1)(x+2)Q(x) + ax+b \]
この恒等式に $x=1$ と $x=-2$ を代入して, $a, b$ の連立方程式を作ろう.
\end{any}

\begin{answer}
\color{printRed}
\textbf{B1} \\
剰余の定理より $P(2)=4$ となればよい. \\
$P(2) = 2^3 + k(2)^2 - 3(2) + 2 = 8 + 4k - 6 + 2 = 4k + 4$ \\
よって $4k+4=4 \iff 4k=0 \iff \boldsymbol{k=0}$

\vspace{0.5em}
\hrule
\vspace{0.5em}

\textbf{B2} \\
求める余りを $ax+b$ とおくと, 条件より
\[ \begin{cases} P(1) = a+b = 3 \\ P(-2) = -2a+b = -3 \end{cases} \]
辺々引くと $3a=6 \iff a=2$. \\
これを代入して $2+b=3 \iff b=1$. \\
よって求める余りは $\boldsymbol{2x+1}$
\end{answer}

\end{multicols*}

\begin{multicols*}{2}

{\large \textbf{確認テスト}}

\begin{prac}{A3(因数の判定)}
整式 $P(x) = x^3 - 2x^2 - 5x + 6$ について, 次の1次式のうち, $P(x)$ の因数であるものを選べ.
\[ x-1, \quad x+1, \quad x-2, \quad x-3 \]
\end{prac}


\begin{prac}{A4(因数分解の実践)}
次の式を因数分解せよ.
\begin{enumerate}
    \item $x^3 - 3x^2 - 4x + 12$ \quad ($P(2)$や$P(3)$を試そう)
    \item $x^3 + 4x^2 + x - 6$
\end{enumerate}
\end{prac}

\begin{answer}
\vspace{6cm}
\end{answer}

\columnbreak

\begin{prac}{B3(係数がついている場合)}
$2x^3 - 3x^2 - 11x + 6$ を因数分解せよ.
\end{prac}

\begin{any}{Hint}
定数項 $6$ の約数 $(\pm1, \pm2, \pm3, \pm6)$ を代入しても $0$ にならない場合は, 分数 $\pm \frac{1}{2}$ などを試す必要がある.
\[ P(3) = 54 - 27 - 33 + 6 = 0 \]
(おっと, 今回は整数 $3$ で見つかりますね. $x-3$ で割ってみよう)
\end{any}

\begin{prac}{B4(因数をもつ条件)}
整式 $P(x) = x^3 + (k+1)x^2 - x - 3$ が $x-1$ を因数にもつとき, 定数 $k$ の値を求めよ. また, そのときの $P(x)$ を因数分解せよ.
\end{prac}

\begin{answer}
% P(1) = 1 + k + 1 - 1 - 3 = k - 2 = 0 -> k=2
% P(x) = x^3 + 3x^2 - x - 3
%      = x^2(x+3) - (x+3) = (x^2-1)(x+3)
%      = (x+1)(x-1)(x+3)
\end{answer}

\end{multicols*}

\begin{multicols*}{2}

{\large \textbf{確認テスト}}

\begin{prac}{A3(因数の判定)}
整式 $P(x) = x^3 - 2x^2 - 5x + 6$ について, 次の1次式のうち, $P(x)$ の因数であるものを選べ.
\[ x-1, \quad x+1, \quad x-2, \quad x-3 \]
\end{prac}

\begin{prac}{A4(因数分解の実践)}
次の式を因数分解せよ.
\begin{enumerate}
    \item $x^3 - 3x^2 - 4x + 12$ \quad ($P(2)$や$P(3)$を試そう)
    \item $x^3 + 4x^2 + x - 6$
\end{enumerate}
\end{prac}

\begin{answer}
\color{printRed}
\textbf{A3} \\
剰余の定理より $P(k)=0$ となるものを確認する.
\begin{itemize}
    \item $P(1) = 1 - 2 - 5 + 6 = 0$ \quad $\to$ \textbf{○ ($x-1$)}
    \item $P(-1) = -1 - 2 + 5 + 6 = 8 \neq 0$
    \item $P(2) = 8 - 8 - 10 + 6 = -4 \neq 0$
    \item $P(3) = 27 - 18 - 15 + 6 = 0$ \quad $\to$ \textbf{○ ($x-3$)}
\end{itemize}
よって答えは $\boldsymbol{x-1, \quad x-3}$

\vspace{0.5em}
\hrule
\vspace{0.5em}

\textbf{A4}
\begin{enumerate}
    \item $P(2) = 8 - 12 - 8 + 12 = 0$ より $x-2$ を因数にもつ. \\
    $(x-2)(x^2-x-6) = \boldsymbol{(x-2)(x+2)(x-3)}$
    \item $P(1) = 1 + 4 + 1 - 6 = 0$ より $x-1$ を因数にもつ. \\
    $(x-1)(x^2+5x+6) = \boldsymbol{(x-1)(x+2)(x+3)}$
\end{enumerate}
\end{answer}

\columnbreak

\begin{prac}{B3(係数がついている場合)}
$2x^3 - 3x^2 - 11x + 6$ を因数分解せよ.
\end{prac}

\begin{any}{Hint}
定数項 $6$ の約数 $(\pm1, \pm2, \pm3, \pm6)$ を代入しても $0$ にならない場合は, 分数 $\pm \frac{1}{2}$ などを試す必要がある.
\[ P(3) = 54 - 27 - 33 + 6 = 0 \]
(おっと, 今回は整数 $3$ で見つかりますね. $x-3$ で割ってみよう)
\end{any}

\begin{prac}{B4(因数をもつ条件)}
整式 $P(x) = x^3 + (k+1)x^2 - x - 3$ が $x-1$ を因数にもつとき, 定数 $k$ の値を求めよ. また, そのときの $P(x)$ を因数分解せよ.
\end{prac}

\begin{answer}
\color{printRed}
\textbf{B3} \\
ヒントより $x-3$ を因数にもつ.
筆算または組立除法で割り算を行うと
\[ (x-3)(2x^2 + 3x - 2) \]
後ろの2次式をたすき掛けで因数分解して,
\[ \boldsymbol{(x-3)(x+2)(2x-1)} \]

\vspace{0.5em}
\hrule
\vspace{0.5em}

\textbf{B4} \\
$x-1$ を因数にもつ $\iff P(1)=0$.
\[ P(1) = 1 + (k+1) - 1 - 3 = k - 2 = 0 \iff \boldsymbol{k=2} \]
このとき $P(x) = x^3 + 3x^2 - x - 3$. \\
$P(1)=0$ なので $x-1$ で割ると,
\[ (x-1)(x^2 + 4x + 3) = \boldsymbol{(x-1)(x+1)(x+3)} \]
\end{answer}

\end{multicols*}

\end{document}