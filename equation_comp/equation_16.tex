\documentclass[b4paper, landscape, dvipdfmx]{jsarticle}
%----- 必要なパッケージ -----
\usepackage{fancybox,ascmac,otf,ulem}
\usepackage{amssymb, amsthm}
\usepackage[leqno]{amsmath}
\usepackage{geometry}
\usepackage{multicol}
\usepackage{tcolorbox}
\usepackage{xcolor}
\usepackage{fancyhdr}
\usepackage{tikz}

% shadowsライブラリ
\usetikzlibrary{
    positioning,
    arrows.meta,
    calc,
    shadows,
    shadows.blur,
    intersections
}

\tcbuselibrary{skins, breakable, theorems}
\usepackage{enumitem}
\setlist[enumerate,1]{label=(\arabic*)}
\setlist[itemize]{leftmargin=*}
\newcommand{\ds}{\displaystyle}

%----- レイアウト設定 -----
\geometry{
  left=15mm,
  right=15mm,
  top=20mm,
  bottom=15mm,
  headheight=25pt
}

%----- 数式環境の上下の余白調整 -----
\AtBeginDocument{
  \setlength{\abovedisplayskip}{5pt}
  \setlength{\belowdisplayskip}{5pt}
  \setlength{\abovedisplayshortskip}{0pt}
  \setlength{\belowdisplayshortskip}{3pt}
}

%===========================================================
%  デザイン設定
%===========================================================

%--- 色の定義 ---
\definecolor{printBlue}{RGB}{0, 50, 100}     % 濃紺
\definecolor{printRed}{RGB}{140, 20, 20}     % 濃エンジ
\definecolor{printTeal}{RGB}{0, 60, 60}      % 濃い青緑
\definecolor{printOrange}{RGB}{200, 100, 0}  % まとめ用オレンジ

%--- 共通スタイル定義 ---
\tcbset{
    chartbox/.style={
        enhanced,
        fonttitle=\sffamily\bfseries,
        boxrule=1pt,
        arc=2pt,
        top=1.0em,
        nobeforeafter,
        enlarge left by=-2mm,
        enlarge right by=-2mm,
        drop fuzzy shadow,
        colback=white,
        attach boxed title to top left={xshift=10pt, yshift*=-\tcboxedtitleheight/2},
        boxed title style={frame hidden, sharp corners, rounded corners=southeast, arc=3pt}
    }
}

% 各種ボックス環境定義
\newenvironment{summary}[1]{
    \begin{tcolorbox}[
        chartbox,
        colframe=printOrange,
        coltitle=white,
        title=\textbf{#1},
        boxed title style={colback=printOrange},
    ]}
{ \end{tcolorbox}}

\newtcolorbox{any}[1]{
    enlarge left by=0mm, enlarge right by=0mm,
    enhanced, frame hidden, colback=white, title={#1},
    attach boxed title to top left={xshift=0mm, yshift=0mm},
    coltitle=white, fonttitle=\bfseries\sffamily,
    boxed title style={
        colback=black!80, frame hidden, arc=4pt, outer arc=4pt,
        sharp corners=south, boxrule=0pt,
        top=1mm, bottom=1mm, left=3mm, right=3mm
    },
    underlay boxed title={
        \draw[thick, black!80] (title.south west) -- (title.south west-|frame.east);
    },
    breakable, top=5mm, left=2mm, right=2mm, bottom=0mm,
    before skip=1em, after skip=1em,
    segmentation style={draw=black!40, dashed}
}

\newenvironment{eg}[1]{
\begin{tcolorbox}[
    chartbox,
    colframe=printBlue,
    coltitle=white,
    title=\textbf{例題 #1},
    boxed title style={colback=printBlue},
    segmentation style={draw=printBlue, line width=0.5pt, dashed}
]}
{\end{tcolorbox}}

\newenvironment{prac}[1]{
\begin{tcolorbox}[
    chartbox,
    colframe=printRed,
    coltitle=white,
    title=\textbf{練習 #1},
    boxed title style={colback=printRed}
]}
{\end{tcolorbox}}

\newenvironment{answer}[1][height fill]{
    \begin{tcolorbox}[
        enhanced,
        title={Memo / Answer},
        colframe=black!80,
        colback=white,
        coltitle=black!60,
        fonttitle=\sffamily\bfseries,
        attach boxed title to top left={xshift=5mm, yshift*=-\tcboxedtitleheight/2},
        boxed title style={frame hidden, colback=white},
        boxrule=1pt,
        arc=1pt,
        nobeforeafter,
        enlarge left by=2mm, 
        enlarge right by=2mm, 
        height fill,
        segmentation style={draw=black!20, solid},
        underlay={
            \begin{tcbclipinterior}
                \draw[step=5mm, black!5, ultra thin] (interior.south west) grid (interior.north east);
            \end{tcbclipinterior}
        }, 
        #1
    ]}
{ \end{tcolorbox}}

%----- 段組の設定 -----
\setlength{\columnsep}{15mm}
\setlength{\columnseprule}{0.4pt}
\renewcommand{\columnseprulecolor}{\color{black!30}}

%----- ヘッダーの設定 -----
\pagestyle{fancy}
\fancyhf{}

% ヘッダーデザイン
\fancyhead[C]{%
    \begin{tikzpicture}[remember picture, overlay]
        \node[anchor=north west, fill=printBlue, minimum width=\paperwidth, minimum height=5pt] at (current page.north west) {};
    \end{tikzpicture}
}
\fancyhead[L]{\small \textcolor{black!90}{数学\ajRoman{2} $>$ 第3章 式と証明 $>$ 第16回--\textbf{単元の総整理(1)}}}
\fancyhead[R]{\small 年 \hspace{1cm} 組 \hspace{1cm} 番 \quad 氏名 \hspace{6cm}}
\renewcommand{\headrulewidth}{0pt}

\begin{document}

\begin{multicols*}{2}

%===========================================================
% 左カラム: 構造の整理
%===========================================================

{\large \textbf{1. 全体像再訪(Map)}}

\begin{summary}{この単元で学んだこと}
\small
\begin{itemize}
    \item \textbf{数の拡張}: 実数 $\mathbb{R} \to$ 複素数 $\mathbb{C}$
    \item \textbf{式の世界}: 恒等式, 方程式, 不等式
\end{itemize}
それぞれの世界で「できること」「できないこと」を整理しよう.
\end{summary}

\begin{any}{「実数」vs「複素数」}
\begin{center}
\begin{tabular}{|c|c|c|}
    \hline
    項目 & \textbf{実数の世界} ($\mathbb{R}$) & \textbf{複素数の世界} ($\mathbb{C}$) \\
    \hline
    2乗すると & $\boldsymbol{x^2 \geqq 0}$ & 負になることもある ($i^2=-1$) \\
    \hline
    大小関係 & \textbf{ある} ($>, <$) & \textbf{ない} (定義できない) \\
    \hline
    方程式の解 & 解なしの場合あり & $\boldsymbol{n}$ \textbf{次式は必ず} $\boldsymbol{n}$ \textbf{個持つ} \\
    \hline
    証明 & 不等式の証明が可能 & 等式の証明がメイン \\
    \hline
\end{tabular}
\end{center}
\end{any}

\begin{any}{「恒等式」vs「方程式」}
\textbf{恒等式}(Identity):
\begin{itemize}
    \item どんな $x$ でも成り立つ. ($x$ にとらわれない)
    \item 例: $(x+1)^2 = x^2+2x+1$
    \item 手法: 係数比較法, 数値代入法
\end{itemize}
\textbf{方程式}(Equation):
\begin{itemize}
    \item 特定の $x$ (解) でしか成り立たない.
    \item 例: $x^2+2x+1 = 0$
    \item 手法: 因数分解, 解の公式, グラフの共有点
\end{itemize}
\end{any}

\begin{eg}{1(概念の確認)}
次の等式が「恒等式」であるか, 「方程式」であるか答えよ.
\begin{enumerate}
    \item $x^2 - 1 = (x+1)(x-1)$
    \item $x^2 - 1 = 0$
    \item $\frac{1}{x(x+1)} = \frac{1}{x} - \frac{1}{x+1}$
\end{enumerate}
\end{eg}

\begin{answer}
\vspace{3cm}
\end{answer}

\columnbreak

%===========================================================
% 右カラム: 融合問題
%===========================================================

{\large \textbf{2. 融合問題演習}}

\begin{any}{複素数と式の値}
「解と係数の関係」や「次数の低下」を駆使する問題.
直接代入は最終手段!
\end{any}

\begin{eg}{2(次数下げ)}
$x = 1+\sqrt{2}i$ のとき, 次の式の値を求めよ.
\[ x^3 - x^2 + 3x + 5 \]
\end{eg}

\begin{any}{方針}
$x-1=\sqrt{2}i$ として両辺を2乗すると, $x^2-2x+1=-2$, つまり $x^2-2x+3=0$ が得られる.
この 2次式 $x^2-2x+3$ で元の式を\textbf{割り算}する.
\[ (\text{元の式}) = (x^2-2x+3)Q(x) + R(x) \]
$x^2-2x+3=0$ なので, 結局 \textbf{余り $R(x)$ に代入すればよい}.
\end{any}

\begin{answer}
% x^2 - 2x + 3 = 0
% 割り算:
% x^3 - x^2 + 3x + 5 = (x^2-2x+3)(x+1) + 2x + 2
% = 0 + 2(1+√2i) + 2 = 4 + 2√2i
\vspace{6cm}
\end{answer}

\begin{any}{不等式の証明と相加相乗}
条件を見て「相加・相乗平均」を使うか, 「平方完成」を使うか判断する.
\begin{itemize}
    \item 「$a>0$」や「積が定数」 $\to$ \textbf{相加・相乗}
    \item 「実数」や「2次式」 $\to$ \textbf{平方完成 ($A^2 \geqq 0$)}
\end{itemize}
\end{any}

\end{multicols*}

\newpage

%===========================================================
% 裏面: 演習問題
%===========================================================

\begin{multicols*}{2}

{\large \textbf{総合演習}}

\begin{prac}{A1(恒等式の係数決定)}
等式 $x^3 = a(x-1)^3 + b(x-1)^2 + c(x-1) + d$ が $x$ についての恒等式となるように, 定数 $a, b, c, d$ の値を定めよ.
\end{prac}

\begin{any}{Hint}
右辺を展開して係数比較してもよいが, $x-1$ の塊に注目し, 組立除法を繰り返し適用する方法(テイラー展開的発想)もある.
あるいは, $x=1, 2, 0, -1$ などを代入してみよう.
\end{any}

\begin{answer}
\vspace{5cm}
\end{answer}


\begin{prac}{A2(高次方程式と虚数解)}
3次方程式 $x^3 - 3x^2 + ax + b = 0$ の1つの解が $1-i$ であるとき, 実数の定数 $a, b$ の値を求めよ. また, 他の解を求めよ.
\end{prac}

\begin{answer}
% 1-i が解なら 1+i も解.
% (x-(1-i))(x-(1+i)) = x^2-2x+2
% 割り算して割り切れる条件, または解と係数の関係(3次)
% 残りの解は x=1
\vspace{5cm}
\end{answer}

\columnbreak

\begin{prac}{B1(条件付き不等式の証明)}
$a+b=1$ のとき, 不等式 $a^2 + b^2 \geqq \frac{1}{2}$ を証明せよ.
また, 等号が成立するのはどのようなときか.
\end{prac}

\begin{answer}
% b = 1-a を代入
% a^2 + (1-a)^2 - 1/2 = 2a^2 - 2a + 1/2 = 2(a - 1/2)^2 >= 0
% 等号成立 a=1/2, b=1/2
\vspace{6cm}
\end{answer}

\begin{prac}{B2(相加・相乗平均の応用)}
$a>0, b>0$ とする. $(a+b)\left(\frac{1}{a}+\frac{4}{b}\right)$ の最小値を求めよ.
\end{prac}

\begin{answer}
% 展開すると 1 + 4a/b + b/a + 4 = 5 + 4a/b + b/a
% 4a/b + b/a >= 2√4 = 4
% 最小値 5+4=9
% 等号 4a/b = b/a -> b^2=4a^2 -> b=2a
\end{answer}

\end{multicols*}

\end{document}