\documentclass[b4paper, landscape, dvipdfmx]{jsarticle}
%----- 必要なパッケージ -----
\usepackage{fancybox,ascmac,otf,ulem}
\usepackage{amssymb, amsthm}
\usepackage[leqno]{amsmath}
\usepackage{geometry}
\usepackage{multicol}
\usepackage{tcolorbox}
\usepackage{xcolor}
\usepackage{fancyhdr}
\usepackage{tikz}

% shadowsライブラリ
\usetikzlibrary{
    positioning,
    arrows.meta,
    calc,
    shadows,
    shadows.blur,
    intersections
}

\tcbuselibrary{skins, breakable, theorems}
\usepackage{enumitem}
\setlist[enumerate,1]{label=(\arabic*)}
\setlist[itemize]{leftmargin=*}
\newcommand{\ds}{\displaystyle}

%----- レイアウト設定 -----
\geometry{
  left=15mm,
  right=15mm,
  top=20mm,
  bottom=15mm,
  headheight=25pt
}

%----- 数式環境の上下の余白調整 -----
\AtBeginDocument{
  \setlength{\abovedisplayskip}{5pt}
  \setlength{\belowdisplayskip}{5pt}
  \setlength{\abovedisplayshortskip}{0pt}
  \setlength{\belowdisplayshortskip}{3pt}
}

%===========================================================
%  デザイン設定
%===========================================================

%--- 色の定義 ---
\definecolor{printBlue}{RGB}{0, 50, 100}     % 濃紺
\definecolor{printRed}{RGB}{140, 20, 20}     % 濃エンジ
\definecolor{printTeal}{RGB}{0, 60, 60}      % 濃い青緑

%--- 共通スタイル定義 ---
\tcbset{
    chartbox/.style={
        enhanced,
        fonttitle=\sffamily\bfseries,
        boxrule=1pt,
        arc=2pt,
        top=1.0em,
        nobeforeafter,
        enlarge left by=-2mm,
        enlarge right by=-2mm,
        drop fuzzy shadow,
        colback=white,
        attach boxed title to top left={xshift=10pt, yshift*=-\tcboxedtitleheight/2},
        boxed title style={frame hidden, sharp corners, rounded corners=southeast, arc=3pt}
    }
}

% 各種ボックス環境定義
\newtcolorbox{any}[1]{
    enlarge left by=0mm, enlarge right by=0mm,
    enhanced, frame hidden, colback=white, title={#1},
    attach boxed title to top left={xshift=0mm, yshift=0mm},
    coltitle=white, fonttitle=\bfseries\sffamily,
    boxed title style={
        colback=black!80, frame hidden, arc=4pt, outer arc=4pt,
        sharp corners=south, boxrule=0pt,
        top=1mm, bottom=1mm, left=3mm, right=3mm
    },
    underlay boxed title={
        \draw[thick, black!80] (title.south west) -- (title.south west-|frame.east);
    },
    breakable, top=5mm, left=2mm, right=2mm, bottom=0mm,
    before skip=1em, after skip=1em,
    segmentation style={draw=black!40, dashed}
}

\newenvironment{eg}[1]{
\begin{tcolorbox}[
    chartbox,
    colframe=printBlue,
    coltitle=white,
    title=\textbf{例題 #1},
    boxed title style={colback=printBlue},
    segmentation style={draw=printBlue, line width=0.5pt, dashed}
]}
{\end{tcolorbox}}

\newenvironment{prac}[1]{
\begin{tcolorbox}[
    chartbox,
    colframe=printRed,
    coltitle=white,
    title=\textbf{練習 #1},
    boxed title style={colback=printRed}
]}
{\end{tcolorbox}}

\newenvironment{answer}[1][height fill]{
    \begin{tcolorbox}[
        enhanced,
        title={Memo / Answer},
        colframe=black!80,
        colback=white,
        coltitle=black!60,
        fonttitle=\sffamily\bfseries,
        attach boxed title to top left={xshift=5mm, yshift*=-\tcboxedtitleheight/2},
        boxed title style={frame hidden, colback=white},
        boxrule=1pt,
        arc=1pt,
        nobeforeafter,
        enlarge left by=2mm, 
        enlarge right by=2mm, 
        height fill,
        segmentation style={draw=black!20, solid},
        underlay={
            \begin{tcbclipinterior}
                \draw[step=5mm, black!5, ultra thin] (interior.south west) grid (interior.north east);
            \end{tcbclipinterior}
        }, 
        #1
    ]}
{ \end{tcolorbox}}

\newenvironment{proofbox}[1][height fill]{
    \begin{tcolorbox}[
        enhanced,
        title={Proof},
        colframe=black!80,
        colback=white,
        coltitle=black!60,
        fonttitle=\sffamily\bfseries,
        attach boxed title to top left={xshift=5mm, yshift*=-\tcboxedtitleheight/2},
        boxed title style={frame hidden, colback=white},
        boxrule=1pt,
        arc=1pt,
        nobeforeafter,
        width=\linewidth,
        underlay={
            \begin{tcbclipinterior}
                \draw[step=5mm, black!5, ultra thin] (interior.south west) grid (interior.north east);
            \end{tcbclipinterior}
        },
        #1
    ]}
{ \end{tcolorbox}}

%----- 段組の設定 -----
\setlength{\columnsep}{15mm}
\setlength{\columnseprule}{0.4pt}
\renewcommand{\columnseprulecolor}{\color{black!30}}

%----- ヘッダーの設定 -----
\pagestyle{fancy}
\fancyhf{}

% ヘッダーデザイン
\fancyhead[C]{%
    \begin{tikzpicture}[remember picture, overlay]
        \node[anchor=north west, fill=printBlue, minimum width=\paperwidth, minimum height=5pt] at (current page.north west) {};
    \end{tikzpicture}
}
\fancyhead[L]{\small \textcolor{black!90}{数学\ajRoman{2} $>$ 第3章 式と証明 $>$ 第12
回--\textbf{不等式の証明(1)}}}
\fancyhead[R]{\small 年 \hspace{1cm} 組 \hspace{1cm} 番 \quad 氏名 \hspace{6cm}}
\renewcommand{\headrulewidth}{0pt}

\begin{document}

\begin{multicols*}{2}

%===========================================================
% 左カラム: 実数の大小関係
%===========================================================

{\large \textbf{1. 実数だけの特権:大小関係}}

\begin{any}{虚数に大小はない!}
私たちはこれまで複素数 ($a+bi$) を学んできたが, 実は\textbf{虚数には大小関係(不等号)が存在しない}.
もし $i > 0$ だと仮定すると, 両辺に $i$ を掛けて $i^2 > 0$ となり, $-1 > 0$ という矛盾が生じる. ($i < 0$ としても同様)

\vspace{0.5em}
\textbf{結論:} 不等式の問題が出たら, それは\textbf{「実数の世界の話ですよ」}という合図である.
\end{any}

\begin{any}{不等式証明の基本原理}
$A$ が $B$ より大きいことを証明するには, \textbf{「引いたらプラスになる」}ことを言えばよい.

\begin{center}
\begin{tcolorbox}[colframe=printRed, colback=white, boxrule=1.5pt, arc=0pt]
    \centering
    $\ds A > B \iff A - B > 0 $
\end{tcolorbox}
\end{center}
\textbf{手順:}
\begin{enumerate}
    \item 左辺 $-$ 右辺 ($A-B$) を計算する.
    \item 式変形(因数分解や平方完成)をする.
    \item その式が「正である」といえる根拠を示す.
\end{enumerate}
\end{any}

\begin{eg}{1(基本的な不等式の証明)}
$a>b$ かつ $x>y$ のとき, 次の不等式を証明せよ.
\[ ax + by > ay + bx \]
\end{eg}

\begin{proofbox}
% $(\text{左辺}) - (\text{右辺}) = ax + by - ay - bx$ \\
% $\phantom{(\text{左辺}) - (\text{右辺})} = a(x-y) - b(x-y)$ \\
% $\phantom{(\text{左辺}) - (\text{右辺})} = (a-b)(x-y)$

% ここで, 仮定より $a>b, \ x>y$ であるから,
% \[ a-b>0, \quad x-y>0 \]
% よって, 正の数同士の積は正であるから
% \[ (a-b)(x-y) > 0 \]
% したがって, $(\text{左辺}) - (\text{右辺}) > 0$ より
% \[ ax + by > ay + bx \quad \qed \]
\end{proofbox}

\columnbreak

%===========================================================
% 右カラム: 因数分解を利用する証明
%===========================================================

{\large \textbf{2. 条件を利用する証明}}

\begin{any}{「1より大きい」条件の使い方}
「$a>1$」という条件は, 「$a-1>0$」という形で使うことが多い.
式変形して $\boldsymbol{(\text{正}) \times (\text{正})}$ の形を作り出すのが目標となる.
\end{any}

\begin{eg}{2(条件付き不等式)}
$a>1, \ b>1$ のとき, 次の不等式を証明せよ.
\[ ab + 1 > a + b \]
\end{eg}

\begin{proofbox}[height=11cm]
% $(\text{左辺}) - (\text{右辺}) = ab + 1 - a - b$ \\
% $\phantom{(\text{左辺}) - (\text{右辺})} = a(b-1) - (b-1)$ \\
% $\phantom{(\text{左辺}) - (\text{右辺})} = (a-1)(b-1)$

% ここで, $a>1, \ b>1$ より
% \[ a-1>0, \quad b-1>0 \]
% よって, $(a-1)(b-1) > 0$ である.

% したがって, $(\text{左辺}) > (\text{右辺})$ より
% \[ ab + 1 > a + b \quad \qed \]
\end{proofbox}

\begin{any}{Check}
式変形の途中で「因数分解できそう」という感覚を持つことが大切.
\begin{itemize}
    \item $xy - x - y + 1 = (x-1)(y-1)$
    \item $a^2 - a - b + ab = a(a-1) + b(a-1) = (a+b)(a-1)$
\end{itemize}
などの形に慣れておこう.
\end{any}

\end{multicols*}

\newpage

%===========================================================
% 裏面: 演習問題
%===========================================================

\begin{multicols*}{2}

{\large \textbf{確認テスト}}

\begin{prac}{A1(基本証明)}
$a>b$ のとき, 次の不等式を証明せよ.
\[ 3a - 4b > 2a - 3b \]
\end{prac}

\begin{proofbox}
\vspace{4cm}
\end{proofbox}


\begin{prac}{A2(因数分解の利用)}
$x>3, \ y>3$ のとき, 次の不等式を証明せよ.
\[ xy + 9 > 3x + 3y \]
\end{prac}

\begin{proofbox}
\vspace{6cm}
\end{proofbox}

\columnbreak

\begin{prac}{B1(大小比較)}
$a>0, \ b>0$ のとき, $\ds \frac{a+b}{2}$ と $\ds \frac{2ab}{a+b}$ の大小を比較せよ.
(ヒント: 引き算をして, 通分してみよう)
\end{prac}

\begin{proofbox}

\end{proofbox}

\begin{any}{振り返り}
不等式の証明は, 「計算」ではなく「説明」である.
\begin{itemize}
    \item 「なぜプラスになるのか」の理由を書くこと.
    \item ($a>2$ より $a-2>0$ など)
\end{itemize}
ここを省略せずに書く癖をつけよう.
\end{any}

\end{multicols*}

\end{document}