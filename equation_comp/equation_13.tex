\documentclass[b4paper, landscape, dvipdfmx]{jsarticle}
%----- 必要なパッケージ -----
\usepackage{fancybox,ascmac,otf,ulem}
\usepackage{amssymb, amsthm}
\usepackage[leqno]{amsmath}
\usepackage{geometry}
\usepackage{multicol}
\usepackage{tcolorbox}
\usepackage{xcolor}
\usepackage{fancyhdr}
\usepackage{tikz}

% shadowsライブラリ
\usetikzlibrary{
    positioning,
    arrows.meta,
    calc,
    shadows,
    shadows.blur,
    intersections
}

\tcbuselibrary{skins, breakable, theorems}
\usepackage{enumitem}
\setlist[enumerate,1]{label=(\arabic*)}
\setlist[itemize]{leftmargin=*}
\newcommand{\ds}{\displaystyle}

%----- レイアウト設定 -----
\geometry{
  left=15mm,
  right=15mm,
  top=20mm,
  bottom=15mm,
  headheight=25pt
}

%----- 数式環境の上下の余白調整 -----
\AtBeginDocument{
  \setlength{\abovedisplayskip}{5pt}
  \setlength{\belowdisplayskip}{5pt}
  \setlength{\abovedisplayshortskip}{0pt}
  \setlength{\belowdisplayshortskip}{3pt}
}

%===========================================================
%  デザイン設定
%===========================================================

%--- 色の定義 ---
\definecolor{printBlue}{RGB}{0, 50, 100}     % 濃紺
\definecolor{printRed}{RGB}{140, 20, 20}     % 濃エンジ
\definecolor{printTeal}{RGB}{0, 60, 60}      % 濃い青緑

%--- 共通スタイル定義 ---
\tcbset{
    chartbox/.style={
        enhanced,
        fonttitle=\sffamily\bfseries,
        boxrule=1pt,
        arc=2pt,
        top=1.0em,
        nobeforeafter,
        enlarge left by=-2mm,
        enlarge right by=-2mm,
        drop fuzzy shadow,
        colback=white,
        attach boxed title to top left={xshift=10pt, yshift*=-\tcboxedtitleheight/2},
        boxed title style={frame hidden, sharp corners, rounded corners=southeast, arc=3pt}
    }
}

% 各種ボックス環境定義
\newtcolorbox{any}[1]{
    enlarge left by=0mm, enlarge right by=0mm,
    enhanced, frame hidden, colback=white, title={#1},
    attach boxed title to top left={xshift=0mm, yshift=0mm},
    coltitle=white, fonttitle=\bfseries\sffamily,
    boxed title style={
        colback=black!80, frame hidden, arc=4pt, outer arc=4pt,
        sharp corners=south, boxrule=0pt,
        top=1mm, bottom=1mm, left=3mm, right=3mm
    },
    underlay boxed title={
        \draw[thick, black!80] (title.south west) -- (title.south west-|frame.east);
    },
    breakable, top=5mm, left=2mm, right=2mm, bottom=0mm,
    before skip=1em, after skip=1em,
    segmentation style={draw=black!40, dashed}
}


\newenvironment{eg}[1]{
\begin{tcolorbox}[
    chartbox,
    colframe=printBlue,
    coltitle=white,
    title=\textbf{例題 #1},
    boxed title style={colback=printBlue},
    segmentation style={draw=printBlue, line width=0.5pt, dashed}
]}
{\end{tcolorbox}}

\newenvironment{prac}[1]{
\begin{tcolorbox}[
    chartbox,
    colframe=printRed,
    coltitle=white,
    title=\textbf{練習 #1},
    boxed title style={colback=printRed}
]}
{\end{tcolorbox}}

\newenvironment{answer}[1][height fill]{
    \begin{tcolorbox}[
        enhanced,
        title={Memo / Answer},
        colframe=black!80,
        colback=white,
        coltitle=black!60,
        fonttitle=\sffamily\bfseries,
        attach boxed title to top left={xshift=5mm, yshift*=-\tcboxedtitleheight/2},
        boxed title style={frame hidden, colback=white},
        boxrule=1pt,
        arc=1pt,
        nobeforeafter,
        enlarge left by=2mm, 
        enlarge right by=2mm, 
        height fill,
        segmentation style={draw=black!20, solid},
        underlay={
            \begin{tcbclipinterior}
                \draw[step=5mm, black!5, ultra thin] (interior.south west) grid (interior.north east);
            \end{tcbclipinterior}
        }, 
        #1
    ]}
{ \end{tcolorbox}}

\newenvironment{proofbox}[1][height fill]{
    \begin{tcolorbox}[
        enhanced,
        title={Proof},
        colframe=black!80,
        colback=white,
        coltitle=black!60,
        fonttitle=\sffamily\bfseries,
        attach boxed title to top left={xshift=5mm, yshift*=-\tcboxedtitleheight/2},
        boxed title style={frame hidden, colback=white},
        boxrule=1pt,
        arc=1pt,
        nobeforeafter,
        width=\linewidth,
        underlay={
            \begin{tcbclipinterior}
                \draw[step=5mm, black!5, ultra thin] (interior.south west) grid (interior.north east);
            \end{tcbclipinterior}
        },
        #1
    ]}
{ \end{tcolorbox}}

%----- 段組の設定 -----
\setlength{\columnsep}{15mm}
\setlength{\columnseprule}{0.4pt}
\renewcommand{\columnseprulecolor}{\color{black!30}}

%----- ヘッダーの設定 -----
\pagestyle{fancy}
\fancyhf{}

% ヘッダーデザイン
\fancyhead[C]{%
    \begin{tikzpicture}[remember picture, overlay]
        \node[anchor=north west, fill=printBlue, minimum width=\paperwidth, minimum height=5pt] at (current page.north west) {};
    \end{tikzpicture}
}
\fancyhead[L]{\small \textcolor{black!90}{数学\ajRoman{2} $>$ 第3章 式と証明 $>$ 第13回--\textbf{不等式の証明(2)}}}
\fancyhead[R]{\small 年 \hspace{1cm} 組 \hspace{1cm} 番 \quad 氏名 \hspace{6cm}}
\renewcommand{\headrulewidth}{0pt}

\begin{document}

\begin{multicols*}{2}

%===========================================================
% 左カラム: 平方の非負性
%===========================================================

{\large \textbf{1. 実数条件最大の武器「2乗」}}

\begin{any}{絶対にマイナスにならない数}
実数 $x$ について, 最も重要な性質の一つがこれである.
\begin{center}
\begin{tcolorbox}[colframe=printRed, colback=white, boxrule=1.5pt, arc=0pt]
    \centering
    $\ds \boldsymbol{x^2 \geqq 0}$ \quad (等号成立は $x=0$ のとき)
\end{tcolorbox}
\end{center}
また, 2乗の和についても同様である.
\[ \boldsymbol{A^2 + B^2 \geqq 0} \]
これを利用すると, 「平方完成」することで不等式を証明できる.
\end{any}

\begin{eg}{1(基本の平方完成)}
不等式 $x^2 - 4x + 6 > 0$ を証明せよ.
\end{eg}

\begin{proofbox}
% 平方完成を行う.
% \begin{align*}
% x^2 - 4x + 6 &= (x-2)^2 - 4 + 6 \\
% &= (x-2)^2 + 2
% \end{align*}
% ここで, $x$ は実数であるから, $\boldsymbol{(x-2)^2 \geqq 0}$. \\
% よって, $\boldsymbol{(x-2)^2 + 2 > 0}$ (0以上の数に2を足すと必ず正). \\
% したがって, $x^2 - 4x + 6 > 0$ である. \qed
\end{proofbox}

\begin{any}{「等号成立」とは?}
不等式 $A \geqq B$ において, $A=B$ となる瞬間のことをいう.
例えば $x^2 \geqq 0$ なら, $x=0$ のとき等号が成立する.

では,そもそもなぜ「等号成立」を考えなければいけないのか??
\end{any}

\begin{any}{最年少は何歳??}
     全員の中で最年少と最年長が何歳なのかを知りたい.
     \begin{itemize}
         \item 「全員15歳以上($x\geqq16$)」なら最年少は15歳であると【~~言える~~/~~言えない~~】.
         \item 「全員16歳未満($x<16$)」なら最年長は15歳であると【~~言える~~/~~言えない~~】.
         \item 「全員15歳以上で,かつ,15歳の人が少なくとも一人いる」なら最年少は15歳であると【~~言える~~/~~言えない~~】.
     \end{itemize}
     $\to$最小値や最大値の問題では,\uwave{等号を成り立たせるような$x$の存在}が極めて重要なのである.
\end{any}

\columnbreak

%===========================================================
% 右カラム: 文字が2つの場合
%===========================================================

{\large \textbf{2. 2変数の不等式}}

\begin{any}{証明の流れ}
\begin{enumerate}
    \item $(\text{左辺}) - (\text{右辺})$ を計算する.
    \item \textbf{平方完成}して $(\quad)^2 + (\quad)^2$ などの形を作る.
    \item 実数の2乗は0以上であることを述べる.
    \item 等号成立条件 ($=0$ になる時) を確認する.
\end{enumerate}
\end{any}

\begin{eg}{2(2変数の不等式)}
不等式 $x^2 + y^2 \geqq 2xy$ を証明せよ. また, 等号が成立するのはどのようなときか答えよ.
\end{eg}

\begin{proofbox}
$(\text{左辺}) - (\text{右辺}) = x^2 + y^2 - 2xy$ \\
$\phantom{(\text{左辺}) - (\text{右辺})} = x^2 - 2xy + y^2$ \\
$\phantom{(\text{左辺}) - (\text{右辺})} = (x-y)^2$

ここで, $x, y$ は実数であるから, $(x-y)^2 \geqq 0$.
よって, $(\text{左辺}) - (\text{右辺}) \geqq 0$ より
\[ x^2 + y^2 \geqq 2xy \]
等号が成立するのは, $x-y=0$, すなわち $\boldsymbol{x=y}$ のときである. \qed
\end{proofbox}

\begin{any}{なぜ実数条件が必要?}
もし虚数でもよいなら, $x=i$ のとき $x^2 = -1 < 0$ となり, この証明は崩壊する.
「2乗して0以上」は, 実数だけの特権なのである.
\end{any}

\end{multicols*}

\newpage

%===========================================================
% 裏面: 演習問題
%===========================================================

\begin{multicols*}{2}

{\large \textbf{確認テスト}}

\begin{prac}{A1(常に正であることの証明)}
不等式 $x^2 + 6x + 10 > 0$ を証明せよ.
\end{prac}

\begin{proofbox}
\vspace{4cm}
\end{proofbox}


\begin{prac}{A2(2変数の不等式)}
不等式 $2(x^2+y^2) \geqq (x+y)^2$ を証明せよ. また, 等号成立条件を求めよ.
\end{prac}

\begin{proofbox}
\vspace{6cm}
\end{proofbox}

\columnbreak

\begin{prac}{B1(応用:平方の和)}
不等式 $a^2 + b^2 \geqq ab$ を証明せよ.
また, 等号成立条件を求めよ.
\end{prac}

\begin{any}{Hint}
$a$ について整理して平方完成する.
$a^2 - ba + b^2 = \left(a - \frac{b}{2}\right)^2 - \frac{b^2}{4} + b^2 = \cdots$
\end{any}

\begin{proofbox}
\vspace{6cm}
\end{proofbox}

\begin{prac}{B2(絶対不等式)}
不等式 $x^2 + y^2 + z^2 \geqq xy + yz + zx$ を証明せよ.
(ヒント: 全体を2倍して $2x^2+2y^2+2z^2-2xy-2yz-2zx$ を作り, $(x-y)^2$ などの和に変形する)
\end{prac}

\begin{proofbox}
\end{proofbox}

\end{multicols*}

\end{document}