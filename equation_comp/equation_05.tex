\documentclass[b4paper, landscape, dvipdfmx]{jsarticle}
%----- 必要なパッケージ -----
\usepackage{fancybox,ascmac,otf,ulem}
\usepackage{amssymb, amsthm}
\usepackage[leqno]{amsmath}
\usepackage{wrapfig}
\usepackage{geometry}
\usepackage{multicol}
\usepackage{tcolorbox}
\usepackage{xcolor}
\usepackage{fancyhdr}
\usepackage{tikz}

% shadowsライブラリ
\usetikzlibrary{
    positioning,
    arrows.meta,
    calc,
    shadows,
    shadows.blur,
    intersections,
    patterns
}

\tcbuselibrary{skins, breakable, theorems}
\usepackage{enumitem}
\setlist[enumerate,1]{label=(\arabic*)}
\setlist[itemize]{leftmargin=*}
\newcommand{\ds}{\displaystyle}

%----- レイアウト設定 -----
\geometry{
  left=15mm,
  right=15mm,
  top=20mm,
  bottom=15mm,
  headheight=25pt
}

%----- 数式環境の上下の余白調整 -----
\AtBeginDocument{
  \setlength{\abovedisplayskip}{5pt}
  \setlength{\belowdisplayskip}{5pt}
  \setlength{\abovedisplayshortskip}{0pt}
  \setlength{\belowdisplayshortskip}{3pt}
}

%===========================================================
%  デザイン設定・共通定義
%===========================================================

\definecolor{chartBlue}{RGB}{0, 65, 120}    % 青
\definecolor{chartRed}{RGB}{160, 20, 20}    % 赤
\definecolor{chartOrange}{RGB}{200, 100, 0} % オレンジ
\definecolor{chartTeal}{RGB}{0, 100, 100}   % ティール

\tcbset{
    chartbox/.style={
        enhanced, fonttitle=\sffamily\bfseries, boxrule=1pt, arc=3pt,
        top=1.2em, nobeforeafter,
        enlarge left by=-2mm, enlarge right by=-2mm,
        drop fuzzy shadow,
        attach boxed title to top left={xshift=10pt, yshift*=-\tcboxedtitleheight/2},
        boxed title style={frame hidden, sharp corners, rounded corners=southeast, arc=4pt}
    }
}

% 解説用ボックス (黒/白) - 親コンテナとして使用
\newtcolorbox{any}[1]{
    enlarge left by=0mm, enlarge right by=0mm,
    enhanced, frame hidden, colback=white, title={#1},
    attach boxed title to top left={xshift=0mm, yshift=0mm},
    coltitle=white, fonttitle=\bfseries\sffamily,
    boxed title style={
        colback=black!80, frame hidden, arc=4pt, outer arc=4pt,
        sharp corners=south, boxrule=0pt,
        top=1mm, bottom=1mm, left=3mm, right=3mm
    },
    underlay boxed title={
        \draw[thick, black!80] (title.south west) -- (title.south west-|frame.east);
    },
    breakable, top=5mm, left=2mm, right=2mm, bottom=0mm,
    before skip=1em, after skip=1em,
    segmentation style={draw=black!40, dashed}
}

% 定理 (thm) - ティール/重要
\newenvironment{thm}[1]{
\begin{tcolorbox}[
    chartbox, colframe=chartTeal, colback=white, coltitle=white,
    title=\textbf{#1}, boxed title style={colback=chartTeal},
    skin=bicolor, colbacklower=white,
    segmentation style={draw=chartTeal, dashed, line width=1pt}
]}{\end{tcolorbox}}

% 例題 (eg) - 青
\newenvironment{eg}[1]{
\begin{tcolorbox}[
    chartbox, colframe=chartBlue, colback=white, coltitle=white,
    title=\textbf{例題 #1}, boxed title style={colback=chartBlue},
    skin=bicolor, colbacklower=white!95!blue,
    segmentation style={draw=chartBlue, line width=0pt}
]}{\end{tcolorbox}}

% 練習 (prac) - 赤
\newenvironment{prac}[1]{
\begin{tcolorbox}[
    chartbox, colframe=chartRed, colback=white, coltitle=white,
    title=\textbf{練習 #1}, boxed title style={colback=chartRed}
]}{\end{tcolorbox}}

% 解答欄 (answer)
\newenvironment{answer}[1][height fill]{
    \noindent
    \begin{tcolorbox}[
        enhanced, title={Memo / Answer},
        colframe=black!80, colback=white, coltitle=black!80,
        fonttitle=\sffamily\bfseries,
        attach boxed title to top left={xshift=5mm, yshift*=-\tcboxedtitleheight/2},
        boxed title style={frame hidden, colback=white},
        boxrule=1pt, arc=1pt, nobeforeafter,
        enlarge left by=-2mm, enlarge right by=-2mm,
        width=\linewidth, #1,
        underlay={ \begin{tcbclipinterior} \draw[step=5mm, black!5, ultra thin] (interior.south west) grid (interior.north east); \end{tcbclipinterior} }
    ]}{\end{tcolorbox}}

%----- 段組の設定 -----
\setlength{\columnsep}{15mm}
\setlength{\columnseprule}{0.4pt}
\renewcommand{\columnseprulecolor}{\color{black!30}}

%----- ヘッダーの設定 -----
\pagestyle{fancy}
\fancyhf{}

% ヘッダーデザイン
\fancyhead[C]{%
    \begin{tikzpicture}[remember picture, overlay]
        \node[anchor=north west, fill=chartBlue, minimum width=\paperwidth, minimum height=5pt] at (current page.north west) {};
    \end{tikzpicture}
}
\fancyhead[L]{\small \textcolor{black!90}{数学\ajRoman{2} $>$ 第2章 複素数と方程式 $>$ 第5回--\textbf{解と係数の関係}}}
\fancyhead[R]{\small 年 \hspace{1cm} 組 \hspace{1cm} 番 \quad 氏名 \hspace{6cm}}
\renewcommand{\headrulewidth}{0pt}

\begin{document}

%===========================================================
% 表面:授業用スライド・板書用
%===========================================================
\begin{multicols*}{2}

%--- 左カラム ---

\begin{any}{1. 解と係数の関係}
2次方程式 $ax^2+bx+c=0$ の2つの解を $\alpha, \beta$ とするとき, 実際に解を求めなくても, \textbf{解の和 $\boldsymbol{\alpha+\beta}$} と \textbf{解の積 $\boldsymbol{\alpha\beta}$} は係数 $a, b, c$ だけで計算できる.

\begin{thm}{解と係数の関係}
2次方程式 $ax^2+bx+c=0$ の2つの解を $\alpha, \beta$ とすると,
\[ \boldsymbol{\alpha+\beta = -\frac{b}{a}}, \quad \boldsymbol{\alpha\beta = \frac{c}{a}} \]
\end{thm}

\begin{eg}{1(基本計算)}
2次方程式 $2x^2 - 6x + 5 = 0$ の2つの解を $\alpha, \beta$ とするとき, 次の式の値を求めよ.
\[ (1)~\alpha+\beta \quad\quad (2)~\alpha\beta \]
\end{eg}

\begin{answer}[height=10cm]
% a=2, b=-6, c=5
\end{answer}

\textbf{<証明の概略>}\\
解が $\alpha, \beta$ である方程式は $a(x-\alpha)(x-\beta)=0$ と表せる.
展開すると $ax^2 - a(\alpha+\beta)x + a\alpha\beta = 0$ となり,
もとの式 $ax^2+bx+c=0$ と係数を比較して導かれる.
\end{any}

\columnbreak

%--- 右カラム ---

\begin{any}{2. 対称式の値}
文字 $\alpha, \beta$ を入れ替えても値が変わらない式を\textbf{対称式}という.
すべての対称式は, 基本対称式 \textbf{$\boldsymbol{\alpha+\beta}$ と $\boldsymbol{\alpha\beta}$ のみで表すことができる}.

\begin{thm}{よく使う変形公式}
\begin{itemize}
    \item $\boldsymbol{\alpha^2+\beta^2 = (\alpha+\beta)^2 - 2\alpha\beta}$
    \item $\ds \boldsymbol{\frac{1}{\alpha}+\frac{1}{\beta} = \frac{\alpha+\beta}{\alpha\beta}}$ \quad (通分する)
\end{itemize}
\end{thm}

\begin{eg}{2(対称式の利用)}
2次方程式 $x^2 + 3x + 4 = 0$ の2つの解を $\alpha, \beta$ とするとき, 次の式の値を求めよ.
\begin{enumerate}
    \item $\alpha^2 + \beta^2$
    \item $\ds \frac{1}{\alpha} + \frac{1}{\beta}$
\end{enumerate}
\end{eg}

\begin{answer}[height=10cm]
\end{answer}
\end{any}

\columnbreak

\begin{any}{3. 2数を解とする方程式を作る}
\textbf{和と積さえ分かれば方程式は復元できる!}
2数 $\alpha, \beta$ を解にもつ2次方程式の一つは $x^2 - (\alpha+\beta)x + \alpha\beta = 0$ である.

\begin{eg}{3(方程式の作成)}
次の2数を解とする2次方程式を一つ作れ.
\begin{enumerate}
    \item $3, -5$
    \item $1+2i, \quad 1-2i$
\end{enumerate}
\end{eg}

\begin{answer}[height=12cm]
\end{answer}
\end{any}

\end{multicols*}

\newpage

%===========================================================
% 裏面:確認テスト(問題編)
%===========================================================
\fancyhead[L]{\small \textcolor{black!90}{数学\ajRoman{2} $>$ 確認テスト}}

\begin{multicols*}{2}

\begin{any}{確認テスト (基本)}
\begin{prac}{A1(基本計算)}
次の2次方程式の2つの解 $\alpha, \beta$ の和と積を求めよ.
\begin{enumerate}
    \item $x^2 - 4x - 2 = 0$
    \item $3x^2 + 2x + 1 = 0$
\end{enumerate}
\end{prac}

\begin{prac}{A2(対称式の値 1)}
2次方程式 $x^2 - 5x + 3 = 0$ の2つの解を $\alpha, \beta$ とするとき, 次の式の値を求めよ.
\begin{enumerate}
    \item $\alpha^2 + \beta^2$
    \item $\ds \frac{1}{\alpha} + \frac{1}{\beta}$
    \item $(\alpha-1)(\beta-1)$
\end{enumerate}
\end{prac}

\begin{answer}[height=12cm]
\end{answer}
\end{any}

\columnbreak

\begin{any}{確認テスト (応用)}
\begin{prac}{B1(対称式の値 2:3乗の和)}
2次方程式 $x^2 + 2x + 4 = 0$ の2つの解を $\alpha, \beta$ とするとき, 次の式の値を求めよ.
\[ \alpha^3 + \beta^3 \]
(Hint: $\alpha^3+\beta^3 = (\alpha+\beta)(\alpha^2-\alpha\beta+\beta^2)$ または $(\alpha+\beta)^3 - 3\alpha\beta(\alpha+\beta)$)
\end{prac}

\begin{prac}{B2(少し複雑な対称式)}
2次方程式 $2x^2 - 4x + 1 = 0$ の2つの解を $\alpha, \beta$ とするとき, 次の式の値を求めよ.
\[ \frac{\beta}{\alpha} + \frac{\alpha}{\beta} \]
\end{prac}

\begin{answer}[height=12cm]
\end{answer}
\end{any}

\end{multicols*}

\newpage

%===========================================================
% 別紙:確認テスト(解答編)
%===========================================================
\fancyhead[L]{\small \textcolor{black!90}{数学\ajRoman{2} $>$ 確認テスト \textbf{[解答・解説]}}}

\begin{multicols*}{2}

\begin{any}{確認テスト (基本) [解答]}
\begin{prac}{A1(基本計算)}
次の2次方程式の2つの解 $\alpha, \beta$ の和と積を求めよ.
\begin{enumerate}
    \item $x^2 - 4x - 2 = 0$
    \item $3x^2 + 2x + 1 = 0$
\end{enumerate}
\end{prac}

\begin{prac}{A2(対称式の値 1)}
2次方程式 $x^2 - 5x + 3 = 0$ の2つの解を $\alpha, \beta$ とするとき, 次の式の値を求めよ.
\begin{enumerate}
    \item $\alpha^2 + \beta^2$
    \item $\ds \frac{1}{\alpha} + \frac{1}{\beta}$
    \item $(\alpha-1)(\beta-1)$
\end{enumerate}
\end{prac}

\begin{answer}[height fill]
\color{chartRed}
\textbf{A1} \\
解と係数の関係 $\alpha+\beta = -b/a, \ \alpha\beta = c/a$ より
\begin{enumerate}
    \item $\ds \alpha+\beta = -\frac{-4}{1} = \boldsymbol{4}, \quad \alpha\beta = \boldsymbol{-2}$
    \item $\ds \alpha+\beta = -\frac{2}{3} = \boldsymbol{-\frac{2}{3}}, \quad \alpha\beta = \boldsymbol{\frac{1}{3}}$
\end{enumerate}

\vspace{0.5em}
\hrule
\vspace{0.5em}

\textbf{A2} \\
$\alpha+\beta = 5, \quad \alpha\beta = 3$ である.
\begin{enumerate}
    \item $\alpha^2+\beta^2 = (\alpha+\beta)^2 - 2\alpha\beta = 5^2 - 2\cdot 3 = 25-6 = \boldsymbol{19}$
    \item $\ds \frac{1}{\alpha}+\frac{1}{\beta} = \frac{\alpha+\beta}{\alpha\beta} = \boldsymbol{\frac{5}{3}}$
    \item $(\alpha-1)(\beta-1) = \alpha\beta - (\alpha+\beta) + 1 = 3 - 5 + 1 = \boldsymbol{-1}$
\end{enumerate}
\end{answer}
\end{any}

\columnbreak

\begin{any}{確認テスト (応用) [解答]}
\begin{prac}{B1(対称式の値 2:3乗の和)}
2次方程式 $x^2 + 2x + 4 = 0$ の2つの解を $\alpha, \beta$ とするとき, 次の式の値を求めよ.
\[ \alpha^3 + \beta^3 \]
\end{prac}

\begin{prac}{B2(少し複雑な対称式)}
2次方程式 $2x^2 - 4x + 1 = 0$ の2つの解を $\alpha, \beta$ とするとき, 次の式の値を求めよ.
\[ \frac{\beta}{\alpha} + \frac{\alpha}{\beta} \]
\end{prac}

\begin{answer}[height fill]
\color{chartRed}
\textbf{B1} \\
$\alpha+\beta = -2, \quad \alpha\beta = 4$ である. \\
3乗の展開公式変形を利用すると,
\begin{align*}
\alpha^3+\beta^3 &= (\alpha+\beta)^3 - 3\alpha\beta(\alpha+\beta) \\
&= (-2)^3 - 3 \cdot 4 \cdot (-2) \\
&= -8 + 24 = \boldsymbol{16}
\end{align*}
(別解) $\alpha^3+\beta^3 = (\alpha+\beta)(\alpha^2-\alpha\beta+\beta^2)$ を用いてもよい.

\vspace{0.5em}
\hrule
\vspace{0.5em}

\textbf{B2} \\
$\ds \alpha+\beta = -\frac{-4}{2} = 2, \quad \alpha\beta = \frac{1}{2}$ である. \\
与式を通分すると,
\begin{align*}
\frac{\beta}{\alpha} + \frac{\alpha}{\beta} &= \frac{\beta^2+\alpha^2}{\alpha\beta} = \frac{(\alpha+\beta)^2 - 2\alpha\beta}{\alpha\beta} \\
&= \frac{2^2 - 2 \cdot \frac{1}{2}}{\frac{1}{2}} = \frac{4-1}{\frac{1}{2}} = \frac{3}{\frac{1}{2}} = \boldsymbol{6}
\end{align*}
\end{answer}
\end{any}

\end{multicols*}

\end{document}