\documentclass[b4paper, landscape, dvipdfmx]{jsarticle}
%----- 必要なパッケージ -----
\usepackage{fancybox,ascmac,otf,ulem}
\usepackage{amssymb, amsthm}
\usepackage[leqno]{amsmath}
\usepackage{geometry}
\usepackage{multicol}
\usepackage{tcolorbox}
\usepackage{xcolor}
\usepackage{fancyhdr}
\usepackage{tikz}

% shadowsライブラリ
\usetikzlibrary{
    positioning,
    arrows.meta,
    calc,
    shadows,
    shadows.blur,
    intersections
}

\tcbuselibrary{skins, breakable, theorems}
\usepackage{enumitem}
\setlist[enumerate,1]{label=(\arabic*)}
\setlist[itemize]{leftmargin=*}
\newcommand{\ds}{\displaystyle}

%----- レイアウト設定 -----
\geometry{
  left=15mm,
  right=15mm,
  top=20mm,
  bottom=15mm,
  headheight=25pt
}

%----- 数式環境の上下の余白調整 -----
\AtBeginDocument{
  \setlength{\abovedisplayskip}{5pt}
  \setlength{\belowdisplayskip}{5pt}
  \setlength{\abovedisplayshortskip}{0pt}
  \setlength{\belowdisplayshortskip}{3pt}
}

%===========================================================
%  デザイン設定
%===========================================================

%--- 色の定義 ---
\definecolor{printBlue}{RGB}{0, 50, 100}     % 濃紺
\definecolor{printRed}{RGB}{140, 20, 20}     % 濃エンジ
\definecolor{printTeal}{RGB}{0, 60, 60}      % 濃い青緑

%--- ボックススタイル定義 ---
\tcbset{
    testbox/.style={
        enhanced,
        fonttitle=\sffamily\bfseries,
        colframe=printBlue,
        colback=white,
        coltitle=white,
        boxrule=1pt,
        arc=2pt,
        top=1.0em,
        nobeforeafter,
        drop fuzzy shadow,
        attach boxed title to top left={xshift=5pt, yshift*=-\tcboxedtitleheight/2},
        boxed title style={frame hidden, sharp corners, rounded corners=southeast, arc=3pt}
    }
}

% 問題用ボックス
\newenvironment{prob}[1]{
\begin{tcolorbox}[
    testbox,
    title=\textbf{第#1問},
]}
{\end{tcolorbox}}

% 解説用ボックス
\newenvironment{explain}[1]{
    \begin{tcolorbox}[
        enhanced,
        colframe=printRed,
        colback=white,
        coltitle=white,
        title=\textbf{【解説】第#1問},
        fonttitle=\sffamily\bfseries,
        boxrule=1pt,
        arc=2pt,
        segmentation style={draw=printRed, dashed}
    ]}
{ \end{tcolorbox}}

%-----------------------------------------------------------
% answer環境 (前回までの定義と全く同じもの)
%-----------------------------------------------------------
\newenvironment{answer}[1][height fill]{
    \begin{tcolorbox}[
        enhanced,
        title={Memo / Answer},
        colframe=black!80,
        colback=white,
        coltitle=black!60,
        fonttitle=\sffamily\bfseries,
        attach boxed title to top left={xshift=5mm, yshift*=-\tcboxedtitleheight/2},
        boxed title style={frame hidden, colback=white},
        boxrule=1pt,
        arc=1pt,
        nobeforeafter,
        enlarge left by=2mm, 
        enlarge right by=2mm, 
        height fill,
        segmentation style={draw=black!20, solid},
        underlay={% 方眼を描画
            \begin{tcbclipinterior}
                \draw[step=5mm, black!5, ultra thin] (interior.south west) grid (interior.north east);
            \end{tcbclipinterior}
        }, 
        #1
    ]}
{ \end{tcolorbox}}


%----- 段組の設定 -----
\setlength{\columnsep}{15mm}
\setlength{\columnseprule}{0.4pt}
\renewcommand{\columnseprulecolor}{\color{black!30}}

%----- ヘッダーの設定 -----
\pagestyle{fancy}
\fancyhf{}

% ヘッダーデザイン
\fancyhead[C]{%
    \begin{tikzpicture}[remember picture, overlay]
        \node[anchor=north west, fill=printBlue, minimum width=\paperwidth, minimum height=5pt] at (current page.north west) {};
    \end{tikzpicture}
}
\fancyhead[L]{\small \textcolor{black!90}{数学\ajRoman{2} $>$ 第2章 複素数と方程式 $>$ 第10回--\textbf{到達度確認テスト}}}
\fancyhead[R]{\small 実施日:\hspace{2cm} \quad /50点 \quad 氏名 \hspace{5cm}}
\renewcommand{\headrulewidth}{0pt}

\begin{document}

%===========================================================
% 1枚目:問題(左)と解答欄(右)
%===========================================================
\begin{multicols*}{2}

%-----------------------------------------------------------
% 左カラム:問題
%-----------------------------------------------------------
{\large \textbf{【問題】 制限時間 30分}}

\begin{prob}{1 (複素数の計算・10点)}
次の計算をし,$a+bi$ の形で答えよ.
\begin{enumerate}
    \item $(3-2i)^2$
    \item $\ds \frac{5}{1-2i}$
\end{enumerate}
\end{prob}

\begin{answer}
    
\end{answer}

\begin{prob}{2 (2次方程式の解・10点)}
2次方程式 $x^2 - 3x + 4 = 0$ の2つの解を $\alpha, \beta$ とするとき,次の式の値を求めよ.
\begin{enumerate}
    \item $\alpha^2 + \beta^2$
    \item $\ds \frac{1}{\alpha} + \frac{1}{\beta}$
\end{enumerate}
\end{prob}

\begin{answer}
    
\end{answer}

\begin{prob}{3 (因数定理と高次方程式・15点)}
3次方程式 $x^3 - 4x^2 + ax + b = 0$ が $x=1$ と $x=2$ を解にもつとき,以下の問いに答えよ.
\begin{enumerate}
    \item 定数 $a, b$ の値を求めよ.
    \item 残りの解を求めよ.
\end{enumerate}
\end{prob}

\begin{answer}
    
\end{answer}

\begin{prob}{4 (高次方程式とオメガ・15点)}
以下の問いに答えよ.
\begin{enumerate}
    \item 4次方程式 $x^4 - 3x^2 - 4 = 0$ を解け.
    \item 方程式 $x^3=1$ の虚数解の一つを $\omega$ とするとき,
    $\omega^{5} + \omega^4 + 1$ の値を求めよ.
\end{enumerate}
\end{prob}


\begin{answer}

\end{answer}



\end{multicols*}

\newpage

%===========================================================
% 2枚目:解答と解説
%===========================================================
\begin{multicols*}{2}

{\large \textbf{【解答・解説】}}

\begin{tcolorbox}[colframe=printTeal, title=\textbf{学習チェックリスト}]
\small
間違えた問題に関連する項目をチェックしよう.
\begin{itemize}
    \item[$\square$] 複素数の四則演算(特に分母の実数化)
    \item[$\square$] 解と係数の関係と対称式の変形
    \item[$\square$] 剰余の定理・因数定理の使い方
    \item[$\square$] 高次方程式の解法(因数分解・複二次式)
    \item[$\square$] 1の3乗根 $\omega$ の性質 ($\omega^3=1, \omega^2+\omega+1=0$)
\end{itemize}
\end{tcolorbox}

\begin{explain}{1}
(1) 展開公式 $(A-B)^2=A^2-2AB+B^2$ を利用.
\begin{align*}
(3-2i)^2 &= 9 - 12i + 4i^2 \\
&= 9 - 12i - 4 \\
&= \boldsymbol{5 - 12i}
\end{align*}

(2) 分母の共役な複素数 $1+2i$ を掛ける.
\begin{align*}
\frac{5}{1-2i} &= \frac{5(1+2i)}{(1-2i)(1+2i)} \\
&= \frac{5(1+2i)}{1^2 + 2^2} \\
&= \frac{5(1+2i)}{5} = \boldsymbol{1 + 2i}
\end{align*}
\end{explain}

\begin{explain}{2}
解と係数の関係より,$\alpha+\beta = 3, \quad \alpha\beta = 4$.

(1)
\begin{align*}
\alpha^2+\beta^2 &= (\alpha+\beta)^2 - 2\alpha\beta \\
&= 3^2 - 2 \cdot 4 \\
&= 9 - 8 = \boldsymbol{1}
\end{align*}

(2)
\begin{align*}
\frac{1}{\alpha} + \frac{1}{\beta} &= \frac{\alpha+\beta}{\alpha\beta} \\
&= \boldsymbol{\frac{3}{4}}
\end{align*}
\end{explain}

\columnbreak

\begin{explain}{3}
(1) $x=1, 2$ を解にもつので,代入して成り立つ.
\[
\begin{cases}
1 - 4 + a + b = 0 \\
8 - 16 + 2a + b = 0
\end{cases}
\iff
\begin{cases}
a + b = 3 \\
2a + b = 8
\end{cases}
\]
これを解いて,$\boldsymbol{a=5, b=-2}$

(2) 方程式は $x^3 - 4x^2 + 5x - 2 = 0$.
$(x-1)(x-2) = x^2-3x+2$ で割り切れるので,
筆算などで割り算を行うと,
$(x-1)(x-2)(x-1) = 0$ となる.
よって残りの解は $\boldsymbol{x=1}$ (重解).
\end{explain}

\begin{explain}{4}
(1) $x^2=X$ とおくと $X^2 - 3X - 4 = 0$.
$(X-4)(X+1) = 0$ より $X=4, -1$.
よって $x^2=4, \ x^2=-1$.
$\boldsymbol{x = \pm 2, \ \pm i}$

(2) $\omega$ の性質を利用する.
\begin{itemize}
    \item $\omega^5 = \omega^3 \cdot \omega^2 = 1 \cdot \omega^2 = \omega^2$
    \item $\omega^4 = \omega^3 \cdot \omega = 1 \cdot \omega = \omega$
\end{itemize}
よって
\[ \omega^5 + \omega^4 + 1 = \omega^2 + \omega + 1 = \boldsymbol{0} \]
\end{explain}

\end{multicols*}

\end{document}