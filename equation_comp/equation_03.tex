\documentclass[b4paper, landscape, dvipdfmx]{jsarticle}
%----- 必要なパッケージ -----
\usepackage{fancybox,ascmac,otf,ulem}
\usepackage{amssymb, amsthm}
\usepackage[leqno]{amsmath}
\usepackage{wrapfig}
\usepackage{geometry}
\usepackage{multicol}
\usepackage{tcolorbox}
\usepackage{xcolor}
\usepackage{fancyhdr}
\usepackage{tikz}

% shadowsライブラリ
\usetikzlibrary{
    positioning,
    arrows.meta,
    calc,
    shadows,
    shadows.blur,
    intersections
}

\tcbuselibrary{skins, breakable, theorems}
\usepackage{enumitem}
\setlist[enumerate,1]{label=(\arabic*)}
\setlist[itemize]{leftmargin=*}
\newcommand{\ds}{\displaystyle}

%----- レイアウト設定 -----
\geometry{
  left=15mm,
  right=15mm,
  top=20mm,
  bottom=15mm,
  headheight=25pt
}

%----- 数式環境の上下の余白調整 -----
\AtBeginDocument{
  \setlength{\abovedisplayskip}{5pt}
  \setlength{\belowdisplayskip}{5pt}
  \setlength{\abovedisplayshortskip}{0pt}
  \setlength{\belowdisplayshortskip}{3pt}
}

%===========================================================
%  デザイン設定(白黒印刷対応・ハイコントラスト)
%===========================================================

%--- 色の定義 ---
\definecolor{printBlue}{RGB}{0, 50, 100}     % 濃紺(例題・解説)
\definecolor{printRed}{RGB}{140, 20, 20}     % 濃エンジ(練習・重要)
\definecolor{printTeal}{RGB}{0, 60, 60}      % 濃い青緑(全体課題)
\definecolor{gridColor}{gray}{0.75}          % 解答欄の方眼

%--- 共通スタイル定義 ---
\tcbset{
    chartbox/.style={
        enhanced,
        fonttitle=\sffamily\bfseries,
        boxrule=1pt,
        arc=2pt,
        top=1.0em,
        nobeforeafter,
        enlarge left by=-2mm,
        enlarge right by=-2mm,
        drop fuzzy shadow,
        colback=white,
        attach boxed title to top left={xshift=10pt, yshift*=-\tcboxedtitleheight/2},
        boxed title style={frame hidden, sharp corners, rounded corners=southeast, arc=3pt}
    }
}

% 各種ボックス環境定義
\newenvironment{overall}[1]{
\begin{tcolorbox}[
    chartbox,
    colframe=printTeal,
    coltitle=white,
    title=\textbf{全体課題 #1},
    boxed title style={colback=printTeal},
]}
{\end{tcolorbox}}

\newenvironment{any}[1]{
    \begin{tcolorbox}[
        enhanced,
        colframe=black,
        colback=white,
        coltitle=white,
        title=\textbf{#1},
        fonttitle=\sffamily,
        attach boxed title to top left={xshift=3mm, yshift*=-\tcboxedtitleheight/2},
        boxed title style={frame hidden, colback=black, sharp corners, rounded corners=northeast, arc=3pt},
        boxrule=1pt,
        arc=2pt,
        top=1em,
        nobeforeafter,
        enlarge left by=-2mm,
        enlarge right by=-2mm,
        segmentation style={draw=black!60, dashed}
    ]}
{ \end{tcolorbox}}

\newenvironment{eg}[1]{
\begin{tcolorbox}[
    chartbox,
    colframe=printBlue,
    coltitle=white,
    title=\textbf{例題 #1},
    boxed title style={colback=printBlue},
    segmentation style={draw=printBlue, line width=0.5pt, dashed}
]}
{\end{tcolorbox}}

\newenvironment{prac}[1]{
\begin{tcolorbox}[
    chartbox,
    colframe=printRed,
    coltitle=white,
    title=\textbf{練習 #1},
    boxed title style={colback=printRed}
]}
{\end{tcolorbox}}

\newenvironment{answer}[1][height fill]{
    \begin{tcolorbox}[
        enhanced,
        title={Memo / Answer},
        colframe=black!80,
        colback=white,
        coltitle=black!60,
        fonttitle=\sffamily\bfseries,
        attach boxed title to top left={xshift=5mm, yshift*=-\tcboxedtitleheight/2},
        boxed title style={frame hidden, colback=white},
        boxrule=1pt,
        arc=1pt,
        nobeforeafter,
        enlarge left by=2mm, 
        enlarge right by=2mm, 
        height fill,
        segmentation style={draw=black!20, solid},
        underlay={
            \begin{tcbclipinterior}
                \draw[step=5mm, black!5, ultra thin] (interior.south west) grid (interior.north east);
            \end{tcbclipinterior}
        }, 
        #1
    ]}
{ \end{tcolorbox}}

\newenvironment{proofbox}[1][height fill]{
    \begin{tcolorbox}[
        enhanced,
        title={Proof},
        colframe=black!80,
        colback=white,
        coltitle=black!60,
        fonttitle=\sffamily\bfseries,
        attach boxed title to top left={xshift=5mm, yshift*=-\tcboxedtitleheight/2},
        boxed title style={frame hidden, colback=white},
        boxrule=1pt,
        arc=1pt,
        nobeforeafter,
        width=\linewidth,
        underlay={
            \begin{tcbclipinterior}
                \draw[step=5mm, black!5, ultra thin] (interior.south west) grid (interior.north east);
            \end{tcbclipinterior}
        },
        #1
    ]}
{ \end{tcolorbox}}

%----- 段組の設定 -----
\setlength{\columnsep}{15mm}
\setlength{\columnseprule}{0.4pt}
\renewcommand{\columnseprulecolor}{\color{black!30}}

%----- ヘッダーの設定 -----
\pagestyle{fancy}
\fancyhf{}

% ヘッダーデザイン
\fancyhead[C]{%
    \begin{tikzpicture}[remember picture, overlay]
        \node[anchor=north west, fill=printBlue, minimum width=\paperwidth, minimum height=5pt] at (current page.north west) {};
    \end{tikzpicture}
}
\fancyhead[L]{\small \textcolor{black!90}{数学\ajRoman{2} $>$ 第2章 複素数と方程式 $>$ 第3回--\textbf{複素数の導入}}}
\fancyhead[R]{\small 年 \hspace{1cm} 組 \hspace{1cm} 番 \quad 氏名 \hspace{6cm}}
\renewcommand{\headrulewidth}{0pt}

\begin{document}

\begin{multicols*}{2}

%===========================================================
% 左カラム: 複素数の定義
%===========================================================

{\large \textbf{1. 数の拡張:New Game!}}

\begin{any}{虚数単位 $i$ の定義}
実数の世界では $x^2=-1$ となる $x$ は存在しない(実数は2乗すると必ず0以上になるから).
そこで, \textbf{2乗して $-1$ になる新しい数} を考え, \textbf{虚数単位 $i$} で表す.
\begin{center}
\begin{tcolorbox}[colframe=printRed, colback=white, boxrule=1.5pt, arc=0pt]
    \centering
    $\ds \Large i^2 = -1 $
\end{tcolorbox}
\end{center}
\end{any}

\begin{any}{複素数 $a+bi$}
実数 $a, b$ と虚数単位 $i$ を用いて, $\boldsymbol{a+bi}$ の形で表される数を\textbf{複素数}という.
\begin{itemize}
    \item $a$ を\textbf{実部}, $b$ を\textbf{虚部}という.
    \item $b \neq 0$ のとき, その複素数を\textbf{虚数}という.
    \item 特に $a=0$ かつ $b \neq 0$ のとき($bi$の形), \textbf{純虚数}という.
\end{itemize}
\end{any}

\begin{eg}{1(複素数の分類)}
次の複素数の実部と虚部をいえ. また, 虚数, 純虚数であるものを選べ.
\[ 3-2i, \quad -5i, \quad \sqrt{3}, \quad \frac{1+i}{2} \]
\end{eg}

\begin{answer}
\vspace{4cm}
\end{answer}

{\large \textbf{2. 複素数の相等}}

\begin{any}{複素数が等しいとは?}
$a, b, c, d$ が実数のとき, 
\begin{itemize}
    \item $\boldsymbol{a+bi = c+di \iff a=c \text{ かつ } b=d}$
    \item $\boldsymbol{a+bi = 0 \quad\quad \iff a=0 \text{ かつ } b=0}$
\end{itemize}
つまり, \textbf{実部同士・虚部同士をそれぞれ比較}すればよい.
\end{any}

\begin{eg}{2(恒等式のように解く)}
等式 $(2x+1) + (y-3)i = 5 + 2i$ を満たす実数 $x, y$ の値を求めよ.
\end{eg}

\begin{answer}
% メモ
% 実部比較: 2x+1 = 5
% 虚部比較: y-3 = 2
\vspace{3cm}
\end{answer}


\columnbreak

%===========================================================
% 右カラム: 計算と負の平方根
%===========================================================

{\large \textbf{3. 複素数の計算}}

\begin{any}{文字式と同じルール+α}
複素数の四則演算は, $i$ を文字 $x$ と同じように扱って計算する.
ただし, \textbf{$\boldsymbol{i^2}$ が出てきたら $\boldsymbol{-1}$ に置き換える}こと.

\begin{itemize}
    \item \textbf{加法・減法}: 実部同士, 虚部同士を計算する.
    \[ (a+bi) \pm (c+di) = (a\pm c) + (b\pm d)i \]
    \item \textbf{乗法}: 分配法則で展開し, $i^2=-1$ を利用して整理する.
    \[ (a+bi)(c+di) = (ac-bd) + (ad+bc)i \]
\end{itemize}
\end{any}

\begin{eg}{3(計算の基本)}
次の計算をせよ.
\begin{enumerate}
    \item $(4+2i) + (3-5i)$
    \item $(2+i)(3-2i)$
\end{enumerate}
\end{eg}

\begin{answer}
% 解説用スペース
% (2) = 6 -4i +3i -2i^2 = 6 -i -2(-1) = 8-i
\vspace{5cm}
\end{answer}

{\large \textbf{4. 負の数の平方根}}

\begin{any}{ルートの中がマイナスになったら}
$a>0$ のとき, $\sqrt{-a}$ は「2乗して $-a$ になる数」であるから,
\[ \sqrt{-a} = \sqrt{a}i \]
と表す.
\begin{itemize}
    \item 例: $\sqrt{-4} = \sqrt{4}i = 2i$
    \item \textbf{注意:} 計算するときは, \textbf{最初に $i$ の形に直してから}計算すること.
    \item \textcolor{red}{\textbf{誤答例:}} $\sqrt{-2}\times\sqrt{-3} = \sqrt{(-2)\times(-3)} = \sqrt{6}$ \quad ($\leftarrow$ ダメ!!)
\end{itemize}
\end{any}

\begin{eg}{4(負の数の平方根)}
次の数を $i$ を用いて表せ. また, 計算せよ.
\begin{enumerate}
    \item $\sqrt{-12}$
    \item $\sqrt{-2} \times \sqrt{-6}$
\end{enumerate}
\end{eg}

\begin{answer}
\end{answer}

\end{multicols*}

\newpage

%===========================================================
% 裏面: 演習問題
%===========================================================

\begin{multicols*}{2}

{\large \textbf{確認テスト}}

\begin{prac}{A1(基本の確認)}
(1) 次の複素数の実部と虚部をいえ.
\[ 5-i, \quad -3i, \quad 7 \]

(2) 次の等式を満たす実数 $x, y$ の値を求めよ.
\[ (x-2) + (x+y)i = 0 \]
\end{prac}

\begin{prac}{A2(複素数の計算)}
次の計算をせよ.
\begin{enumerate}
    \item $(2-4i) - (1-3i)$
    \item $(3+2i)(1-4i)$
    \item $(2+3i)^2$
\end{enumerate}
\end{prac}

\begin{answer}

\end{answer}

\columnbreak

\begin{prac}{B1(負の数の平方根)}
次の計算をせよ. \textbf{注意: 必ず先に $i$ を外に出すこと.}
\begin{enumerate}
    \item $\sqrt{-3} \times \sqrt{-12}$
    \item $\ds \frac{\sqrt{-16}}{\sqrt{2}}$
    \item $\ds \frac{\sqrt{27}}{\sqrt{-3}}$
\end{enumerate}
\end{prac}

\begin{prac}{B2(式の値)}
$x = 2+i$ のとき, 次の式の値を求めよ.
\[ x^2 - 4x + 5 \]
(ヒント: そのまま代入するか, $x-2=i$ として両辺を2乗して関係式を作る)
\end{prac}

\begin{answer}
\end{answer}

\end{multicols*}

\end{document}