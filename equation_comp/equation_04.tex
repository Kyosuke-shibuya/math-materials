\documentclass[b4paper, landscape, dvipdfmx]{jsarticle}
%----- 必要なパッケージ -----
\usepackage{fancybox,ascmac,otf,ulem}
\usepackage{amssymb, amsthm}
\usepackage[leqno]{amsmath}
\usepackage{wrapfig}
\usepackage{geometry}
\usepackage{multicol}
\usepackage{tcolorbox}
\usepackage{xcolor}
\usepackage{fancyhdr}
\usepackage{tikz}

% shadowsライブラリ
\usetikzlibrary{
    positioning,
    arrows.meta,
    calc,
    shadows,
    shadows.blur,
    intersections,
    patterns
}

\tcbuselibrary{skins, breakable, theorems}
\usepackage{enumitem}
\setlist[enumerate,1]{label=(\arabic*)}
\setlist[itemize]{leftmargin=*}
\newcommand{\ds}{\displaystyle}

%----- レイアウト設定 -----
\geometry{
  left=15mm,
  right=15mm,
  top=20mm,
  bottom=15mm,
  headheight=25pt
}

%----- 数式環境の上下の余白調整 -----
\AtBeginDocument{
  \setlength{\abovedisplayskip}{5pt}
  \setlength{\belowdisplayskip}{5pt}
  \setlength{\abovedisplayshortskip}{0pt}
  \setlength{\belowdisplayshortskip}{3pt}
}

%===========================================================
%  デザイン設定・共通定義
%===========================================================

\definecolor{chartBlue}{RGB}{0, 65, 120}    % 青
\definecolor{chartRed}{RGB}{160, 20, 20}    % 赤
\definecolor{chartOrange}{RGB}{200, 100, 0} % オレンジ
\definecolor{chartTeal}{RGB}{0, 100, 100}   % ティール

\tcbset{
    chartbox/.style={
        enhanced, fonttitle=\sffamily\bfseries, boxrule=1pt, arc=3pt,
        top=1.2em, nobeforeafter,
        enlarge left by=-2mm, enlarge right by=-2mm,
        drop fuzzy shadow,
        attach boxed title to top left={xshift=10pt, yshift*=-\tcboxedtitleheight/2},
        boxed title style={frame hidden, sharp corners, rounded corners=southeast, arc=4pt}
    }
}

% 解説用ボックス (黒/白)
\newtcolorbox{any}[1]{
    enlarge left by=0mm, enlarge right by=0mm,
    enhanced, frame hidden, colback=white, title={#1},
    attach boxed title to top left={xshift=0mm, yshift=0mm},
    coltitle=white, fonttitle=\bfseries\sffamily,
    boxed title style={
        colback=black!80, frame hidden, arc=4pt, outer arc=4pt,
        sharp corners=south, boxrule=0pt,
        top=1mm, bottom=1mm, left=3mm, right=3mm
    },
    underlay boxed title={
        \draw[thick, black!80] (title.south west) -- (title.south west-|frame.east);
    },
    breakable, top=5mm, left=2mm, right=2mm, bottom=0mm,
    before skip=1em, after skip=1em,
    segmentation style={draw=black!40, dashed}
}

% 定理 (thm) - ティール/重要
\newenvironment{thm}[1]{
\begin{tcolorbox}[
    chartbox, colframe=chartTeal, colback=white, coltitle=white,
    title=\textbf{定理:#1}, boxed title style={colback=chartTeal},
    skin=bicolor, colbacklower=white,
    segmentation style={draw=chartTeal, dashed, line width=1pt}
]}{\end{tcolorbox}}

% 例題 (eg) - 青
\newenvironment{eg}[1]{
\begin{tcolorbox}[
    chartbox, colframe=chartBlue, colback=white, coltitle=white,
    title=\textbf{例題 #1}, boxed title style={colback=chartBlue},
    skin=bicolor, colbacklower=white!95!blue,
    segmentation style={draw=chartBlue, line width=0pt}
]}{\end{tcolorbox}}

% 練習 (prac) - 赤
\newenvironment{prac}[1]{
\begin{tcolorbox}[
    chartbox, colframe=chartRed, colback=white, coltitle=white,
    title=\textbf{練習 #1}, boxed title style={colback=chartRed}
]}{\end{tcolorbox}}

% 解答欄 (answer)
\newenvironment{answer}[1][height fill]{
    \noindent
    \begin{tcolorbox}[
        enhanced, title={Memo / Answer},
        colframe=black!80, colback=white, coltitle=black!80,
        fonttitle=\sffamily\bfseries,
        attach boxed title to top left={xshift=5mm, yshift*=-\tcboxedtitleheight/2},
        boxed title style={frame hidden, colback=white},
        boxrule=1pt, arc=1pt, nobeforeafter,
        enlarge left by=-2mm, enlarge right by=-2mm,
        width=\linewidth, #1,
        underlay={ \begin{tcbclipinterior} \draw[step=5mm, black!5, ultra thin] (interior.south west) grid (interior.north east); \end{tcbclipinterior} }
    ]}{\end{tcolorbox}}

%----- 段組の設定 -----
\setlength{\columnsep}{15mm}
\setlength{\columnseprule}{0.4pt}
\renewcommand{\columnseprulecolor}{\color{black!30}}

%----- ヘッダーの設定 -----
\pagestyle{fancy}
\fancyhf{}

% ヘッダーデザイン
\fancyhead[C]{%
    \begin{tikzpicture}[remember picture, overlay]
        \node[anchor=north west, fill=chartBlue, minimum width=\paperwidth, minimum height=5pt] at (current page.north west) {};
    \end{tikzpicture}
}
\fancyhead[L]{\small \textcolor{black!90}{数学\ajRoman{2} $>$ 第2章 複素数と方程式 $>$ 第4回--\textbf{共役複素数と判別式}}}
\fancyhead[R]{\small 年 \hspace{1cm} 組 \hspace{1cm} 番 \quad 氏名 \hspace{6cm}}
\renewcommand{\headrulewidth}{0pt}

\begin{document}

\begin{multicols*}{2}

%===========================================================
% 左カラム: 共役な複素数
%===========================================================

{\large \textbf{1. 共役(きょうやく)な複素数}}

\begin{any}{虚部の符号を変えたペア}
複素数 $z = a+bi$ に対して, 虚部の符号を変えた複素数 $a-bi$ を, $z$ の\textbf{共役な複素数}といい, 記号 $\boldsymbol{\bar{z}}$ で表す.
\begin{center}
\begin{tcolorbox}[colframe=chartRed, colback=white, boxrule=1.5pt, arc=0pt]
    \centering
    $z = a+bi$ のとき, $\quad \bar{z} = \boldsymbol{a-bi}$
\end{tcolorbox}
\end{center}
\textbf{性質:和と積が実数になる}
このペアは非常に仲が良く, 足しても掛けても $i$ が消える.
\begin{itemize}
    \item \textbf{和}: $(a+bi) + (a-bi) = 2a$ \quad (実数)
    \item \textbf{積}: $(a+bi)(a-bi) = a^2 - (bi)^2 = a^2 - b^2(-1) = \boldsymbol{a^2+b^2}$ \quad (実数)
\end{itemize}


\begin{eg}{1(共役な複素数)}
次の複素数の共役な複素数をいえ. また, もとの複素数との\textbf{積}を求めよ.
\begin{enumerate}
    \item $3+2i$
    \item $5i$
    \item $-4$
\end{enumerate}
\end{eg}

\begin{answer}
\vspace{8cm}
\end{answer}
\end{any}
%===========================================================
% 右カラム: 複素数の割り算
%===========================================================

\columnbreak

{\large \textbf{2. 複素数の除法}}

\begin{any}{分母の実数化}
複素数の割り算(分数)は, \textbf{分母の共役な複素数}を分母・分子に掛けることで, 分母を実数にして計算する.
これは平方根の「有理化」と同じ発想である.

\[ \frac{c+di}{a+bi} = \frac{(c+di)\textcolor{chartRed}{(a-bi)}}{(a+bi)\textcolor{chartRed}{(a-bi)}} = \frac{(ac+bd)+(ad-bc)i}{a^2+b^2} \]
(公式として覚えるのではなく, \textbf{「共役を掛ける」手順}を覚えること!)


\begin{eg}{2(分母の実数化)}
次の式を $a+bi$ の形で表せ.
\begin{enumerate}
    \item $\ds \frac{1+2i}{3+i}$
    \item $\ds \frac{5}{2-i}$
\end{enumerate}
\end{eg}

\begin{any}{計算のコツ}
\begin{itemize}
    \item 分母は $(a+bi)(a-bi) = a^2+b^2$ なので, 瞬時に計算できるようにしよう.
    \item 約分を忘れないように注意.
\end{itemize}
\end{any}

\begin{answer}
% (1) 分母分子に 3-i を掛ける
% 分母: 9+1=10
% 分子: (1+2i)(3-i) = 3 -i +6i -2i^2 = 5+5i
% よって 5+5i / 10 = 1/2 + 1/2 i
\vspace{8cm}
\end{answer}
\end{any}


\end{multicols*}


%===========================================================
% 表面:授業用スライド・板書用
%===========================================================
\begin{multicols*}{2}

%--- 左カラム:講義内容 ---
{\large\textbf{3. 解の公式再訪}}
\begin{any}{虚数解の出現}
2次方程式 $ax^2+bx+c=0$ の解の公式
\[ x = \frac{-b \pm \sqrt{b^2-4ac}}{2a} \]
において, ルートの中身 $b^2-4ac < 0$ となる場合, 負の数の平方根(虚数)を用いることで解を表すことができる.
\begin{itemize}
    \item これまでは $\to$ \textbf{「解なし」} (実数解をもたない)
    \item これからは $\to$ \textbf{「異なる2つの虚数解」}
\end{itemize}

\begin{eg}{3(虚数解を持つ2次方程式)}
次の2次方程式を解け.
\begin{enumerate}
    \item $x^2 + x + 1 = 0$
    \item $2x^2 - 4x + 3 = 0$
\end{enumerate}
\end{eg}

\begin{answer}[height=10cm]
    
\end{answer}
\end{any}

\columnbreak

\begin{any}{判別式再訪}
    

\begin{thm}{判別式 $D$ (Discriminant)}
2次方程式 $ax^2+bx+c=0$ の解の種類は, \textbf{判別式} $\boldsymbol{D = b^2 - 4ac}$ の符号で決まる.
\end{thm}

\begin{center}
\begin{tikzpicture}[scale=0.6, >=stealth]
    % D>0
    \begin{scope}[xshift=0cm]
        \draw[->] (-1.5,0) -- (1.5,0) node[right] {$x$};
        \draw[thick, chartBlue] plot[domain=-1.2:1.2] (\x, {\x*\x - 0.5});
        \node[below, font=\footnotesize] at (0,-1) {$D>0$};
        \node[below, font=\footnotesize] at (0,-1.6) {異なる2実数解};
    \end{scope}
    % D=0
    \begin{scope}[xshift=4cm]
        \draw[->] (-1.5,0) -- (1.5,0) node[right] {$x$};
        \draw[thick, chartTeal] plot[domain=-1.2:1.2] (\x, {\x*\x});
        \node[below, font=\footnotesize] at (0,-1) {$D=0$};
        \node[below, font=\footnotesize] at (0,-1.6) {重解};
    \end{scope}
    % D<0
    \begin{scope}[xshift=8cm]
        \draw[->] (-1.5,0) -- (1.5,0) node[right] {$x$};
        \draw[thick, chartRed] plot[domain=-1.2:1.2] (\x, {\x*\x + 0.5});
        \node[below, font=\footnotesize] at (0,-1) {$D<0$};
        \node[below, font=\footnotesize] at (0,-1.6) {異なる2虚数解};
    \end{scope}
\end{tikzpicture}
\end{center}
\begin{itemize}
    \item $D > 0 \iff$ 異なる2つの実数解
    \item $D = 0 \iff$ ただ1つの実数解 (重解)
    \item $D < 0 \iff$ 異なる2つの虚数解
\end{itemize}
\textbf{※ 偶数公式}: $x$の係数が偶数($2b'$)の時は $\ds \frac{D}{4} = (b')^2 - ac$ を利用する.


\begin{eg}{4(解の種類の判別)}
次の2次方程式の解の種類を判別せよ.
\begin{enumerate}
    \item $x^2 - 3x - 5 = 0$
    \item $4x^2 - 12x + 9 = 0$
    \item $x^2 + 2x + 4 = 0$
\end{enumerate}
\end{eg}

\begin{answer}[height=7cm]
    
\end{answer}

\end{any}

\end{multicols*}

\newpage


%===========================================================
% 裏面:確認テスト(問題編)
%===========================================================
\begin{multicols*}{2}

\begin{any}{確認テスト(問題)}

\begin{prac}{A1(共役な複素数)}
次の複素数の共役な複素数を答えよ.
\begin{enumerate}
    \item $1-i$
    \item $\sqrt{3} + i$
    \item $-3i$
\end{enumerate}
\end{prac}

\begin{prac}{A2(計算練習)}
次の式を計算し, $a+bi$ の形で表せ.
\begin{enumerate}
    \item $\ds \frac{2-i}{1+i}$
    \item $\ds \frac{1}{i}$
    \item $\ds \frac{3+2i}{2-3i}$
\end{enumerate}
\end{prac}

\begin{answer}
\vspace{9cm}
\end{answer}

\columnbreak

\begin{prac}{B1(方程式への応用)}
等式 $(1+i)x - (1-3i)y = -1+i$ を満たす実数 $x, y$ の値を求めよ.
\end{prac}

\begin{any}{Hint}
左辺を計算して (実部)$+$(虚部)$i$ の形に整理し, 両辺の係数を比較する.
あるいは, $x, y$ が混ざっているのが嫌なら連立方程式とみることもできる.
\end{any}



\begin{prac}{B2(式の値の工夫)}
$x = \ds \frac{1+\sqrt{3}i}{2}$ のとき, $x^2-x+1$ の値を求めよ.
\end{prac}

\begin{answer}[height=12cm]
\end{answer}
\end{any}

\end{multicols*}

\begin{multicols*}{2}

\begin{any}{確認テスト(問題)}
    
\begin{prac}{A3(基本:2次方程式の解法)}
次の2次方程式を解け.
\begin{enumerate}
    \item $3x^2 + 5x - 2 = 0$
    \item $x^2 - 2x + 5 = 0$
    \item $2x^2 + 3x + 2 = 0$
\end{enumerate}
\end{prac}

\begin{prac}{A4(基本:解の判別)}
次の2次方程式の解の種類を判別せよ.
\begin{enumerate}
    \item $2x^2 - x + 3 = 0$
    \item $9x^2 + 6x + 1 = 0$
    \item $x^2 - \sqrt{5}x + 1 = 0$
\end{enumerate}
\end{prac}

% A問題用解答欄
\begin{answer}[height=8cm]
\end{answer}

\columnbreak

\begin{prac}{B3(標準:条件を満たす定数)}
2次方程式 $x^2 - 4x + k = 0$ が次のような解を持つように, 定数 $k$ の値または範囲を定めよ.
\begin{enumerate}
    \item 異なる2つの実数解を持つ
    \item 重解を持つ (また, そのときの重解を求めよ)
    \item 異なる2つの虚数解を持つ
\end{enumerate}
\end{prac}

\begin{prac}{B4(発展:グラフとの融合)}
2次方程式 $2x^2 - 4x + 3 = 0$ の解を求めよ.
また, 2次関数 $y=2x^2-4x+3$ のグラフを書き, 頂点の座標を求めよ.
($x$軸と共有点を持たないことを確認しよう)
\end{prac}

% B問題用解答欄
\begin{answer}[height=12cm]
\end{answer}
\end{any}
\end{multicols*}

\newpage

%===========================================================
% 別紙:確認テスト(解答編)
%===========================================================

\begin{multicols*}{2}

\begin{any}{確認テスト(解答・解説)}

\begin{prac}{A1(共役な複素数)}
次の複素数の共役な複素数を答えよ.
\begin{enumerate}
    \item $1-i$
    \item $\sqrt{3} + i$
    \item $-3i$
\end{enumerate}
\end{prac}

\begin{prac}{A2(計算練習)}
次の式を計算し, $a+bi$ の形で表せ.
\begin{enumerate}
    \item $\ds \frac{2-i}{1+i}$
    \item $\ds \frac{1}{i}$
    \item $\ds \frac{3+2i}{2-3i}$
\end{enumerate}
\end{prac}

\begin{answer}
\color{chartRed}
\textbf{A1}
\begin{enumerate}
    \item 虚部の符号を変えるので $\boldsymbol{1+i}$
    \item $\boldsymbol{\sqrt{3}-i}$
    \item $0-3i$ と考えると $\boldsymbol{3i}$
\end{enumerate}

\vspace{0.5em}
\hrule
\vspace{0.5em}

\textbf{A2}
\begin{enumerate}
    \item $\ds \frac{(2-i)(1-i)}{(1+i)(1-i)} = \frac{2-2i-i+i^2}{1+1} = \frac{1-3i}{2} = \boldsymbol{\frac{1}{2} - \frac{3}{2}i}$
    \item $\ds \frac{i}{i^2} = \frac{i}{-1} = \boldsymbol{-i}$
    \item $\ds \frac{(3+2i)(2+3i)}{(2-3i)(2+3i)} = \frac{6+9i+4i+6i^2}{4+9} = \frac{13i}{13} = \boldsymbol{i}$
\end{enumerate}
\end{answer}

\columnbreak

\begin{prac}{B1(方程式への応用)}
等式 $(1+i)x - (1-3i)y = -1+i$ を満たす実数 $x, y$ の値を求めよ.
\end{prac}

\begin{any}{Hint}
左辺を計算して (実部)$+$(虚部)$i$ の形に整理し, 両辺の係数を比較する.
あるいは, $x, y$ が混ざっているのが嫌なら連立方程式とみることもできる.
\end{any}



\begin{prac}{B2(式の値の工夫)}
$x = \ds \frac{1+\sqrt{3}i}{2}$ のとき, $x^2-x+1$ の値を求めよ.
\end{prac}

\begin{answer}[height=12cm]
\color{chartRed}
\textbf{B1} \\
左辺を整理すると,
$(x-y) + (x+3y)i = -1+i$ \\
$x, y$ は実数だから, 実部と虚部を比較して
\[ \begin{cases} x-y = -1 \\ x+3y = 1 \end{cases} \]
これを解いて, $\boldsymbol{x=-\frac{1}{2}, y=\frac{1}{2}}$

\vspace{0.5em}
\hrule
\vspace{0.5em}

\textbf{B2} \\
与式より $2x = 1+\sqrt{3}i \iff 2x-1 = \sqrt{3}i$. \\
両辺を2乗して, $(2x-1)^2 = -3$ \\
$4x^2 - 4x + 1 = -3 \iff 4x^2 - 4x + 4 = 0$ \\
全体を4で割ると $\boldsymbol{x^2 - x + 1 = 0}$. \\
よって求める値は $\boldsymbol{0}$.
\end{answer}
\end{any}

\end{multicols*}

\begin{multicols*}{2}

\begin{any}{確認テスト(解答・解説)}

\begin{prac}{A3(基本:2次方程式の解法)}
次の2次方程式を解け.
\begin{enumerate}
    \item $3x^2 + 5x - 2 = 0$
    \item $x^2 - 2x + 5 = 0$
    \item $2x^2 + 3x + 2 = 0$
\end{enumerate}
\end{prac}

\begin{prac}{A4(基本:解の判別)}
次の2次方程式の解の種類を判別せよ.
\begin{enumerate}
    \item $2x^2 - x + 3 = 0$
    \item $9x^2 + 6x + 1 = 0$
    \item $x^2 - \sqrt{5}x + 1 = 0$
\end{enumerate}
\end{prac}

\begin{answer}[height fill]
\color{chartRed}
\textbf{A3}
\begin{enumerate}
    \item 因数分解できる. $(x+2)(3x-1)=0$ より $\ds \boldsymbol{x = -2, \frac{1}{3}}$
    \item 解の公式より $\ds x = \frac{-(-2) \pm \sqrt{(-2)^2 - 4 \cdot 1 \cdot 5}}{2} = \frac{2 \pm \sqrt{-16}}{2} = \frac{2 \pm 4i}{2} = \boldsymbol{1 \pm 2i}$
    \item 解の公式より $\ds x = \frac{-3 \pm \sqrt{3^2 - 4 \cdot 2 \cdot 2}}{2 \cdot 2} = \frac{-3 \pm \sqrt{9-16}}{4} = \boldsymbol{\frac{-3 \pm \sqrt{7}i}{4}}$
\end{enumerate}

\vspace{0.5em}
\hrule
\vspace{0.5em}

\textbf{A4}
\begin{enumerate}
    \item $D = (-1)^2 - 4 \cdot 2 \cdot 3 = 1 - 24 = -23 < 0$. よって\textbf{異なる2つの虚数解}.
    \item $D/4 = 3^2 - 9 \cdot 1 = 9 - 9 = 0$. よって\textbf{ただ1つの実数解(重解)}.
    \item $D = (-\sqrt{5})^2 - 4 \cdot 1 \cdot 1 = 5 - 4 = 1 > 0$. よって\textbf{異なる2つの実数解}.
\end{enumerate}
\end{answer}

\columnbreak

\begin{prac}{B3(標準:条件を満たす定数)}
2次方程式 $x^2 - 4x + k = 0$ が次のような解を持つように, 定数 $k$ の値または範囲を定めよ.
\begin{enumerate}
    \item 異なる2つの実数解を持つ
    \item 重解を持つ (また, そのときの重解を求めよ)
    \item 異なる2つの虚数解を持つ
\end{enumerate}
\end{prac}

\begin{prac}{B4(発展:グラフとの融合)}
2次方程式 $2x^2 - 4x + 3 = 0$ の解を求めよ.
また, 2次関数 $y=2x^2-4x+3$ のグラフを書き, 頂点の座標を求めよ.
\end{prac}

\begin{answer}[height fill]
\color{chartRed}
\textbf{B3} \\
判別式を $D$ とすると, $\ds \frac{D}{4} = (-2)^2 - 1 \cdot k = 4 - k$.
\begin{enumerate}
    \item $D > 0$ より $4-k > 0 \iff \boldsymbol{k < 4}$
    \item $D = 0$ より $4-k = 0 \iff \boldsymbol{k = 4}$. \\
    このとき方程式は $x^2 - 4x + 4 = 0 \iff (x-2)^2=0$ より, 重解は $\boldsymbol{x=2}$.
    \item $D < 0$ より $4-k < 0 \iff \boldsymbol{k > 4}$
\end{enumerate}

\vspace{0.5em}
\hrule
\vspace{0.5em}

\textbf{B4} \\
解の公式より
\[ x = \frac{-(-2) \pm \sqrt{(-2)^2 - 2 \cdot 3}}{2} = \frac{2 \pm \sqrt{4-6}}{2} = \boldsymbol{\frac{2 \pm \sqrt{2}i}{2}} \]

平方完成を行うと,
\[ y = 2(x^2 - 2x) + 3 = 2\{(x-1)^2 - 1\} + 3 = 2(x-1)^2 + 1 \]
よって, \textbf{頂点 (1, 1), 下に凸の放物線}.
\begin{center}
\begin{tikzpicture}[scale=0.7]
    \draw[->] (-1,0) -- (3.5,0) node[right] {$x$};
    \draw[->] (0,-0.5) -- (0,4) node[above] {$y$};
    \draw[thick, chartBlue] plot[domain=0:2.2] (\x, {2*(\x-1)*(\x-1) + 1});
    \draw[dashed] (1,0) node[below]{1} -- (1,1) -- (0,1) node[left]{1};
    \node[right] at (0,3) {3};
    \fill (0,3) circle (2pt);
    \node[above] at (1,1) {頂点};
\end{tikzpicture}
\end{center}
グラフが $x$ 軸より上にある(共有点なし) $\iff$ 虚数解を持つ.
\end{answer}

\end{any}

\end{multicols*}

\end{document}