\documentclass[b4paper, landscape, dvipdfmx]{jsarticle}
%----- 必要なパッケージ -----
\usepackage{fancybox,ascmac,otf}
\usepackage{amssymb, amsthm}
\usepackage[leqno]{amsmath}
\usepackage{geometry}
\usepackage{multicol}
\usepackage{tcolorbox}
\usepackage{xcolor}
\usepackage{fancyhdr}
\usepackage{tikz}

% TikZライブラリ
\usetikzlibrary{
    positioning,
    arrows.meta,
    calc,
    shadows,
    shadows.blur,
    intersections
}

% tcolorboxライブラリ
\tcbuselibrary{skins, breakable, theorems}

\usepackage{enumitem}
\setlist[enumerate,1]{label=(\arabic*)}
\setlist[itemize]{leftmargin=*}
\newcommand{\ds}{\displaystyle}

%----- レイアウト設定 -----
\geometry{
  left=15mm,
  right=15mm,
  top=20mm,
  bottom=15mm,
  headheight=25pt
}

%----- 数式環境の上下の余白調整 -----
\AtBeginDocument{
  \setlength{\abovedisplayskip}{5pt}
  \setlength{\belowdisplayskip}{5pt}
  \setlength{\abovedisplayshortskip}{0pt}
  \setlength{\belowdisplayshortskip}{3pt}
}

%===========================================================
%  デザイン設定
%===========================================================

%--- 色の定義 ---
\definecolor{printBlue}{RGB}{0, 50, 100}     % 濃紺
\definecolor{printRed}{RGB}{140, 20, 20}     % 濃エンジ
\definecolor{printTeal}{RGB}{0, 60, 60}      % 濃い青緑

%--- 共通スタイル定義 ---
\tcbset{
    chartbox/.style={
        enhanced,
        fonttitle=\sffamily\bfseries,
        boxrule=1pt,
        arc=2pt,
        top=1.0em,
        nobeforeafter,
        enlarge left by=-2mm,
        enlarge right by=-2mm,
        drop fuzzy shadow,
        colback=white,
        attach boxed title to top left={xshift=10pt, yshift*=-\tcboxedtitleheight/2},
        boxed title style={frame hidden, sharp corners, rounded corners=southeast, arc=3pt}
    }
}

%--- 各種ボックス環境定義 ---

% セクション・枠組み用 (any)
\newtcolorbox{any}[1]{
    enlarge left by=0mm, enlarge right by=0mm,
    enhanced, frame hidden, colback=white, title={#1},
    attach boxed title to top left={xshift=0mm, yshift=0mm},
    coltitle=white, fonttitle=\bfseries\sffamily,
    boxed title style={
        colback=black!80, frame hidden, arc=4pt, outer arc=4pt,
        sharp corners=south, boxrule=0pt,
        top=1mm, bottom=1mm, left=3mm, right=3mm
    },
    underlay boxed title={
        \draw[thick, black!80] (title.south west) -- (title.south west-|frame.east);
    },
    breakable, top=5mm, left=2mm, right=2mm, bottom=0mm,
    before skip=1em, after skip=1em,
    segmentation style={draw=black!40, dashed}
}

% 例題 (eg)
\newtcolorbox{eg}[1]{
    chartbox,
    colframe=printBlue,
    coltitle=white,
    title=\textbf{例題 #1},
    boxed title style={colback=printBlue},
    segmentation style={draw=printBlue, line width=0.5pt, dashed}
}

% 練習 (prac)
\newtcolorbox{prac}[1]{
    chartbox,
    colframe=printRed,
    coltitle=white,
    title=\textbf{練習 #1},
    boxed title style={colback=printRed}
}

% 定理 (thm)
\newtcolorbox{thm}[1]{
    chartbox,
    colframe=printTeal,
    coltitle=white,
    title=\textbf{#1},
    boxed title style={colback=printTeal}
}

% 解答欄 (answer)
\newtcolorbox{answer}[1][height fill]{
    enhanced,
    title={Memo / Answer},
    colframe=black!80,
    colback=white,
    coltitle=black!60,
    fonttitle=\sffamily\bfseries,
    attach boxed title to top left={xshift=5mm, yshift*=-\tcboxedtitleheight/2},
    boxed title style={frame hidden, colback=white},
    boxrule=1pt,
    arc=1pt,
    nobeforeafter,
    enlarge left by=-2mm, 
    enlarge right by=-2mm, 
    height fill,
    segmentation style={draw=black!20, solid},
    underlay={
        \begin{tcbclipinterior}
            \draw[step=5mm, black!5, ultra thin] (interior.south west) grid (interior.north east);
        \end{tcbclipinterior}
    }, 
    #1
}

%----- 段組の設定 -----
\setlength{\columnsep}{15mm}
\setlength{\columnseprule}{0.4pt}
\renewcommand{\columnseprulecolor}{\color{black!30}}

%----- ヘッダーの設定 -----
\pagestyle{fancy}
\fancyhf{}
\fancyhead[C]{%
    \begin{tikzpicture}[remember picture, overlay]
        \node[anchor=north west, fill=printBlue, minimum width=\paperwidth, minimum height=5pt] at (current page.north west) {};
    \end{tikzpicture}
}
\fancyhead[L]{\small \textcolor{black!90}{数学C $>$ 第1章--平面ベクトル $>$ 第3回 \textbf{成分表示と大きさ}}}
\fancyhead[R]{\small 年 \hspace{1cm} 組 \hspace{1cm} 番 \quad 氏名 \hspace{6cm}}
\renewcommand{\headrulewidth}{0pt}

\begin{document}

%=============================================================================
% 1枚目:成分の定義と計算
%=============================================================================
\begin{multicols}{2}

%-----------------------------------------------------------------------------
% 左カラム:成分表示の仕組み
%-----------------------------------------------------------------------------
\begin{any}{1. 矢印を「移動量」として表す}
    座標平面上でベクトルを考えるとき, 「右にどれだけ, 上にどれだけ進むか」という\textbf{移動量}で表すと計算がしやすくなる. これを成分表示という.
    
    \begin{thm}{成分表示 (Component)}
        ベクトル $\vec{a}$ が, $x$軸方向に $a_1$, $y$軸方向に $a_2$ 進む移動を表すとき, 次のように書く.
        \[ \vec{a} = (a_1, a_2) \]
        この $a_1$ を$x$成分, $a_2$ を$y$成分という.
        
        \vspace{0.5em}
        また, 基本ベクトル $\vec{e}_1=(1,0), \vec{e}_2=(0,1)$ を用いると, 
        \[ \vec{a} = a_1 \vec{e}_1 + a_2 \vec{e}_2 \]
        と分解して表すことができる.
    \end{thm}
    
    \textbf{座標との関係:} \\
    始点を原点 O$(0,0)$ にとったとき, ベクトル $\vec{a}$ の終点 A の座標は, そのまま成分 $(a_1, a_2)$ と一致する.
    
    \begin{center}
    \begin{tikzpicture}[scale=1.0, >=stealth]
        \draw[->, gray] (-0.5,0) -- (4,0) node[right] {$x$};
        \draw[->, gray] (0,-0.5) -- (0,3) node[above] {$y$};
        \coordinate (O) at (0,0);
        \coordinate (A) at (3,2);
        \coordinate (E1) at (1,0);
        \coordinate (E2) at (0,1);
        
        % 補助線
        \draw[dashed] (3,0) -- (A) -- (0,2);
        
        % 成分の移動量表示
        \draw[<->, printBlue!50] (0,-0.3) -- (3,-0.3) node[midway, below] {右に $a_1$ ($x$成分)};
        \draw[<->, printBlue!50] (-0.3,0) -- (-0.3,2) node[midway, left] {上に $a_2$ ($y$成分)};
        
        % ベクトル
        \draw[->, very thick, printBlue] (O) -- (A) node[midway, above left] {$\vec{a}$};
        \fill (A) circle (2pt) node[above right] {$A(a_1, a_2)$};
        
        % 基本ベクトル
        \draw[->, thick, printRed] (O) -- (E1) node[below right, scale=0.8] {$\vec{e}_1$};
        \draw[->, thick, printRed] (O) -- (E2) node[above left, scale=0.8] {$\vec{e}_2$};
    \end{tikzpicture}
    \end{center}
\end{any}

%-----------------------------------------------------------------------------
% 右カラム:成分による演算と大きさ
%-----------------------------------------------------------------------------
\columnbreak

\begin{any}{2. 成分による演算と大きさ}
    成分表示すると, ベクトルの和・差・実数倍は, 単なる「成分ごとの計算」になる.
    
    \begin{thm}{成分の演算ルール}
        $\vec{a}=(a_1, a_2), \vec{b}=(b_1, b_2)$ のとき,
        \begin{enumerate}
            \item \textbf{和}: $\vec{a}+\vec{b} = (a_1+b_1, \ a_2+b_2)$
            \item \textbf{差}: $\vec{a}-\vec{b} = (a_1-b_1, \ a_2-b_2)$
            \item \textbf{実数倍}: $k\vec{a} = (ka_1, \ ka_2)$
        \end{enumerate}
    \end{thm}

    \begin{thm}{ベクトルの大きさ(三平方の定理)}
        $\vec{a} = (a_1, a_2)$ の大きさ(長さ)は,
        \[ |\vec{a}| = \sqrt{{a_1}^2 + {a_2}^2} \]
        2点 $A(a_1, a_2), B(b_1, b_2)$ 間の距離は $|\overrightarrow{\text{AB}}|$ と等しい.
        \[ |\overrightarrow{\text{AB}}| = \sqrt{(b_1-a_1)^2 + (b_2-a_2)^2} \]
    \end{thm}

    \begin{eg}{1 (成分計算)}
        $\vec{a} = (3, -1), \vec{b} = (-2, 4)$ のとき, 次を求めよ.
        \begin{enumerate}
            \item $2\vec{a} - \vec{b}$ の成分
            \item $|2\vec{a} - \vec{b}|$ (大きさ)
        \end{enumerate}
        \tcblower
        \vspace{6cm}
    \end{eg}
\end{any}

\end{multicols}

%=============================================================================
% 2枚目:平行条件と応用
%=============================================================================
\newpage
\begin{multicols}{2}

%-----------------------------------------------------------------------------
% 左カラム:平行なベクトル
%-----------------------------------------------------------------------------
\begin{any}{3. 成分による平行条件}
    ベクトルが平行であるということは, 矢印の\textbf{「傾き」}が同じということである.
    
    \begin{thm}{平行条件 (Component ver.)}
        $\vec{a}=(a_1, a_2) \neq \vec{0}, \ \vec{b}=(b_1, b_2) \neq \vec{0}$ のとき,
        \[ \vec{a} // \vec{b} \iff \vec{b} = k\vec{a} \quad (\text{実数倍}) \]
        
        成分で考えると, $\displaystyle \frac{a_2}{a_1} = \frac{b_2}{b_1}$ (傾きが等しい) ということである.
        分母を払って整理すると, 以下の重要公式が得られる.
        \[ a_1b_2 - a_2b_1 = 0 \quad (\text{たすき掛けの積が等しい}) \]
    \end{thm}

    \begin{tcolorbox}[colback=yellow!10, frame hidden, title={なぜこの公式を使う?}]
        $a_1=0$ の場合など, 分数 ($\frac{a_2}{a_1}$) で書けないケースでも $a_1b_2 - a_2b_1 = 0$ なら問題なく使えるため, この「たすき掛け」の形で覚えるのが安全である.
    \end{tcolorbox}

    \begin{eg}{2 (平行なベクトル)}
        $\vec{a} = (3, -4)$ に平行で, 大きさが $10$ であるベクトル $\vec{x}$ を求めよ.
        \tcblower
        \textbf{方針:} 
        平行なので $\vec{x} = k\vec{a}$ とおける.
        その後, 大きさの条件 $|\vec{x}| = 10$ から $k$ を決定する.
        \vspace{8cm} 
    \end{eg}
\end{any}
%-----------------------------------------------------------------------------
% 右カラム:平行四辺形の頂点
%-----------------------------------------------------------------------------
\columnbreak

\begin{any}{4. 図形の座標を求める}
    「ベクトルが等しい」 $\iff$ 「成分がそれぞれ等しい」
    
    \begin{eg}{3 (平行四辺形の第4頂点)}
        3点 $A(1, 1), B(4, 2), C(5, 5)$ がある. 四角形ABCDが平行四辺形となるような点Dの座標を求めよ.
        
        \tcblower
        \textbf{ヒント:} 
        平行四辺形ABCDにおいて, $\overrightarrow{\text{AD}} = \overrightarrow{\text{BC}}$.
        点Dを $(x, y)$ とおいて成分計算する.
        \vspace{8cm}
    \end{eg}
    
    \begin{tcolorbox}[colback=yellow!10, frame hidden, title={注意点}]
        「平行四辺形ABCD」と順序が指定されている場合は1通りだが, 単に「4点A,B,C,Dを頂点とする平行四辺形」といわれた場合は, 3通りの可能性があることに注意.
    \end{tcolorbox}
\end{any}

\end{multicols}

%=============================================================================
% 3枚目:確認テスト(問題)
%=============================================================================
\newpage
\fancyhead[L]{\small \textcolor{black!90}{数学C $>$ 第1章--平面ベクトル $>$ 第3回--\textbf{確認テスト}}}
\begin{multicols}{2}

\begin{any}{確認テスト (A: 基本)}
    \begin{prac}{A1 (成分計算)}
        $\vec{a} = (4, -3), \vec{b} = (-1, 2)$ とする.
        \begin{enumerate}
            \item $2\vec{a} + 3\vec{b}$ の成分を求めよ.
            \item $|\vec{a}|$ を求めよ.
            \item $|\vec{a} + \vec{b}|$ を求めよ.
        \end{enumerate}
    \end{prac}
    \begin{answer}[height=5cm]
    \end{answer}

    \begin{prac}{A2 (ベクトルの分解)}
        $\vec{a}=(2, 1), \vec{b}=(-1, 3)$ のとき, $\vec{c}=(5, -1)$ を
        $s\vec{a} + t\vec{b}$ の形で表せ.
    \end{prac}
    \begin{answer}[height=5cm]
    \end{answer}
\end{any}

\columnbreak

\begin{any}{確認テスト (B: 標準)}
    \begin{prac}{B1 (平行条件と大きさ)}
        $\vec{a} = (1, -2)$ に平行で, 大きさが $\sqrt{20}$ であるベクトル $\vec{x}$ をすべて求めよ.
    \end{prac}
    \begin{answer}[height=7cm]
    \end{answer}

    \begin{prac}{B2 (平行四辺形)}
        A$(-1, 3)$, B$(2, -1)$, C$(4, 1)$ とする.
        四角形ABDCが平行四辺形となるとき, 頂点Dの座標を求めよ.
        (※頂点の順序に注意: ABCDではなくABDC)
    \end{prac}
    \begin{answer}[height fill]
    \end{answer}
\end{any}

\end{multicols}

%=============================================================================
% 4枚目:確認テスト(解答)
%=============================================================================
\newpage
\fancyhead[L]{\small \textcolor{black!90}{数学C $>$ 第1章--平面ベクトル $>$ 第3回 \textbf{【解答解説】}}}

\begin{multicols}{2}

\begin{any}{解答 (A: 基本)}
    \begin{prac}{A1 (成分計算)}
        (1) $2(4, -3) + 3(-1, 2) = (8, -6) + (-3, 6)$ \\
        $= (8-3, -6+6) = \boldsymbol{(5, 0)}$
        
        (2) $|\vec{a}| = \sqrt{4^2 + (-3)^2} = \sqrt{16+9} = \sqrt{25} = \boldsymbol{5}$
        
        (3) $\vec{a}+\vec{b} = (3, -1)$ なので, \\
        $|\vec{a}+\vec{b}| = \sqrt{3^2 + (-1)^2} = \boldsymbol{\sqrt{10}}$
    \end{prac}

    \begin{answer}[height=8cm]
    \color{printRed}
    \textbf{A2 解答:} \\
    $\vec{c} = s\vec{a} + t\vec{b}$ より,
    \[ (5, -1) = s(2, 1) + t(-1, 3) = (2s-t, \ s+3t) \]
    成分を比較して連立方程式を解く.
    \[
    \begin{cases}
        2s - t = 5 & \cdots (1) \\
        s + 3t = -1 & \cdots (2)
    \end{cases}
    \]
    (1)より $t = 2s - 5$. (2)に代入.
    $s + 3(2s-5) = -1 \implies 7s - 15 = -1 \implies 7s=14 \implies s=2$.
    $t = 4-5 = -1$.
    
    よって, $\boldsymbol{\vec{c} = 2\vec{a} - \vec{b}}$
    \end{answer}
\end{any}

\columnbreak

\begin{any}{解答 (B: 標準)}
    \begin{answer}[height=7cm]
    \color{printRed}
    \textbf{B1 解答:} \\
    $\vec{x}$ は $\vec{a}$ に平行なので, 実数 $k$ を用いて $\vec{x} = k\vec{a} = k(1, -2) = (k, -2k)$ とおける.
    
    大きさが $\sqrt{20} = 2\sqrt{5}$ なので, 
    \begin{align*}
        |\vec{x}|^2 &= 20 \\
        k^2 + (-2k)^2 &= 20 \\
        5k^2 &= 20 \\
        k^2 &= 4 \quad \therefore k = \pm 2
    \end{align*}
    $k=2$ のとき, $\vec{x} = (2, -4)$. \\
    $k=-2$ のとき, $\vec{x} = (-2, 4)$.
    
    \textbf{答え:} $\boldsymbol{\vec{x} = (2, -4), \ (-2, 4)}$
    \end{answer}

    \begin{answer}[height fill]
    \color{printRed}
    \textbf{B2 解答:} \\
    四角形ABDCが平行四辺形のとき, 対辺のベクトルが等しい.
    \[ \overrightarrow{\text{AB}} = \overrightarrow{\text{CD}} \]
    (※頂点順序に注意. ABに対応するのはCD. ACではない)
    
    点Dを $(x, y)$ とすると, 
    $\overrightarrow{\text{CD}} = (x-4, \ y-1)$.
    $\overrightarrow{\text{AB}} = (2-(-1), \ -1-3) = (3, -4)$.
    
    成分比較すると:
    \[ x-4 = 3 \implies x = 7 \]
    \[ y-1 = -4 \implies y = -3 \]
    
    よって, \textbf{D(7, -3)}
    
    (別解: $\overrightarrow{\text{AC}} = \overrightarrow{\text{BD}}$ で解いても同じ結果になる)
    \end{answer}
\end{any}

\end{multicols}
\end{document}