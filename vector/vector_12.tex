\documentclass[b4paper, landscape, dvipdfmx]{jsarticle}
%----- 必要なパッケージ -----
\usepackage{fancybox,ascmac,otf}
\usepackage{amssymb, amsthm}
\usepackage[leqno]{amsmath}
\usepackage{geometry}
\usepackage{multicol}
\usepackage{tcolorbox}
\usepackage{xcolor}
\usepackage{fancyhdr}
\usepackage{tikz}

% TikZライブラリ
\usetikzlibrary{
    positioning,
    arrows.meta,
    calc,
    shadows,
    shadows.blur,
    intersections,
    angles,
    quotes,
    decorations.pathmorphing
}

% tcolorboxライブラリ
\tcbuselibrary{skins, breakable, theorems}

\usepackage{enumitem}
\setlist[enumerate,1]{label=(\arabic*)}
\setlist[itemize]{leftmargin=*}
\newcommand{\ds}{\displaystyle}

%----- レイアウト設定 -----
\geometry{
  left=15mm,
  right=15mm,
  top=20mm,
  bottom=15mm,
  headheight=25pt
}

%----- 数式環境の上下の余白調整 -----
\AtBeginDocument{
  \setlength{\abovedisplayskip}{5pt}
  \setlength{\belowdisplayskip}{5pt}
  \setlength{\abovedisplayshortskip}{0pt}
  \setlength{\belowdisplayshortskip}{3pt}
}

%===========================================================
%  デザイン設定
%===========================================================

%--- 色の定義 ---
\definecolor{printBlue}{RGB}{0, 50, 100}     % 濃紺
\definecolor{printRed}{RGB}{140, 20, 20}     % 濃エンジ
\definecolor{printTeal}{RGB}{0, 60, 60}      % 濃い青緑

%--- 共通スタイル定義 ---
\tcbset{
    chartbox/.style={
        enhanced,
        fonttitle=\sffamily\bfseries,
        boxrule=1pt,
        arc=2pt,
        top=1.0em,
        nobeforeafter,
        enlarge left by=-2mm,
        enlarge right by=-2mm,
        drop fuzzy shadow,
        colback=white,
        attach boxed title to top left={xshift=10pt, yshift*=-\tcboxedtitleheight/2},
        boxed title style={frame hidden, sharp corners, rounded corners=southeast, arc=3pt}
    }
}

%--- 各種ボックス環境定義 ---
\newtcolorbox{any}[1]{
    enlarge left by=0mm, enlarge right by=0mm,
    enhanced, frame hidden, colback=white, title={#1},
    attach boxed title to top left={xshift=0mm, yshift=0mm},
    coltitle=white, fonttitle=\bfseries\sffamily,
    boxed title style={
        colback=black!80, frame hidden, arc=4pt, outer arc=4pt,
        sharp corners=south, boxrule=0pt,
        top=1mm, bottom=1mm, left=3mm, right=3mm
    },
    underlay boxed title={
        \draw[thick, black!80] (title.south west) -- (title.south west-|frame.east);
    },
    breakable, top=5mm, left=2mm, right=2mm, bottom=0mm,
    before skip=1em, after skip=1em,
    segmentation style={draw=black!40, dashed}
}

\newtcolorbox{eg}[1]{
    chartbox,
    colframe=printBlue,
    coltitle=white,
    title=\textbf{例題 #1},
    boxed title style={colback=printBlue},
    segmentation style={draw=printBlue, line width=0.5pt, dashed}
}

\newtcolorbox{prac}[1]{
    chartbox,
    colframe=printRed,
    coltitle=white,
    title=\textbf{練習 #1},
    boxed title style={colback=printRed}
}

\newtcolorbox{thm}[1]{
    chartbox,
    colframe=printTeal,
    coltitle=white,
    title=\textbf{#1},
    boxed title style={colback=printTeal}
}

\newtcolorbox{answer}[1][height fill]{
    enhanced,
    title={Memo / Answer},
    colframe=black!80,
    colback=white,
    coltitle=black!60,
    fonttitle=\sffamily\bfseries,
    attach boxed title to top left={xshift=5mm, yshift*=-\tcboxedtitleheight/2},
    boxed title style={frame hidden, colback=white},
    boxrule=1pt,
    arc=1pt,
    nobeforeafter,
    enlarge left by=-2mm, 
    enlarge right by=-2mm, 
    height fill,
    segmentation style={draw=black!20, solid},
    underlay={
        \begin{tcbclipinterior}
            \draw[step=5mm, black!5, ultra thin] (interior.south west) grid (interior.north east);
        \end{tcbclipinterior}
    }, 
    #1
}

%----- ヘッダーの設定 -----
\pagestyle{fancy}
\fancyhf{}
\fancyhead[C]{%
    \begin{tikzpicture}[remember picture, overlay]
        \node[anchor=north west, fill=printBlue, minimum width=\paperwidth, minimum height=5pt] at (current page.north west) {};
    \end{tikzpicture}
}
\fancyhead[L]{\small \textcolor{black!90}{数学C $>$ 第1章--平面ベクトル $>$ 第12回 \textbf{ベクトル方程式の演習}}}
\fancyhead[R]{\small 年 \hspace{1cm} 組 \hspace{1cm} 番 \quad 氏名 \hspace{6cm}}
\renewcommand{\headrulewidth}{0pt}

\begin{document}

%=============================================================================
% 1枚目:直線のまとめ
%=============================================================================
\begin{multicols}{2}

%-----------------------------------------------------------------------------
% 左カラム:直線の復習
%-----------------------------------------------------------------------------
\begin{any}{1. 直線の総復習}
    直線を決定する要素は「1点」+「向き」である.
    向きの指定方法によって2パターンの式がある.
    
    \begin{tcolorbox}[colback=white, colframe=black!80, title={直線のベクトル方程式まとめ}]
        点 $A(\vec{a})$ を通り...
        \begin{enumerate}
            \item \textbf{方向ベクトル} $\vec{d}$ に\textbf{平行}:
            \[ \vec{p} = \vec{a} + t\vec{d} \]
            \item \textbf{法線ベクトル} $\vec{n}$ に\textbf{垂直}:
            \[ \vec{n} \cdot (\vec{p} - \vec{a}) = 0 \]
        \end{enumerate}
    \end{tcolorbox}
    
    \begin{center}
    \begin{tikzpicture}[scale=0.9, >=stealth]
        % Left: Parallel
        \begin{scope}[shift={(0,0)}]
            \draw[thick, gray] (0,0) -- (3,1.5);
            \draw[->, thick, printBlue] (0.5, 0.25) -- (1.5, 0.75) node[midway, below right] {$\vec{d}$};
            \fill (0.5, 0.25) circle (1.5pt) node[left]{A};
            \node at (1.5, -0.5) {平行 ($\vec{p}=\vec{a}+t\vec{d}$)};
        \end{scope}
        
        % Right: Perpendicular (修正箇所: 座標定義を追加)
        \begin{scope}[shift={(5,0)}]
            \coordinate (Lstart) at (0,1.5);
            \coordinate (A_pt) at (1.5, 0.75);
            \coordinate (N_end) at (2.5, 2.75);
            \coordinate (Lend) at (3,0);
            
            \draw[thick, gray] (Lstart) -- (Lend);
            \draw[->, thick, printRed] (A_pt) -- (N_end) node[midway, left] {$\vec{n}$};
            \fill (A_pt) circle (1.5pt) node[below]{A};
            
            % right angle は定義した座標名を使用
            \pic [draw, angle radius=2mm] {right angle = Lstart--A_pt--N_end};
            
            \node at (1.5, -0.5) {垂直 ($\vec{n}\cdot\overrightarrow{\text{AP}}=0$)};
        \end{scope}
    \end{tikzpicture}
    \end{center}

    \begin{eg}{1 (基本演習)}
        次の直線の方程式をベクトルを用いて求め, 最終的に $ax+by+c=0$ の形で答えよ.
        \begin{enumerate}
            \item 点 $(1, 2)$ を通り, ベクトル $\vec{d}=(3, -1)$ に平行な直線.
            \item 点 $(3, -4)$ を通り, ベクトル $\vec{n}=(2, 5)$ に垂直な直線.
        \end{enumerate}
        \tcblower
        \vspace{7cm}
    \end{eg}
\end{any}

%-----------------------------------------------------------------------------
% 右カラム:円の復習
%-----------------------------------------------------------------------------
\columnbreak

\begin{any}{2. 円の総復習}
    円を決定する要素は「中心」と「半径」, または「直径の両端」である.
    
    \begin{tcolorbox}[colback=white, colframe=black!80, title={円のベクトル方程式まとめ}]
        \begin{enumerate}
            \item \textbf{基本形} (中心 C($\vec{c}$), 半径 $r$):
            \[ |\vec{p} - \vec{c}| = r \iff |\vec{p} - \vec{c}|^2 = r^2 \]
            \item \textbf{直径形} (A($\vec{a}$), B($\vec{b}$) が直径):
            \[ (\vec{p} - \vec{a}) \cdot (\vec{p} - \vec{b}) = 0 \]
        \end{enumerate}
    \end{tcolorbox}
    
    \begin{eg}{2 (式の読み取り)}
        次の方程式はどのような図形を表すか.
        \[ (\vec{p} + \vec{a}) \cdot (\vec{p} - \vec{a}) = 0 \]
        \tcblower
        \textbf{ヒント:} 
        「内積が0」ということは, 何かと何かが垂直である.
        $\vec{p} + \vec{a} = \vec{p} - (-\vec{a})$ と変形できる.
        \vspace{4cm}
    \end{eg}

    \begin{eg}{3 (式の変形)}
        方程式 $|\vec{p} - \vec{a}| = 2|\vec{p} - \vec{b}|$ はどのような図形を表すか.
        両辺を2乗して計算せよ.
        \tcblower
        \vspace{5cm}
    \end{eg}
\end{any}

\end{multicols}

%=============================================================================
% 2枚目:円の接線
%=============================================================================
\newpage
\begin{multicols}{2}

%-----------------------------------------------------------------------------
% 左カラム:接線の考え方
%-----------------------------------------------------------------------------
\begin{any}{3. 円の接線の方程式}
    円の接線は, \textbf{「半径と垂直である」}という性質を持つ.
    これはまさに「法線ベクトル」の考え方がそのまま使える場面である.
    
    \begin{thm}{円の接線のベクトル方程式}
        中心 C($\vec{c}$), 半径 $r$ の円上の点 P$_0$($\vec{p}_0$) における接線の式は,
        \[ \vec{n} \cdot (\vec{p} - \vec{p}_0) = 0 \]
        ここで, 法線ベクトル $\vec{n}$ は半径ベクトル $\overrightarrow{\text{CP}_0}$ である.
        つまり:
        \[ (\vec{p}_0 - \vec{c}) \cdot (\vec{p} - \vec{p}_0) = 0 \]
    \end{thm}
    
    \begin{center}
    \begin{tikzpicture}[scale=1.0, >=stealth]
        \coordinate (O) at (0, -0.5);
        \coordinate (C) at (1, 1);
        \def\rad{1.5}
        
        % Circle
        \draw[thick, gray!50] (C) circle (\rad);
        \fill (C) circle (1.5pt) node[left] {C($\vec{c}$)};
        
        % P0
        \coordinate (P0) at ($(C) + (30:\rad)$);
        \fill (P0) circle (2pt) node[right] {P$_0$($\vec{p}_0$)};
        
        % Tangent Line
        \draw[thick, printBlue] ($(P0)!1.5!90:(C)$) -- ($(P0)!1.5!-90:(C)$);
        
        % Vectors
        \draw[->, thick, printRed] (C) -- (P0) node[midway, above left] {$\vec{n}$};
        
        % P on line
        \coordinate (P) at ($(P0) + (0, 1.2)$); % slightly off geometry for generic P
        \coordinate (P_on_line) at ($(P0)!0.7!90:(C)$);
        \fill (P_on_line) circle (2pt) node[right] {P($\vec{p}$)};
        
        \draw[->, thick, printTeal] (P0) -- (P_on_line) node[midway, right] {$\vec{p}-\vec{p}_0$};
        
        % Right angle
        \pic [draw, angle radius=3mm] {right angle = C--P0--P_on_line};
        
        \node[below, align=left] at (1, -1) {半径 $\vec{CP}_0$ がそのまま\\法線ベクトルになる};
    \end{tikzpicture}
    \end{center}
\end{any}

%-----------------------------------------------------------------------------
% 右カラム:練習
%-----------------------------------------------------------------------------
\columnbreak

\begin{eg}{4 (接線の方程式)}
    中心 C$(1, 2)$, 半径 $\sqrt{10}$ の円上の点 P$_0(4, 3)$ における接線の方程式を, ベクトルを用いて求めよ.
    
    \tcblower
    \vspace{12cm}
\end{eg}

\end{multicols}

%=============================================================================
% 3枚目:確認テスト(問題)
%=============================================================================
\newpage
\fancyhead[L]{\small \textcolor{black!90}{数学C $>$ 第1章--平面ベクトル $>$ 第12回--\textbf{演習テスト}}}
\begin{multicols}{2}

\begin{any}{演習テスト (A: 基本)}
    \begin{prac}{A1 (直線の決定)}
        以下の直線の方程式を $ax+by+c=0$ の形で求めよ.
        \begin{enumerate}
            \item 点 A$(-2, 1)$ を通り, ベクトル $\vec{d}=(4, 3)$ に平行な直線.
            \item 点 A$(3, 5)$ を通り, 直線 $2x-y+4=0$ に垂直な直線. \\
            (\textbf{ヒント:} 元の直線の法線ベクトルを利用する)
        \end{enumerate}
    \end{prac}
    \begin{answer}[height=6cm]
    \end{answer}

    \begin{prac}{A2 (円の決定)}
        2点 A$(2, 1)$, B$(-4, 3)$ を直径の両端とする円の方程式を求めよ.
        (ベクトルを用いて立式し, $x, y$ の式で表せ)
    \end{prac}
    \begin{answer}[height=6cm]
    \end{answer}
\end{any}

\columnbreak

\begin{any}{演習テスト (B: 応用)}
    \begin{prac}{B1 (接線の方程式)}
        円 $(x-1)^2 + (y+2)^2 = 25$ 上の点 P$_0(5, 1)$ における接線の方程式を, ベクトルの内積を用いて求めよ.
    \end{prac}
    \begin{answer}[height=6cm]
    \end{answer}

    \begin{prac}{B2 (軌跡の特定)}
        $|\vec{p} + 2\vec{a}| = |\vec{p} - 4\vec{a}|$ を満たす点 P($\vec{p}$) はどのような図形を描くか.
        「2点 $\bigcirc, \bigcirc$ を結ぶ線分の...」という形で答えよ.
    \end{prac}
    \begin{answer}[height=8cm]
    \end{answer}
\end{any}

\end{multicols}

%=============================================================================
% 4枚目:確認テスト(解答)
%=============================================================================
\newpage
\fancyhead[L]{\small \textcolor{black!90}{数学C $>$ 第1章--平面ベクトル $>$ 第12回 \textbf{【解答解説】}}}

\begin{multicols}{2}

\begin{any}{解答 (例題)}
    \begin{eg}{1 解答}
        (1) 平行 $\to$ 方向ベクトル.
        $x = 1+3t, \ y=2-t$.
        $t = 2-y$ を代入して $x = 1+3(2-y) \implies \boldsymbol{x+3y-7=0}$.
        
        (2) 垂直 $\to$ 法線ベクトル.
        $2(x-3) + 5(y-(-4)) = 0$.
        $2x-6 + 5y+20 = 0 \implies \boldsymbol{2x+5y+14=0}$.
    \end{eg}
    
    \begin{eg}{3 解答}
        両辺を2乗: $|\vec{p}-\vec{a}|^2 = 4|\vec{p}-\vec{b}|^2$.
        $|\vec{p}|^2 - 2\vec{p}\cdot\vec{a} + |\vec{a}|^2 = 4(|\vec{p}|^2 - 2\vec{p}\cdot\vec{b} + |\vec{b}|^2)$.
        $3|\vec{p}|^2 - 8\vec{p}\cdot\vec{b} + 2\vec{p}\cdot\vec{a} + 4|\vec{b}|^2 - |\vec{a}|^2 = 0$.
        (計算が複雑になる場合は内分・外分を利用)
        
        $|\vec{p}-\vec{a}| : |\vec{p}-\vec{b}| = 2 : 1$.
        これは線分 AB を $2:1$ に内分する点と外分する点を直径の両端とする円(アポロニウスの円).
    \end{eg}
    
    \begin{eg}{4 解答}
        中心 C$(1, 2)$, 接点 P$_0(4, 3)$.
        法線 $\vec{n} = \overrightarrow{\text{CP}_0} = (4-1, 3-2) = (3, 1)$.
        接線は $\vec{n} \cdot (\vec{p} - \vec{p}_0) = 0$ より,
        $3(x-4) + 1(y-3) = 0 \implies 3x - 12 + y - 3 = 0$.
        $\boldsymbol{3x + y - 15 = 0}$.
    \end{eg}
\end{any}

\columnbreak

\begin{any}{解答 (演習テスト)}
    \begin{prac}{A1 解答}
        (1) $x = -2+4t, y = 1+3t \implies \frac{x+2}{4} = \frac{y-1}{3}$.
        $3(x+2) = 4(y-1) \implies \boldsymbol{3x - 4y + 10 = 0}$.
        
        (2) 「直線 $2x-y+4=0$ に垂直」ということは, この直線の法線ベクトル $(2, -1)$ が, 求める直線の\textbf{方向ベクトル}になる.
        逆に, 求める直線の法線ベクトルは $(1, 2)$ となる.
        $1(x-3) + 2(y-5) = 0 \implies \boldsymbol{x + 2y - 13 = 0}$.
    \end{prac}

    \begin{prac}{A2 解答}
        $(\vec{p} - \vec{a}) \cdot (\vec{p} - \vec{b}) = 0$ より,
        $(x-2)(x-(-4)) + (y-1)(y-3) = 0$.
        $(x-2)(x+4) + (y-1)(y-3) = 0$.
        $x^2 + 2x - 8 + y^2 - 4y + 3 = 0$.
        $\boldsymbol{x^2 + y^2 + 2x - 4y - 5 = 0}$.
        (中心 $(-1, 2)$, 半径 $\sqrt{10}$ の円)
    \end{prac}

    \begin{prac}{B1 解答}
        中心 C$(1, -2)$, 接点 P$_0(5, 1)$.
        法線ベクトル $\vec{n} = \overrightarrow{\text{CP}_0} = (5-1, \ 1-(-2)) = (4, 3)$.
        接線の方程式は,
        $4(x - 5) + 3(y - 1) = 0$.
        $4x - 20 + 3y - 3 = 0$.
        $\boldsymbol{4x + 3y - 23 = 0}$.
    \end{prac}

    \begin{prac}{B2 解答}
        $|\vec{p} - (-2\vec{a})| = |\vec{p} - 4\vec{a}|$.
        これは, 2点 $-2\vec{a}$ と $4\vec{a}$ からの距離が等しい点の集合である.
        よって, \textbf{線分} $\mathbf{-2\vec{a}, 4\vec{a}}$ \textbf{を結ぶ線分の垂直二等分線}.
        
        (補足: この線分の中点は $\vec{a}$. すなわち点 A を通り, ベクトル $\vec{a}$ に垂直な直線となる)
    \end{prac}
\end{any}

\end{multicols}
\end{document}